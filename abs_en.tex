%\begin{center}
%\large \bf \runtitle
%\end{center}
%\vspace{1cm}
\chapter*{\runtitle}

\cnote{Reescribir en consonancia con el abstract en castellano.}Distributed software resulting from emerging paradigms such as \emph{service-oriented computing} (SOC) and Cloud/Fog computing are transforming the world of software systems, giving impulse to what is called the API's economy. The underlying idea of it is that it is possible to construct software artifacts by composing services provided by third parties and previously registered in repositories. Applications running over globally available computational resources and communication infrastructure dynamically and transparently reconfigured, at run-time, by the intervention of a dedicated middleware capable of discovering and binding a running application with a certain requirement, to a service capable of fulfilling it.
In general the most important aspects of an API's behaviour are documented informally limiting the possibility of achieving SOC's utopia: automatic brokering of services. In this quest, thus, a key element is the existence of formal languages, together with associated analysis techniques, capable of fully expressing the API behavioral contract.
The way in which these formalisms are defined prescribe that correctness of the communication usually reduced to the absence of certain configurations (\emph{deadlock}, \emph{unspecified reception}, and \emph{orphan message}) can only be asserted in the presence of all the participants involved. Thus, when used for formalising an inter operability check over a communication channel with more than two require points, all participants must commit to be bound even when they may not be required at the beginning of the communication.
\emph{Generalised Multi-party Compatibility} (GMC) is a sufficient condition for the correctness condition detailed above
In this work we study:
\begin{inparaenum}[1)]
 \item a new class of CFSMs, called \emph{Multichannel Communicating Finite State Machines -- mCFSMs}, with an explicit definition of the communication channels enabling, for a participant, the possibility of having more than one channel with the other participants
 \item a definition of the GMC property for systems of mCFSMs, 
 \item a class of \emph{Asynchronous Communicating Automatas -- ACAs} with the capability of internalising the communication as read / write operations on internal buffers, enabling partial composition of communicating automata, and
 \item a method for mapping an ACA to a mCFSM providing a checking mechanism of the GMC property for the class of ACA.
 \end{inparaenum}


\bigskip

\noindent\textbf{Keywords:} Automata, CFSM, SOC, GMC, formalisms.