%!TEX root = ./main.tex
\newpage
\chapter{Conclusiones y Trabajo futuro} 
\label{conclusiones}

En esta sección resumimos las contribuciones de esta tesis junto con conclusiones y posibles futuras líneas de investigación.


\section{Conclusiones} 

El objetivo de esta tesis era expandir sobre los modelos formales para servicios, dado que consideramos que son necesarios para la implementación de SOC en toda su capacidad. Partiendo del ideal del paradigma SOC, es decir un contexto que permita biniding en tiempo de ejecución, el objetivo principal fue crear un formalismo que soporte binding parcial. Con este objetivo en mente desarrollamos los AFCA que permiten absorber el envío de mensajes como comportamiento interno cuando algunas de las componentes que participan de una comunicación se ecuentran conectadas. Esto se hace a través del mecanismo de composición que además preserva la comunicación externa de ambos componentes hacia otros y los canales por donde se envían los mensajes. Entonces al componer toda interacción entre estos componentes que era invisible a participantes ajenos, sigue estando sin modificarse. Las mCFSMs, una extensión de las CFSMs con múltiples canales entre cada par de participantes, cumplen el rol de expresar la interfaz externa de comunicación de ese tipo de autómatas para expresar componentes. Para garantizar la correcta interacción entre participantes extendimos la noción de GMC para mCFSMs. Por último demostramos que, dado un conjunto de AFCA y sus respectivas mCFSMs, la semántica de la comunicación internalizada al componer los AFCA es equivalente a la semántica comunicacional de las mCFSMs. 



