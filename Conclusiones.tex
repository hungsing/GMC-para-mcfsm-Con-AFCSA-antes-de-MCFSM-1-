%!TEX root = ./main.tex
\newpage
\chapter{Conclusiones y Trabajo futuro} 
\label{conclusiones}

En esta sección resumimos las contribuciones de esta tesis junto con conclusiones y posibles futuras líneas de investigación.


\section{Conclusiones} 

El objetivo de esta tesis era expandir sobre los modelos formales para servicios, dado que consideramos que son necesarios para la implementación de SOC en toda su capacidad. Partiendo del ideal del paradigma SOC, es decir un contexto que permita binding en tiempo de ejecución, el objetivo principal fue crear un formalismo que soporte binding parcial. Con este objetivo en mente desarrollamos los AFCA que permiten absorber el envío de mensajes como comportamiento interno cuando algunas de las componentes que participan de una comunicación se ecuentran conectadas. Esto se hace a través del mecanismo de composición que además preserva la comunicación externa de ambos componentes hacia otros y los canales por donde se envían los mensajes. Entonces al componer, toda interacción entre estos componentes que era invisible a participantes ajenos, sigue estando sin modificarse.\\

Para expresar la interfaz externa de comunicación de ese tipo de autómatas que expresan componentes utilizamos mCFSMs, una extensión de las CFSMs con múltiples canales entre cada par de participantes. Como método para garantizar que los distintos participantes puedan interactuar entre sí en forma correcta Extendimos la noción de GMC para mCFSMs. Por último pudimos demostrar que la equivalencia de semántica comunicacional entre un conjunto de AFCA y sus respectivas mCFSMs. Nuestro lenguaje internaliza el comportamiento externo en la composición, la comunicación entre un conjunto de componentes, cuando se componen se transforma en comunicación interna del AFCA resultante. La semántica de comunicación interna del AFCA compuesto es equivalente a la semántica de comunicación del conjunto de mCFMs. Esto significa que el mecanismo de composición de nuestro lenguaje preserva la semántica de comunicación original de los componentes.

\section{Trabajo futuro}

Si bien el lenguaje soporta binding parcial, queda por estudiar si la composición parcial habilitada por el lenguaje formal resulta equivalente a la composición cuando se realiza en forma total.

