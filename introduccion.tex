%!TEX root = ./main.tex
\section{Introducción}
La teoría de autómatas surge como un campo interdiscipilnario, con raíces en distintas áreas de la ciencia, incluyendo matemática pura, electrónica y ciencias de la computación. La teoría elemnental de autómatas permite tanto la especificación como la verificación de propiedades simples de secuencias finitas de símbolos. 

La tendencia hacia los sistemas distribuidos y las redes a medida que va creciendo genera la necesidad de protocolos de comunicación más complejos. Para manejar esta complejidad se han ido introduciendo métodos formales de especificación y análisis. 

Service oriented computing (SOC) es un paradigma reciente que encara la computación en "computadoras globales", infraestructuras computacionales disponibles a nivel global que corren aplicaciones de software que pueden descubrir y bindearse dinámicamente, en tiempo de ejecución, con servicios ofrecidos por proveedores. 