\documentclass[paper=a4, fontsize=11pt, spanish]{scrartcl} % A4 paper and 11pt font size
\usepackage[T1]{fontenc} % Use 8-bit encoding that has 256 glyphs
%\usepackage{fourier} % Use the Adobe Utopia font for the document - comment this line to return to the LaTeX default
\usepackage[spanish]{babel} % English language/hyphenation
\selectlanguage{spanish}
\usepackage[utf8]{inputenc}
\usepackage{amssymb}
\usepackage{amsfonts,amsmath,amsthm} % Math packages
\usepackage{mathabx}
\usepackage{mathtools}
\usepackage{tikz}
\usetikzlibrary{babel,automata,positioning,arrows}
\usepackage{graphicx}
\usepackage[rightcaption]{sidecap}
\usepackage{pgfplots}
\pgfplotsset{compat=newest}
\usepackage{xcolor}
\usepackage{paralist}
\usepackage[shortlabels]{enumitem}
% \usepackage{txfonts}
% \usepackage{framed,color}
% \usepackage{todonotes}
\newcommand*{\corner}{\mbox{\LARGE{$\mathrlap{\cdot}\lrcorner$}}}
\usepackage{sectsty} % Allows customizing section commands
\allsectionsfont{\centering \normalfont\scshape} % Make all sections centered, the default font and small caps

\usepackage{xspace}
\usepackage{xcolor}
\usepackage{framed,color}
\usepackage{todonotes}


\usepackage{fancyhdr} % Custom headers and footers
\pagestyle{fancyplain} % Makes all pages in the document conform to the custom headers and footers
\fancyhead{} % No page header - if you want one, create it in the same way as the footers below
\fancyfoot[L]{} % Empty left footer
\fancyfoot[C]{} % Empty center footer
\fancyfoot[R]{\thepage} % Page numbering for right footer
\renewcommand{\headrulewidth}{0pt} % Remove header underlines
\renewcommand{\footrulewidth}{0pt} % Remove footer underlines
\setlength{\headheight}{13.6pt} % Customize the height of the header

\numberwithin{equation}{section} % Number equations within sections (i.e. 1.1, 1.2, 2.1, 2.2 instead of 1, 2, 3, 4)
\numberwithin{figure}{section} % Number figures within sections (i.e. 1.1, 1.2, 2.1, 2.2 instead of 1, 2, 3, 4)
\numberwithin{table}{section} % Number tables within sections (i.e. 1.1, 1.2, 2.1, 2.2 instead of 1, 2, 3, 4)

\setlength\parindent{0pt} % Removes all indentation from paragraphs - comment this line for an assignment with lots of text
\newtheorem{example}{Example}
\newtheorem{definition}{Definición}
\newtheorem{prop}{Proposición}
\newcommand{\osred}{\rightarrow}                % one-step reduction relation
\newcommand{\TRANS}[1]{\xrightarrow{#1}}
\newcommand{\R}{\osred}
\newcommand{\msred}{\osred^\ast}     % multi-step reduction
\newcommand{\RR}{\msred}
\renewcommand{\vec}[1]{\mathbf{#1}}
\newcommand{\SGV}{N}
\newcommand{\RSN}[2]{\RS_#1(#2)}
\newcommand{\RSB}[1]{\RSN{1}{#1}}
\newcommand{\TSdelta}{\hat{\delta}}

\newcommand{\sgs}{n}
\newcommand{\evset}{\mathcal{E}}
\newcommand{\evtequiT}[1]{\evtequi}
\newcommand{\evtequi}{\bowtie}
\newcommand{\trsysL}[1]{\xrightarrow{#1}}
\newcommand{\qst}{\;\colon\;} %such that
\newcommand{\STSE}{\mathit{TS}} % Synchronous TS abbrev
\newcommand{\SG}[1]{\STSE(#1)} % Synchronous TS (retro compa)
\newcommand{\trstabL}[1]{\stackrel{#1}{\Rightarrow}}
\newcommand{\trstab}{\trstabL{}}
\newcommand{\tra}{\pi}




\newcommand{\Q}{\mathbf{\mathsf{Q}}}
\newcommand{\C}{\mathbf{\mathsf{C}}}
\newcommand{\msgs}{\mbox{\textcolor{blue}{$\mathsf{Msg}$}}}
\newcommand{\amsg}[1][m]{\textcolor{blue}{\mathsf{#1}}}
\newcommand{\aport}[1][{}]{\textcolor{blue}{\pi^{#1}}}

\newcommand{\grmeq}{\; ::= \;}
\newcommand{\grmor}{\; \big| \;}
\newcommand{\const}[1]{\textcolor{orange}{\mathsf{#1}}}
\newcommand{\mykeyword}[1]{\textsf{\textcolor{orange}{\upshape \textbf{#1}}}}
\newcommand{\fvar}[1]{\mathrm{fv}\big(#1\big)}
\newcommand{\truek}{\mykeyword{true}}
\newcommand{\falsek}{\mykeyword{false}}
\newcommand{\ENTAILS}{\supset}
\newcommand{\iv}[1][x]{\textcolor{orange}{#1}}
\newcommand{\bool}{\textsf{Bool}}
\newcommand{\intTp}{\mathsf{Int}}



% ------------
% theorems

%\newtheorem{example}{Example}
%\newtheorem{definition}{Definition}
\newtheorem{observation}{Note}
% end theorems
% ------------

\newcommand{\actset}{A}

\newcommand{\arn}{\ensuremath{\sf ARN}\xspace}
\newcommand{\arnint}[1]{\ensuremath{\mathcal{I}_{#1}}}

\newcommand{\anarn}{\ensuremath{\alpha =  \left(X, P, C, \gamma, M_X, M_C, \mu, \Lambda\right)}\xspace}
\newcommand{\Sit}{\textcolor{blue}{\ensuremath{\mathit{S}}}}
\newcommand{\vectorize}[1]{\ensuremath{\overset{\rightarrow}{\mathbf{#1}}}}
\newcommand{\sysconf}[2]{\ensuremath{(\vectorize{#1}, \vectorize{#2})}}

%\newcommand{\cfsmalphabet}{\ensuremath{\mathcal{A}}}
\newcommand{\cfsmalphabet}{\msgs}
\newcommand{\cfsmfinitewords}{\ensuremath{\cfsmalphabet^\star}}

\newcommand{\machine}[1]{\ensuremath{\mathsf{#1}}\xspace}
\newcommand{\machp}{\machine{p}}
\newcommand{\machq}{\machine{q}}
\newcommand{\machs}{\machine{s}}
\newcommand{\machr}{\machine{r}}
\newcommand{\machx}{\machine{x}}
\newcommand{\machc}{\machine{c}}
\newcommand{\macha}{\machine{a}}

\newcommand{\FS}{\ensuremath{\mathcal{F}}\xspace}
\newcommand{\aut}[1]{\ensuremath{\textgoth{#1}}\xspace}

\newcommand{\reduct}[2]{\ensuremath{#1\hspace{-0.4em}\downharpoonccw_{#2}}\xspace}

\newcommand{\Participants}{\mathcal{P}}

\newcommand{\PSet}{\textcolor{orange}{\mathsf{P}}}
\newcommand{\chanset}{C}
\newcommand{\ptp}[1]{\mathsf{#1}}
\newcommand{\p}{\ptp p}
\newcommand{\q}{\ptp q}
\newcommand{\s}{\ptp s}
\newcommand{\rr}{\ptp r}
\newcommand{\ASigma}{\Sigma}
\newcommand{\action}{\ell}
\newcommand{\silent}{\tau}
\newcommand{\chan}[1]{\ptp{#1}}
\newcommand{\PSEND}[2]{\chan{#1} \oact \msgsort{#2}}
\newcommand{\PRECEIVE}[2]{\chan{#1} \iact \msgsort{#2}}
\newcommand{\oact}{\ptp !}
\newcommand{\iact}{\ptp ?}
\newcommand{\msgsort}[1]{\mathsf{#1}}
\newcommand{\Act}{\mathit{Act}}
\newcommand{\Inter}{\mathit{Int}}

\newcommand{\Async}[1]{\mathcal{A}(#1)}
\newcommand{\Sync}[1]{\mathcal{S}(#1)}

\newcommand{\tset}{\to}
\newcommand{\tguard}{\mathcal{G}}
\newcommand{\GSet}{\mathbb{G}}
\newcommand{\Nats}{\mathbb{N}}
\newcommand{\VSet}{\mathcal{V}}
\newcommand{\incfunc}{\mathcal{I}}
\newcommand{\Kfunc}{\mathcal{K}}

\newcommand{\trans}[2][{}]{\,\xrightarrow{#2}_{#1}\,}
\newcommand{\csconf}[2]{\conf{\vec{#1} \ ; \ \vec{#2}}}

\newcommand{\gcsconf}[4]{\conf{\vec{#1} \ ;\  \vec{#2} \ ; \ \vec{#3} \ ; \ \mathcal{#4}}}

\newcommand{\cscfg}[2]{\conf{{#1} \ ; \ {#2}}}
\newcommand{\gcscfg}[3]{\conf{{#1} \ ; \ {#2}\ ; \ {#3}}}

\newcommand{\ith}[2]{\vec{#1}[\ptp{#2}]}
\newcommand{\conf}[1]{\langle #1 \rangle}
\newcommand{\TRANSS}[1]{{\xRightarrow{\raisebox{-.3ex}[0pt][0pt]{\scriptsize $#1$} }}}
\newcommand{\Smv}{\vec{S}}
\newcommand{\Conf}[1]{\mathsf{C}_{#1}}

\newcommand{\instantiate}{\mathit{inst}}

\newcommand{\NUL}{\varepsilon}
\newcommand{\RS}{\mathbf{RS}}
\newcommand{\st}{\ \big| \ }
\newcommand{\defref}[2][{}]{Definition{#1}~\ref{#2}}
\newcommand{\secref}[2][{}]{Section{#1}~\ref{#2}}
\newcommand{\gnodes}{V}
\newcommand{\gedges}{A}
\newcommand{\glabel}{\Lambda}
\newcommand{\LSet}{\mathsf{L}}
\newcommand{\mmdef}{:=}
\newcommand{\asend}[3]{\ptp{#1} \rightarrowtriangle\ptp{#2} : \ptp{#3}}
\newcommand{\spipe}{\; | \;}

\newcommand{\ccsprc}[1]{\mathsf{#1}}
\newcommand{\Cell}{\ccsprc{Cell}}
\newcommand{\Buffer}{\ccsprc{Buffer}}
\newcommand{\Ping}{\ccsprc{Ping}}
\newcommand{\Pong}{\ccsprc{Pong}}
\newcommand{\ccsin}[2]{\mathsf{#1}?{\mathsf{#2}}}
\newcommand{\ccsout}[2]{{\mathsf{#1}!{\mathsf{#2}}}}

\newcommand{\efb}[2]{\mathcal{E}(#1,#2)}

\newcommand{\chosyn}{\fxfatal{chosyn command is not defined }}

\newcommand{\context}{\textit{context}\xspace}
\newcommand{\powerset}[1]{\ensuremath{2^{#1}}\xspace}
\newcommand{\defname}[1]{\textbf{\emph{[#1]}}}
\newcommand{\theoname}[1]{\textbf{\emph{[#1]}}}
\newcommand{\propname}[1]{\textbf{\emph{[#1]}}}
\newcommand{\romboeqb}{\underline{\Diamond}}
\newcommand{\romboeqn}{\underline{\blacklozenge}}

\pgfkeys{/pgfplots/Axis Style/.style={
width=7.5cm, height=8cm,
axis x line=center, 
axis y line=middle, 
samples=100,
ymin=-1.5, ymax=5.5,
xmin=-7.0, xmax=7.0,
domain=-2*pi:2*pi,
title style={at={(1,0.5)}}}}

\begin{document}
%!TEX root = ./main.tex


 \title{Automata for partial binding of services
	\thanks{Carlos G.\ Lopez Pombo's research is supported by Agencia Nacional de Promoción Científica y Tecnológica by grant PICT 2013-2129, and Consejo Nacional de Investigaciones Científicas y Técnicas by grant PIP 11220130100148CO.}
}

\author{Carlos G. Lopez Pombo \inst{1,2} \and Ignacio Vissani \inst{1} \and Ezequiel Davidovich Caballero \inst{1}}
\author{C.G. Lopez Pombo et. al.}
% \institute{Universidad de Buenos Aires. School of science, Department of computing. \and CONICET--Universidad de Buenos Aires. Instituto de Investigación en Ciencias de la Computación (ICC).} %\\ \email{clpombo@dc.uba.edu.ar}}
\index{Lopez Pombo, Carlos G.}
% % \index{Vissani, Ignacio}
% % \index{Davidovich Caballero, Ezequiel}


\maketitle

Distributed software resulting from emerging paradigms such as \emph{service-oriented computing} (SOC), Cloud/Fog computing and the Internet of Things are transforming the world of software systems in order to support applications able to respond and adapt to the changes of their execution environment, giving impulse to what is called the API's economy. The underlying idea of the API's economy is that it is possible to construct software artifacts by composing services provided by third parties and previously registered in repositories. This envisages a generation of applications running over globally available computational resources and communication infrastructure, which, at run-time, are dynamically and transparently reconfigured by the intervention of a dedicated middleware with the capability to discover and bind a running application with a certain requirement, to a service capable of fulfilling it, subject to the negotiation of a \emph{Service Level Agreement -- SLA}; in this way, services can collectively fulfill a certain business goal 
\cite{fiadeiro:fac_23-4}.

Under this paradigm there is no control as to the nature of the components that an application can bind to because 
software services are accessed by their API. In general APIs only describe the data type of the arguments needed to invoke the service while the most important aspects of its behaviour are documented informally resulting in a major drawback in the possibility of achieving SOC's utopia: automatic broking of services. In this quest, 
Thus, a key element is the existence of formal languages, together with associated analysis techniques, capable of fully expressing the API behavioral contract.

\emph{Asynchronous Relational Nets} \cite{fiadeiro:fase2011} is a language for describing the orchestration of services, supporting the explicit dynamic reconfiguration of a system by declaring ports behaving either as provide points or as require points, depending on their role in the service binding and execution. In \cite{vissani:wadt14-f} it was given a formal operational semantics supporting the transparent run-time composition of services. In \cite{vissani:places15}, \emph{Communicating Finite State Machine} \cite{brand:jacm-30_2} were used to label provides points and \emph{Global Graphs} \cite{denielou:esop12} to label communication channels, in order to describe a procedure for automatically checking interoperability of services.

The way in which these formalisms are defined prescribe that correctness of the communication (usually reduced to the absence of certain configurations\cite{lange:popl15}:
\begin{inparaenum}[1)]
\item \emph{deadlock}: a participant is in a state in which it can only consume messages from participants whose corresponding message queues are empty,
\item \emph{unspecified reception}: a participant is in a state in which it can only consume messages from participants which are not in the corresponding message queues, and
\item \emph{orphan message}: all participants are in a state of no outgoing transition and there is a non-empty buffer.
\end{inparaenum}
) can only be asserted in the presence of all the participants involved. Thus, when they are used for formalising an interoperability check over a communication channel with more than two require points, all participants must commit to be bound even when they may not be required at the beginning of the communication.

\emph{Generalised Multi-party Compatibility} (GMC), defined in \cite{lange:popl15}, is a sufficient condition for the correctness condition detailed above, requiring the system to satisfy:
\begin{inparaenum}[1)]
\item \emph{representability}: each trace $t$ of each participant $P$ of the system is ``represented'' by a trace $t'$ in the global transition system in which the action taken by $P$ in $t$ appear in $t'$ in the exact same order, 
\item \emph{branching property}: each branching configuration of the global transition system either consists of independent transitions (that is they all interleave) or they represent a distributed choice with a unique participant making the choice, and any other participant involved in it has a different first action on each two different branches of the choice, and 
\item for each participant $P$ involved in a choice, there cannot be a race condition between the messages that $P$ can receive (only one input message gets to $P$ during the choice.
\end{inparaenum}

In this work we study:
\begin{inparaenum}[1)]
 \item a new class of CFSMs, called \emph{Multichannel Communicating Finite State Machines -- mCFSMs}, with an explicit definition of the communication channels enabling, for a participant, the possibility of having more than one channel with the other participants
 \item a definition of the GMC property for systems of mCFSMs, 
 \item a class of \emph{Asynchronous Communicating Automatas -- ACAs} with the capability of internalising the communication as read / write operations on internal buffers, enabling partial composition of communicating automata, and
 \item a method for mapping an ACA to a mCFSM providing a checking mechanism of the GMC property for the class of ACA.
 \end{inparaenum}

% \bibliographystyle{splncs}
% \bibliography{bibdatabase}
%!TEX root = ./main.tex
\chapter{Introducción}
La teoría de autómatas es el estudio de artefactos abstractos de computación o ``máquinas''. Antes de la existencia de computadoras, Alan Turing estudió una máquina abstracta que tenía todas las capacidades de las computadoras de hoy, en el sentido de tener la capacidad de computar \cite{turing:plms-s2-42_1,turing:plms-s2-43_1}. \cnote{No tengo claro qué quiere decir esta oración.}El objetivo de Turing era describir con precisión el límite entre lo que una computadora podía hacer y lo que no. Estas conclusiones no aplican a sus máquinas abstractas sino a las máquinas reales de hoy en día \cite{ullman}.

Entre los 40s y los 50s se estudiaron máquinas más sencillas, que hoy denominamos `autómatas finitos'. Estos autómatas, originalmente pensados para modelar el funcionamiento del cerebro, resultaron ser muy útiles para una variedad de propósitos. Noam Chomsky inició el estudio de gramáticas formales \cite{chomsky:iretit-2_3}. Si bien no eran estrictamente máquinas, estas gramáticas tienen una relación muy cercana con los autómatas abstractos y hoy en día sirven como la base de algunos componentes de software importantes, incluyendo partes de compiladores.

Todos estos desarrollos teóricos tienen relación directa con lo que se estudia en ciencias de la computación. Algunos de estos conceptos, como autómatas finitos y ciertas gramáticas formales, se usan en el diseño y la construcción de distintos tipos de artefactos de software.

Los autómatas consisten de estados (representados gráficamente con círculos) y transiciones (representadas con flechas). Cuando el autómata ve el símbolo de entrada, ejecuta la transición a otro estado, de acuerdo con su función de transición que toma el estado actual y el símbolo reciente como parámetros de entrada. Un autómata finito puede tener uno o más estados finales o \emph{accepting states}. Estos estados representan el fin de la ejecución del proceso modelado por el autómata.\\

La tendencia hacia los sistemas distribuidos genera la necesidad de mecanismos de comunicación más complejos. Para manejar esta complejidad se han introducido lenguajes de especificación y métodos formales de análisis que permiten asegurar ciertas propiedades de dichos mecanismos como por ejemplo, las \emph{Communicating Finite State Machines} \cite{brand:jacm-30_2}, los \emph{Global Graphs} \cite{castagna:lmcs-8_1}, los \emph{Session types} en su gran diversidad de variantes \cite{honda:esop98,honda:popl08} y los \emph{Interface automata} \cite{dealfaro:esec-fse-01}. Tomando esto en consideración, los autómatas finitos resultan útiles como lenguaje primitivo de modelado, subyacente en muchos de los formalismos mencionados anteriormente, que permite representar algunos de estos aspectos de dichos sistemas.

Service oriented computing (SOC) es un paradigma de computación distribuida que cambió el modo en que los sistemas de software son concebidos. El corazón del paradigma son servicios que proveen elementos computacionales autónomos, independientes de la plataforma y que se ejecutan sobre una infraestructura de cómputo y comunicación existente. Estos pueden ser descritos, publicados, descubiertos y programados usando protocolos estándar para construir redes de aplicaciones que colaboran entre sí, incluso distribuidas dentro de distintas fronteras organizacionales, con la misión de, colectivamente, alcanzar un objetivo de negocios. Uno de los elementos centrales de este paradigma es que dichos elementos computacionales son procurados en tiempo de ejecución y bajo demanda; una demanda, resulta local a cada ejecución, lo que implica que no todos los servicios necesarios son al mismo tiempo y algunos, puede que ni siquiera lo sean.

Esta mirada sobre cómo un sistema de software evoluciona reconfigurándose en tiempo de ejecución pone de relieve la necesidad de contar con un lenguaje de descripción que posibilite un enfoque para la composición que sea parcial, y que el resultado de dicha composición resulte una descripción legítima de una componente; una característica que la gran mayoría de los lenguaje utilizados en la descripción de sistemas distribuidos no posee. Volveremos sobre esto en la Sec.~\ref{trabajo-relacionado} donde discutiremos otros lenguajes formales relacionados, que hemos mencionado más arriba, al final de la presente.\\

En este trabajo buscaremos definir un lenguaje formal de modelado de componentes de software con el objeto de satisfacer las necesidades mencionadas anteriormente. Este lenguaje debe satisfacer las siguientes propiedades:
\begin{inparaenum}[1.]
\item debe posibilitar la convivencia de elementos computacionales internos de una componente (transiciones que expresan cambios locales de estado) con su interfaz de comunicación (transiciones que expresan envío o recepción de mensajes),
\item debe poseer un mecanismo claro de composición que no requiera que todos los participantes, y
\item debe tener semántica de comunicación asincrónica, compatible con la semántica de los lenguajes formales conocidos como los mencionados anteriormente).
\end{inparaenum}

Para satisfacer estos objetivos definiremos una clase de autómatas finitos, a la que llamaremos \emph{Autómatas Finitos de Comunicación Asincrónica} (AFCA), que cuentan con transiciones internas (denotando cambios locales de estado) y transiciones de comunicación sobre canales de comunicación (que expresan la comunicación con otras componentes), adicionalmente, estos autómatas pueden ser compuestos internalizando la comunicación a través de la creación de buffers dedicados que permiten reemplazar los canales de comunicación. Adaptaremos el lenguaje de las Communicating Finite State Machines, a las llamaremos \emph{multichannel Communicating Finite State Machines} (mCFSM), con el objeto de que sean capaces de reflejar la interfaz de comunicación de estos autómatas. Por último, probaremos la equivalencia entre la semántica de la composición de una familia de AFCAs y la del \emph{communicating system} obtenido a partir de la familia de CFSM correspondientes a cada uno de dichos AFCAs (ver Fig.~\ref{fig:equivalencia}).\\

En la siguiente sección presentamos brevemente distintos trabajos relacionados y como esos modelos no cumplen del todo con lo que queríamos modelar. El resto de esta tesis se divide en cuatro capítulos. Primero en el capítulo~\ref{preliminares} describimos las CFSM como lenguaje y la noción de Generalized Multiparty Compatibility (GMC, \cite{lange:popl15}) como condición suficiente para que un conjunto de CFSMs formen un communicating system seguro. En el capítulo~\ref{AFCA} definimos los AFCA, su composición, y la proyección de su interfaz de comunicación. También definimos mCFSM como extensión de CFSM y proyección de la interfaz de comunicación de un AFCA, y adaptamos la propiedad de GMC para este nuevo modelo. En el capítulo~\ref{resultados} exploramos la equivalencia de la semántica de un conjunto de AFCA y el communicating system resultante de su mCFSM proyectadas, y demostramos la validez de esta propiedad. Por último, en el capítulo~\ref{conclusiones} cerramos este trabajo con las conclusiones resultantes y algunas ideas para trabajo a futuro.


\section{Trabajo relacionado}
\label{trabajo-relacionado}

\begin{itemize}
\item \emph{Communicating Finite State Machines} \cite{brand:jacm-30_2}:  Las CFSM son un modelo para protocolos de comunicación, basado en máquinas de estado finitas que representan procesos que se comunican entre vía el intercambio asincrónico de mensajes a través de canales FIFO. Las mCFSM son una versión extendida de dicho modelo con múltiples canales entre cada par de participantes y la interfaz de comunicación de los AFCA se proyecta como una mCFSM.
\item \emph{Session types} \cite{honda:esop98,honda:popl08}: Session Types es un cálculo tipado para procesos móbiles que introduce una nueva noción de tipos en la que las interacciones que incluyen múltiples participante se abstraen directamente como un escenario global. Un global type cumple el rol de un acuerdo compartido entre pares que se comunican y es la base de type checking eficiente a través de su proyección sobre participantes individuales. Las propiedades fundamentales de la disciplina de session types como seguridad communication safety, progress y session fidelity son establecidas para interacciones asincrónicas de n participantes. Esta noción de acuerdos entre participantes formalizada a través de los global types permite modelar interacciones entre participantes de una comunicación con garantía de 
\item \emph{Global Graphs} \cite{castagna:lmcs-8_1}: 
 y 
\item \emph{Interface automata} \cite{dealfaro:esec-fse-01}
\end{itemize}

Discutir el trabajo que encontramos.
%!TEX root = ./main.tex
\section{Preliminares}
\label{preliminares}
A continuación presentaremos algunas definiciones y resultados preliminares que serán de utilidad en las restantes secciones de esta tesis.

%\subsection{Definiciones básicas}

\begin{definition}[Sistema de transición etiquetado, \cite{keller:cacm-19_7}]
Una estructura $S = \langle Q, \Sigma, \longrightarrow \rangle$ se dice que es un \emph{sistema de transición etiquetado} si satisface las siguientes propiedades:
\begin{itemize}
\item $Q$ es un conjunto finito llamado \emph{conjunto de estados},
\item $\Sigma$ es un conjunto finito llamado \emph{conjunto de etiquetas}, y
\item $\longrightarrow \subseteq Q \times \Sigma \times Q$.
\end{itemize}
Sean $q, q' \in Q$, $a \in \Sigma$, cuando $\langle q, a, q' \rangle \in \longrightarrow$ lo denotaremos como $q \xrightarrow{a} q'$. Denotaremos $\Sigma^*$ al conjunto de secuencias (finitas o infinitas) $\sigma_0, \sigma_1, \sigma_2, \ldots, \sigma_n, \ldots$ tal que para todo $i \in \nat$, $\sigma_i \in \Sigma$. A su vez, dado $q \in Q$, definiremos $\Sigma^* [q]$, llamado el lenguaje de $q$, como $\{ \sigma_0, \sigma_1, \ldots, \sigma_n, \ldots \in \Sigma^*\ |\ \text{existe } \{q_i\}_{i \in \nat} \text{ tal que } q = q_0 \text{ y para todo } i \in \nat \text{, } q_i \xrightarrow{\sigma_i} q_{i+1}\}$. Dados $q, q' \in Q$ y $\sigma \in \Sigma^*$, $q \xrightarrow{\sigma} q'$ si y solo si $\sigma = \sigma_0, \sigma_1, \ldots, \sigma_n$ y existen $\{q_i\}_{0 \leq i \leq n} \subseteq Q$ tal que:
\begin{inparaenum}[1-]
\item $q_0 = q$,
\item $q_n = q'$, y
\item para todo $0 \leq i < n$, $q_i \xrightarrow{\sigma_i} q_{i+1}$.
\end{inparaenum}
\end{definition}

%\begin{definition}[Relación de equivalencia] Es una relación binaria reflexiva, simétrica y transitiva. La relación ``es igual a''  es el ejemplo canónico donde para todo trío de objetos a, b y c, vale: 
%\begin{enumerate}
%    \item a = a (propiedad reflexiva)
%    \item Si a = b entonces b = a (propiedad de simetría)
%    \item Si a = b y b = c entonces a = c (propiedad transitiva)
%\end{enumerate}
%\end{definition}
%

En lo subsiguiente introduciremos la noción de equivalencia entre estados que usaremos a lo largo del presente trabajo. Intuitivamente, solo deseamos que dos estados sean distinguibles si tienen la capacidad ser diferenciados por alguna continuación a partir de ellos. A continuación daremos una primera aproximación a esta noción de equivalencia.

\begin{definition}[Bisimulación \cite{milner89}, Chap.~4, Def.~1] Dado un sistema de transición etiquetado $ S = \langle Q, \Sigma, \delta \rangle $, una \emph{bisimulación} es una relación binaria $R \subseteq Q \times Q$, tal que: para cada par de elementos $p, q \in Q$ tal que $\langle p, q \rangle \in R$ vale:
\begin{itemize}
    \item $(\forall \sigma \in \Sigma)(p \xrightarrow{\sigma} p' \implies (\exists q' \in Q)(q \xrightarrow{\sigma} q' \land \langle p', q' \rangle \in R))$ y, simétricamente, vale 
    \item $(\forall \sigma \in \Sigma)(q \xrightarrow{\sigma} q' \implies (\exists p' \in Q)(p \xrightarrow{\sigma} p' \land \langle p', q' \rangle \in R))$
\end{itemize}
Dados dos estados $p, q \in Q $, $p$ es bisimilar a $q$, si existe una bisimulación $R$ tal que $\langle p, q \rangle \in R$.
\end{definition}

Si observamos la definición anterior, dos estados que pertenecen a la relación de bisimulación no tienen posibilidad de ser distinguidos bajo ninguna posible sucesión de transciones comenzando en ellos. Si bien esta definición caracteriza una relación de equivalencia muy útil para razonar sobre sistemas de software, en nuestro caso no será suficiente dado que el tipo de autómatas con los que trabajaremos poseen etiquetas de diferente naturaleza. Para poder razonar sobre ellos se introducen las siguientes definiciones.

En \cite[Sec.~5.1, Def.~5]{milner89} Milner nos da una definición de \emph{bisimulación débil} basada en la eliminación de las transiciones silenciosas, o transiciones $\epsilon$ (ver \cite[Defs.~1 a~4]{milner89}). En nuestro caso, adaptaremos dicha definición para hacerla paramétrica en el conjunto de transiciones que deseamos silenciar a los efectos de determinar bisimilaridad débil.

\begin{definition}
Sea $\Sigma$ un conjunto de etiquetas, $\Sigma' \subseteq \Sigma$ y $\sigma \in \Sigma^*$, se define $\widehat{\sigma}_{\Sigma'}$ como la secuencia de etiquetas obtenida a partir de eliminar de $\sigma$ toda ocurrencia de etiquetas en $\Sigma'$.
\end{definition}

\begin{definition}[Bisimulación débil] Dado un sistema de transición etiquetado $ S = \langle Q, \Sigma, \delta \rangle $ y $\Sigma' \subseteq \Sigma$ el conjunto de etiquetas a silenciar, una \emph{bisimulación débil sobre $\Sigma'$} es una relación binaria $R \subseteq Q \times Q$, tal que: para cada par de elementos $p, q \in Q$ tal que $\langle p, q \rangle \in R$ vale:
\begin{itemize}
    \item $(\forall \sigma \in \Sigma \setminus \Sigma')(p \xrightarrow{\sigma} p' \implies (\exists q' \in Q)(\exists \sigma^* \in \Sigma^*)(q \xrightarrow{\sigma^*} q' \land \widehat{\sigma^*}_{\Sigma'} = \sigma \land \langle p', q' \rangle \in R))$ y, simétricamente, vale 
    \item $(\forall \sigma \in \Sigma \setminus \Sigma')(q \xrightarrow{\sigma} q' \implies (\exists p' \in Q)(\exists \sigma^* \in \Sigma^*)(p \xrightarrow{\sigma^*} p' \land \widehat{\sigma^*}_{\Sigma'} = \sigma \land \langle p', q' \rangle \in R))$
\end{itemize}
Dados dos estados $p, q \in Q $, $p$ es debilmente bisimilar sobre $\Sigma'$ a $q$, denotado como $p \sim_{\Sigma'} q$, si existe una bisimulación débil $R$ tal que $\langle p, q \rangle \in R$.
\end{definition}


\subsection{Comunicación Asincrónica}
Como se dijo en la primera sección, buscamos una implementación del paradigma de SOC donde la comunicación entre distintos componentes se realiza en forma asincrónica. Esto no es un requerimiento del paradigma propiamente dicho pero resulta una implementación más eficiente. El intercambio de mensajes en forma asincrónica permite a los componentes maximizar su uso de la cpaacidad de cómputo.  El intercambio de mensajes con cada componente es independiente entre sí, salvo que exista alguna necesidad de lo contrario. Para representar este comportamiento utilizamos un tipo de autómata finito denominado \emph{Communicating Finite State Machines} (llamado a partir de ahora CFSM). El concepto de CFSM fue introducido en \cite{brand:jacm-30_2} con el objetivo modelar y estudiar el comportamiento de sistemas distribuidos constituidos por un conjunto de procesos secuenciales que ejecutan concurrentemente y se comunican a partir de intercambiar mensajes a través de canales de comunicación previamente declarados. Las CFSMs son autómatas que modelan únicamente la comunicación externa entre participantes. De este modo cada CFSM representa los distintos componentes del sistema distribuido, pero solo el comportamiento que es relevante a la interacción entre los mismos. Una CFSM se puede ver como la proyección de este comportamiento específico de un autómata más complejo que tenga otro tipo de transiciones adicionales.

La naturaleza dinámica no pre programada de los sistemas SOC hace que se requiera algún mecanismo para verificar que todos los componentes puedan funcionar entre sí en forma correcta. Para esto en esta sección detallamos el concepto de \emph{Generalised multiparty compatibility} (GMC), definido originalmente en \cite{lange:popl15}. GMC se define a partir de una serie de condiciones que debe cumplir el conjunto de CFSMs participantes del sistema. Esto condiciona al sistema (o a quienes lo diseñen) a tener un conocimiento previo de los posibles sistemas que vayan a interactuar entre sí. Esto no necesariamente rompe con la idea del descubrimiento y binding en tiempo de ejecución dado que se pueden generar métodos para hacer estas comprobaciones a medida que el sistema va conectándose con los distintos participantes. %(existen ya?)%

\begin{definition}[Communicating Finite State Machines] Sea $\mathcal{M}$ un conjunto finito de mensajes y $\mathcal{P}$ un conjunto finito de participantes, definimos una CFSM sobre $\mathcal{M}$ como un sistema de transición finito $\langle Q, C, q_0, \mathcal{M}, \delta \rangle$ donde
\begin{itemize}
  \item $Q$ es un conjunto finito de estados;
  \item $C = \left\{ pq \in \mathcal{P}^2 \left|\right. p \not= q\right\}$ es un conjunto de canales
  \item $q_0 \in Q$ es el estado inicial;
  \item $\delta \subseteq Q \times (C \times \{!,?\} \times \mathcal{M}) \times Q$ es un conjunto finito de \emph{transiciones}.
  \end{itemize}
\end{definition} 

A partir de la introducción de las CFSMs, la siguiente definición introduce el concepto de \emph{Communicating System} como un conjunto de CFSMs que cumplen determinadas propiedades.

\begin{definition}[Communicating System, \cite{lange:popl15}, Def.~2.2] Dado un conjunto de mensajes $\mathcal{M}$, una CFSM $\textit{M}_p = \langle Q_p, \mathcal{C}_p, q_{0_p}, \delta_p \rangle$ para cada participante $p \in \mathcal{P}$, la tupla $S=\langle M_p \rangle_{p \in \mathcal{P}}$ es un communicating system (CS).
\end{definition}

La semántica de los CSs está dada por un sistema de transición etiquetado cuyos estados y transiciones determinan las posibles ejecuciones del conjunto de procesos sobre el que está definido el sistema.

\begin{definition}[Estados y configuraciones alcanzables]
\label{def:estadosyconf} Dado un conjunto de mensajes $\mathcal{M}$, una CFSM $\textit{M}_p = \langle Q_p, \mathcal{C}_p, q_{0_p}, \delta_p \rangle$ para cada participante $p \in \mathcal{P}$ y $S=\langle M_p \rangle_{p \in \mathcal{P}}$ un CS, el sistema de transición etiquetado que determinala semántica de $S$ (denotado como $M_S = \langle Q_S, 
{q_0}_S, \delta_S \rangle$ una configuración de S es un par $s = \langle \overrightarrow{q} ; \overrightarrow{\omega} \rangle$ donde $\overrightarrow{q} = (q_p)_{p \in \mathcal{P}}$ con $q_p \in Q_p$ y donde $\overrightarrow{\omega} = (\omega_{pq})_{pq \in C}$ con $\omega_{pq} \in \mathcal{M}^*$. La componente $\overrightarrow{q}$ es el estado de control y $q_p \in Q_p$ es el estado local de la máquina $ M_p$. La configuración inicial de S es ${q_0}_S = \langle \overrightarrow{q_0} ; \overrightarrow{\epsilon} \rangle$ con $\overrightarrow{q_0} = (q_{0_p})_{p \in \mathcal{P}}$ y $\overrightarrow{\epsilon} = (\epsilon_p)_{p \in \mathcal{P}}$.

Dadas $c' = \langle \overrightarrow{q'},\overrightarrow{\omega'} \rangle$, $c = \langle \overrightarrow{q},\overrightarrow{\omega} \rangle$ y $l \in \mathcal{C}_s \times \{!,?\} \times \mathcal{M}$, decimos que $c'$ es alcanzable desde $c$ a través de la etiqueta $l$, denotado como $\langle c, l, c' \rangle \in \delta_S$, si y sólo si:
    \begin{enumerate}
		\item $l=sr!m$ y $\langle q_s,l,q'_s\rangle \in \delta_s$ y 
			\begin{enumerate}
				\item $q'_p = q_p$ para todo $\p \neq s$; y
				\item $\omega'_{sr} = \omega_{sr} \cdot m$ y  $\omega'_{pq} = \omega_{pq}$ para todo $pq \neq sr$; o bien
			\end{enumerate}
		\item $l=sr?m$ y $\langle q_r,l,q'_r\rangle \in \delta_r$ y 
			\begin{enumerate}
			\item $q'_{p} = q_{p}$ para todo $p \neq r$; y
				\item $\omega_{sr} = m \cdot \omega'_{pq}$ y $\omega'_{pq} = \omega_{pq}$ para todo $pq' \neq sr$
			\end{enumerate}
	\end{enumerate}
El conjunto de configuraciones alcanzables de S es $RS(S) = \{q \in Q_S | \langle {q_0}_S,l_0 \ldots l_n,q \rangle \in \delta_S^* \}$ donde $\delta_S^*$ es la clausura reflexo-transitiva de $\delta_S$
\end{definition}

En adelante presentaremos definiciones y resultados que permiten expresar propiedades de los CSs introducidos en la definición anterior.

\begin{definition}[Deadlock]Sea $S$ un communicating system, una transición $t$ del mismo y $s= \langle \overrightarrow{q} ; \overrightarrow{\omega} \rangle$ con $\overrightarrow{q}= \langle q_1, \ldots, q_n \rangle$ y sea $\overrightarrow{\omega}= \langle \omega_1, \ldots, \omega_n \rangle$ una de sus configuraciones. Decimos que $s$ es una \textit{configuración de deadlock} si $\overrightarrow{\omega} = \overrightarrow{\epsilon}$, existe $r \in \mathcal{P}$ tal que $qr$ es un estado receptor y $\langle q_r,sr?a,q'_r \rangle \in \delta_r$ y para todo $p \in \mathcal{P}$, $q_p$ es un estado receptor o final. Es decir todos los canales están vacíos, hay al menos una máquina esperando un mensaje y todas las otras máquinas están en un estado final o receptor.
\end{definition}

\begin{definition}[Recepción no especificada]Sea $S$ un communicating system, una transición $t$ del mismo y $s= \langle \overrightarrow{q} ; \overrightarrow{\omega} \rangle$ con $\overrightarrow{q}= \langle q_1, \ldots, q_n \rangle$ y sea $\overrightarrow{\omega}= \langle \omega_1, \ldots, \omega_n \rangle$ una de sus configuraciones. Decimos que $s$ es una \textit{configuración de recepción no especificada} si existe $r \in \mathcal{P}$ tal que $qr$ es un estado receptor y $\langle q_r,sr?a,q'_r \rangle \in \delta_r$ implica que $|\omega_sr| > 0$ y $\omega_sr \notin a\mathcal{M}*$. Una configuración de recepción no especificada corresponde a una configuración en la que la máquina está bloqueada definitivamente por su incapacidad para recibir el mensaje que se encuentra en sus canales.
\end{definition}

\begin{definition}[Mensaje Huérfano] Sea $S$ un communicating system, una transición $t$ del mismo y $s= \langle \overrightarrow{q} ; \overrightarrow{\omega} \rangle$ con $\overrightarrow{q}= \langle q_1, \ldots, q_n \rangle$ y sea $\overrightarrow{\omega}= \langle \omega_1, \ldots, \omega_n \rangle$ una de sus configuraciones. Decimos que $s$ es una \textit{configuración de mensaje huérfano} si todos los $q_p \in \overrightarrow{q}$ son finales pero $\overrightarrow{\omega} \neq \overrightarrow{\epsilon}$. Es decir quedó un mensaje en algún canal pero no hay ninguna transición que lo reciba.
\end{definition}

\begin{definition}[Communicating System seguro] Sea $S$ un CS; se dice que $S$ es \emph{seguro} si para cada $s \in RS(S)$:
\begin{enumerate}
\item $s$ no es una configuración de deadlock 
\item $s$ no posee recepciones no especificadas, y
\item $s$ no posee mensajes huérfanos  
\end{enumerate} 

Para poder expresar esta condición de seguridad es necesario identificar conjuntos de acciones que pueden ser llevadas a cabo concurrentemente. Para esto definimos las siguientes relaciones sobre el conjunto de transiciones de una CFSM. Dados $q, q' \in Q$, se define $\mathit{act}(q,q') = \left\{\ell \ \left|\right. \ (q,\ell,q') \in \delta \right\}$ y $\Diamond, \blacklozenge \subseteq \delta \times \delta$ como las relaciones de equivalencia más pequeñas que contienen $\romboeqb$ y $\underline{\blacklozenge}$ donde:
\begin{itemize}
\item $(q_1, \ell, q_2) \underline{\Diamond} (q'_1, \ell, q'_2)$ sii $ l \notin \mathit{act}(q_1, q'_1) \land \mathit{act}(q_1, q'_1) = \mathit{act}(q_2, q'_2) \land \mathit{act}(q_2, q'_2) \neq \emptyset $
\item  $(q_1, \ell, q_2) \underline{\blacklozenge} (q'_1, \ell, q'_2)$ sii $ (q_1, \ell, q_2) \romboeqb (q'_1, \ell, q'_2) $ y $\forall(q,\ell,q') \in [(q_1, \ell, q_2) ]^{\Diamond}, \ \mathit{act}(q_1,q) = \mathit{act}(q_2,q') \land \mathit{act}(q'_1,q) = \mathit{act}(q'_2,q')$  
\end{itemize}
donde $[(q_1, \ell, q_2) ]^{\Diamond}$ es la clase de equivalencia de $(q, \ell, q')$ respecto de la relación $\Diamond$ (resp. $\blacklozenge$). Intuitivamente dos transiciones están $\blacklozenge$-relacionadas si se refieren a la misma acción aún teniendo en cuenta el interleaving.
\end{definition}

\begin{example}[Relaciones $\underline{\Diamond}$ y $\underline{\blacklozenge}$]
\label{ex:relaciones}
Consideremos la siguiente CFSM:
\begin{center}
\begin{tikzpicture}[->, thick]
 \node[state,initial] (q_0)   {$q_0$}; 
 \node[state] (q_1) [right= of q_0 ] {$q_1$};
 \node[state] (q_5) [right= of q_1 ] {$q_5$};
 \node[state] (q_2) [below= of q_0 ] {$q_2$};
 \node[state] (q_3) [right= of q_2 ] {$q_3$};
 \node[state,accepting] (q_6) [right= of q_3 ] {$q_6$};
 \draw[]        
        (q_0) edge[above] node{sr!a} (q_1)
        (q_0) edge[right] node{sr'!b} (q_2)
        (q_1) edge[right] node{sr'!b} (q_3)
        (q_1) edge[above] node{sr!a} (q_5)
        (q_2) edge[above] node{sr!a} (q_3)
        (q_2) edge[bend right, below] node{sr!c} (q_6)
        (q_3) edge[above] node{sr!a} (q_6)
        (q_5) edge[right] node{sr'!b} (q_6)
        ;
\end{tikzpicture} 
\end{center}

\begin{enumerate}
    \item \label{ex:cond1} $(q_0, sr!a,q1) \underline{\Diamond} (q_2,sr!a,q3)  $ %dado que está entrelazada con $sr'!b$
    \item \label{ex:cond2} $(q_0, sr!a,q1) \underline{\blacklozenge} (q_2,sr!a,q3) $ %por la misma razón que 1
    \item \label{ex:cond3} No vale $ ((q_0, sr!a,q1) \underline{\Diamond} (q_1,sr'!b,q5))$ %porque la transición entre $q_0$ y $q_1$ pasa a través de $sr!a$. Es decir que las dos transiciones son secuenciales no concurrentes. 
    \item \label{ex:cond4} $ (q_0, sr'!b,q2) \underline{\Diamond} (q_1, sr'!b,q3)$ %la relación e 
    \item \label{ex:cond5} No vale $((q_0, sr'!b,q2) \underline{\blacklozenge} (q_1,sr'!b,q_3)) $
\end{enumerate}

Las relaciones en Ej.~\ref{ex:relaciones}.\ref{ex:cond1}--~\ref{ex:relaciones}.\ref{ex:cond2} se sostienen dado que ambas transiciones están entrelazadas con $sr'!b$. La relacion en Ej.~\ref{ex:relaciones}.\ref{ex:cond3} no se sostiene debido a que la transición entre el origen de una $(q_0)$ y el del otro $(q_1)$ pasa por $sr!a$. Ambas transiciones en Ej.~\ref{ex:relaciones}.\ref{ex:cond3} son secuenciales, no concurrentes. La relación en Ej.~\ref{ex:relaciones}.\ref{ex:cond4} se sostiene, pero en Ej.~\ref{ex:relaciones}.\ref{ex:cond5} no porque $(q_5,sr'!b,q_6)$ está en la clase de $\Diamond$-equivalencia de $(q_0,sr'!b,q_2)$ para la cual la condición no se sostiene (debido a la transición con la etiqueta $sr!c$).
\end{example}

\begin{definition}[Eventos] Dados un conjunto de participantes $\mathcal{P}$ y un conjunto de mensajes $\mathcal{M}$ definimos un evento $e$ como es una tupla $\langle q_s, q_r, s, r, a \rangle)$ (también escrita como $\langle q_s, q_r, s \rightarrow r:a \rangle$), tal que $s,r \in \mathcal{P}$, indicando que $s$ y $r$ pueden intercambiar el mensaje a, cuando están en el estado $q_s$ y $q_r$, respectivamente.
\end{definition}

Tomando en consideración la definición anterior, para distinguir el paralelsimo a nivel máquina introducimos una relación de equivalencia sobre eventos que identifica eventos cuyas transiciones son $\blacklozenge$-equivalentes.

\begin{definition}[Equivalencia entre eventos] Definimos la equivalencia entre eventos como $\bowtie = \bowtie_s \cap \bowtie_r \subseteq \mathcal{E} \times \mathcal{E}$ donde $\mathcal{E}$ es el conjunto de eventos del sistema y se cumplen las siguientes condiciones:

\begin{itemize}
\item $(q_1, q_2, s \rightarrow r:a) \bowtie_s (q'_1, q'_2, s \rightarrow r:a) \iff$ \\ 
 $\forall (q_1, sr!a,q_3),(q'_1, sr!a,q'_3) \in \delta_s:(q_1, sr!a,q_3) \underline{\blacklozenge} (q'_1, SR!a,q'_3) $
\item $(q_1, q_2, s \rightarrow r:a) \bowtie_r (q'_1, q'_2, s \rightarrow r:a) \iff \\ 
\forall (q_2, sr?a,q_3),(q'_2, sr?a,q'_4) \in \delta_ r:(q_2, sr?a,q_4) \romboeqn (q'_2, sr?a,q'_4)$ \end{itemize}
\end{definition}
\begin{example}[Equivalencia entre eventos]
\label{ex:equiveventos}
Considere el siguiente Communicating System

\begin{tikzpicture}[->, thick]
 \node[state,initial] (q_0)   {$q_0$}; 
 \node[state] (q_1) [right= of q_0 ] {$q_1$};
  \node[state] (q_2) [below= of q_0 ] {$q_2$};
 \node[state] (q_3) [right= of q_2 ] {$q_3$};

 \draw[]        
        (q_0) edge[above] node{pr!a} (q_1)
        (q_0) edge[right] node{sp?b} (q_2)
        (q_1) edge[right] node{sp?b} (q_3)
        (q_2) edge[above] node{pr!a} (q_3)
        ; 
\end{tikzpicture} 
\qquad
\begin{tikzpicture}[->, thick]
 \node[state,initial] (q_0)   {$q_0$}; 
 \node[state] (q_1) [below= of q_0 ] {$q_1$};

 \draw[]        
        
        (q_0) edge[right] node{pr?a} (q_1)
        
        ;
\end{tikzpicture} 
\qquad
\begin{tikzpicture}[->, thick]
 \node[state,initial] (q_0)   {$q_0$}; 
 \node[state] (q_1) [below= of q_0 ] {$q_1$};

 \draw[]        
        
        (q_0) edge[right] node{sp!b} (q_1)
        
        ;
\end{tikzpicture} 

En el sistema de arriba podemos ver los siguientes eventos $(q_{0p}, q_{0r}, p \rightarrow r:a)$, $(q_{2p}, q_{0r}, p \rightarrow r:a)$, $(q_{0s}, q_{0p}, s \rightarrow p:b)$ y $(q_{0s}, q_{3p}, s \rightarrow p:b)$. Queremos ver si se cumple que $(q_{0p}, q_{0r}, p \rightarrow r:a) \bowtie (q_{2p}, q_{0r}, p \rightarrow r:a)$, es decir que son eqivalentes tanto bajo $\bowtie_p$ como en $\bowtie_r$. 

Para el primero queremos ver que $(q_{0p}, pr!a, q_{1p}) \romboeqn (q_{2p}, pr!a, q_{3p})$
Para esto necesitamos que valga $(q_{0p}, pr!a, q_{1p}) \romboeqb (q_{2p}, pr!a, q_{3p})$. Esto se cumple dado que $pr!a \notin act(q_{0p},q_{2p})$ y $act(q_{0p},q_{2p}) = act(q_{1p},q_{3p}) \neq \emptyset$. Ahora tenemos que ver la clase de equivalencia $[(q_{0p}, pr!a, q_{1p})]^{\Diamond}$. La misma es el conjunto unitario $ \{(q_{2p}, pr!a, q_{3p})\}$. Por último queremos ver que  $act(q_{0p}, q_{2p}) = 
act(q_{1p}, q_{3p}) \land act(q_{2p}, q_{2p}) = act(q_{3p}, q_{3p})$. La segunda es trivial, la primera se cumple siendo el conjunto unitario $\{sp?b\}$. Con esto demostramos que $(q_{0p}, q_{0r}, p \rightarrow r:a) \bowtie_p (q_{2p}, q_{0r}, p \rightarrow r:a)$ 

Nos queda probar que vale $(q_{0p}, q_{0r}, p \rightarrow r:a) \bowtie_r (q_{2p}, q_{0r}, p \rightarrow r:a)$. Esta equivalencia es más sencilla de probar, dado que tenemos una única transición. Entonces vemos que es trivial que $(q_{0r}, pr?a, q_{1r}) \underline{\blacklozenge} (q_{0r}, pr?a, q_{1r})$. Con esto vemos que $(q_{0p}, q_{0r}, p \rightarrow r:a)$ y $(q_{2p}, q_{0r}, p \rightarrow r:a)$ son equivalentes en $\bowtie_p$ y $\bowtie_r$, por lo tanto están en $\bowtie = \bowtie_p \cap \bowtie_r$.
\end{example} 

A continuación definimos la noción de Sistema de Transición Sincrónico, para reflejar el comportamiento de un sistema cuando envíos y recepciones son emparejados para mostrar que ocurren al mismo tiempo.

\begin{definition}[Sistema de transición sincrónico] Dados un $S = (M_P)_{p \in \mathcal{P}}$ un CS, sea $\langle N,\hat{\delta}, E \rangle$, donde: 
\begin{itemize}
    \item[] $N = \{\overrightarrow{q} \ | \ (\overrightarrow{q}; \overrightarrow{\epsilon}) \in RS_1(S) \}$,
    \item[] $\hat{\delta}= \{(n, e, n') \ | \ (n;\overrightarrow{\epsilon}) s_1 \overset{sr!a}{\longrightarrow}\overset{sr?a}{\longrightarrow} (n';\overrightarrow{\epsilon})	\land e= n[s], n[r], s \rightarrow r:a \}$, y
    \item[] $ E = \{ \exists n, n' \in N : (n,e,n') \in \hat{\delta}\} \subseteq \mathcal{E}$,
\end{itemize}
   el \emph{Sistema de Transición Sincrónico} de S es $TS(S)= \langle N, n_0, E/ \bowtie,\rightrightharpoons \rangle$ donde $n_0= \overrightarrow{q_0} $ es el estado inicial, $n \overset{[e]}{\rightrightharpoons} n' \iff (n,e,n') \in \hat{\delta}$. Fijamos un conjunto $\hat{E}$ de elementos representativos de cada clase de equivalencia $\bowtie$ (ej: $\hat{E} \subseteq E$ y $\left(\forall e \in E\right)\left(\exists!e' \in \hat{E}\right)\left(e' \in [e] \right)$) y escribimos $n \overset{e'}{\rightrightharpoons} n'$ para $ n \overset{[e]}{\rightrightharpoons} n'$ cuando $ e' \in [e] \cap \hat{E} $. Las secuencias de eventos se notan con un símbolo $\pi$ y extendemos la notación de $ \rightarrow$ en la Def.~\ref{def:estadosyconf} a $\rightrightharpoons$ (ej: $si \pi = e_1 ...e_k, n_1 \overset{\pi}{\rightrightharpoons}n_{k+1} sii n_1 \overset{e_1}{\rightrightharpoons} n_2 \overset{e_2}{\rightrightharpoons}...\overset{e_k}{\rightrightharpoons} n_{k+1}$).

$TS(S)$ representa todas las posibles ejecuciones sincrónicas del sistema $S$; y cada transición es etiquetada con un evento $e$.
\end{definition}

\begin{example}[Sistema de Transición Sincrónico] 
\label{ex:STS}
Consideremos el Ej.~\ref{ex:equiveventos}, su Sistema de Transición Sincrónico es el siguiente

\begin{tikzpicture}[->, very thick]
\node[state] (q_0) {};
\node[state] (q_1) [right= 3.5cm of q_0 ] {};
\node[state] (q_2) [below= of q_0 ] {};
\node[state] (q_3) [right= 3.5cm of q_2 ] {};
%$(q_{0p}, q_{0r}, p \rightarrow r:a)$, $(q_{2p}, q_{0r}, p \rightarrow r:a)$, $(q_{0s}, q_{0p}, s \rightarrow p:)$ y $(q_{0s}, q_{3p}, s \rightarrow p:b)$
\draw[]        
        (q_0) edge[above] node{$(q_{0p}, q_{0r}, p \rightarrow r:a)$} (q_1)++
        (q_0) edge[left] node{$(q_{0s}, q_{0p}, s \rightarrow p:b)$} (q_2)
        (q_1) edge[right] node{$(q_{0s}, q_{3p}, s \rightarrow p:b)$} (q_3)
        (q_2) edge[below] node{$(q_{2p}, q_{0r}, p \rightarrow r:a)$} (q_3)
        ; 
\end{tikzpicture}

Tenemos $(q_{0p}, q_{0r}, p \rightarrow r:a) \bowtie (q_{2p}, q_{0r}, p \rightarrow r:a)$ y $(q_{0s}, q_{0p}, s \rightarrow p:b) \bowtie (q_{0s}, q_{3p}, s \rightarrow p:b)$. Podemos considerar equivalentes los eventos de las transiciones verticales por un lado y las de las transiciones horizontales por el otro. Esto nos permite identificar un par de interacciones concurrentes, pero seguir diferenciándolas de otras instancias de comunicación $p \rightarrow r:a$ y $s \rightarrow p:b$
\end{example}

\begin{definition}[Proyecciones] La proyección de un evento e sobre un participante p, denotado por $e \downharpoonright_p$ se define de la siguiente manera:
\begin{equation}
(q_s,q_r,s \rightarrow r:a) \downharpoonright_p = \begin{cases} 
pr!a & \mathit{if} s=p \\
sp?a & \mathit{if} r=p \\
\epsilon & \mathit{en \ otro \ caso} \\
\end{cases} 
\end{equation}

La proyección se define sobre secuencias de eventos en el modo evidente. La proyección $TS(S)= (N, n_0, \hat{E}, \rightrightharpoons)$ sobre el participante p, notada $ TS(S) \downharpoonright_p $, es el autómata $(Q, q_0, \Sigma, \delta)$ donde $Q=N, \ q_0 = n_0$, $\Sigma$ es el conjunto de etiquetas y $\delta \subseteq Q \times \Sigma \cup \{ \epsilon \} \times Q $ es el conjunto de transiciones, tal que $(n_1, e \downharpoonright_p, n_2) \in \delta \iff n_1 \overset{e}{\rightrightharpoons} n_2 $
\end{definition}

Introducimos el concepto de generalized multiparty compatibility (GMC) como una condición completa y sólida para construir CFSMs. 
% Usé sólida como traducción de sound. En el paper habla de construir global graphs, término que acá no usamos, no se si puse bien
A partir de este punto, fijamos un sistema $S =(M_p)_{p \in \mathcal{P}} $ con $TS(S)= (N, n_0, \hat{E}, \rightrightharpoons)$. GMC depende de dos condiciones, representabilidad y propiedad de ramificación.\\

La propiedad de representabilidad determina que cada máquina, traza y elección estén representadas en el sistema de transición.

\begin{definition}
Para un lenguaje $\mathcal{L}$, $hd(\mathcal{L})$ devuelve las primeras acciones de $\mathcal{L}$ si las tiene: 
\begin{center}
$hd(\mathcal{L})= \{\ell \ | \ \exists q \in Act*: \ell \cdot q \in \mathcal{L} \}$	\ \	 $hd(\{ \epsilon \}) = \{ \epsilon \}$
\end{center}
Dado $n \in N$, sea $ TS(S)\langle n \rangle$ el sistema de transición TS(S) donde se reemplaza al estado inicial $n_0$ por $n$. Escribimos LT(S,n,p) para $\mathcal{L}(TS(S)\langle n \rangle)\downharpoonright_p $, es decir LT(S,n,p) es el lenguaje que se obtiene estableciendo a n como nodo inicial de TS(S) y proyectando el nuevo sistema de transición sobre p.
\end{definition}

\begin{definition}[Representabilidad]
\label{def:representabilidad}
Un sistema S es representable si
\begin{enumerate}
\item $\mathcal{L}(M_p) = LT(S,n_0,p) $ y
\item $\forall q \in Q_p \ \exists n \in N: n[p] = q \land  \cup_{(q,\ell, q') \in \delta_p} \{ \ell \} \subseteq hd (LT(S,n,p))$
\end{enumerate}
para todo $p \in \mathcal{P} $ \\

La primera condición asegura que cada traza de cada máquina esté en $TS(S)$, a su vez, la segunda condición es necesaria para asegurar que cada elección en cada máquina esté representada en $TS(S)$.
\end{definition}

La propiedad de ramificación estimula que cada vez que hay una decisión en un sistema $TS(S)$, una única máquina toma esa decisión y cada uno de los otros participantes es notificado de la rama que se tomó o no participa en esa elección. 
 
\begin{definition}[Propiedad de ramificación] Un sistema S posee la propiedad de ramificación si para todo $n \in N$ y para todo $e_1 \neq e_2 \in \hat{E} \ \mathit{tq} \ n \overset{e_1}{\rightrightharpoons} n_1 \ \mathit{y} \  \ n_1 \overset{e_2}{\rightrightharpoons} n_2$ luego tenemos
\begin{enumerate}
\item o bien existe $n'\in N$ tal que $n_1 \overset{e_2}{\rightrightharpoons} n' \ \mathit{y} \ n_2 \overset{e_1}{\rightrightharpoons} n'$, o 
\item para cada $(n'_1, n'_2) \in ln(n, e_1, e_2)$ quedando 

$L_p^i = hd (\{e_i \downharpoonright_p \cdot \phi \  | \ \phi \in LT(S,n'_i,p) \} ) $ con $i \in \{1,2\}$ y $p \in P$, y se cumplen las condiciones 2a, 2b y 2c definidas abajo 	

\begin{enumerate}[(a)]
\item Choice awareness: $\forall p \in \mathcal{P}$ valen \begin{enumerate}[i.]
\item $L_p^1 \cap	L_p^2 \subseteq \{ \epsilon \} \ \mathit{y} \  \epsilon \in L_p^1 \iff \epsilon \in L_p^2$, o 
\item $\exists n' \in N, \pi_1, \pi_2$: $n'_1 \overset{\pi_1}{\rightrightharpoons} n' \land  n'_2 \overset{\pi_2}{\rightrightharpoons} n' \land (e_1 \cdot \pi_1) \downharpoonright_p= (e_2 \cdot \pi_2)\downharpoonright_p = \epsilon$
\end{enumerate}
\item selector único: $\exists!s \in \mathcal{P}: L_s^1 \cap L_s^2 = \emptyset \land \exists sr!a \in L_s^1 \cup L_s^2 $
\item no race: $\forall r \in \mathcal{P}: L_r^1\cap L_r^2 = \emptyset \Rightarrow \forall s_1r?a_1 \in L_r^1, \forall s_2r?a_1 \in L_r^2:\forall i \neq j \in \{1,2\}: n'_i \overset{\pi_i}\rightrightharpoons  $ \\
$\Rightarrow dep (s_i \rightarrow r: a_i, e_i \cdot \pi_i, s_j \rightarrow r: a_j) $
\end{enumerate}
\end{enumerate} 	
\end{definition}

Representabilidad garantiza que TS(S) contiene suficiente información para decdir propiedades seguras de cualquier ejecución asincrónica de S. La propiedad de ramificación asegura que si una rama en TS(S) representa una elección esta está "bien formada".\\
 
Con esto definimos que toda ramificación es o bien (1) la ejecución concurrente de dos eventos; o, para cada participante p (2(a)I) si p no termina antes de n entonces las primeras dos acciones de p en dos ramas distintas son disjuntas; o (2(a)II) p no está involucrado en la elección, o sea la ramas se juntan antes de que p realice ninguna acción; (2(b)) hay un único participante s tomando la decisión; y (2(c)) para cada participante r involucrado en la elección, no puede haber race condition entre los mensajes que puede recibir r. La no race condition asegura que en ninguna ejecución (asincrónica) de S si una máquina tiene más de un buffer no vacío, entonces puede leerlos en cualquier orden (interleaving es posible). Notar que si una máquina r recibe todos sus mensajes de un mismo emisor, entonces hay una $\triangleleft$-relación entre todas sus acciones.

\begin{definition}[Generalised Multiparty Compatibilty] Un communicating system S es generalised multiparty compatible (GMC) si es representable y posee la propiedad de ramifiación. 
\end{definition}

\begin{theorem}[Solvencia]Si S es GMC entonces es seguro (no tiene configuraciones de deadlock, recepción no especificada o mensaje huérfano)
\end{theorem}

El teorema dice que ninguna ejecución asincrónica de S va a resultar en una configuración de mensaje huérfano, deadlock o recepción no especificada. Basándose en la propiedad de representabilidad (toda transición y ramificación de cada máquina está representada en TS(S)), la demostración \cite{lange:popl15} muestra que todo mensaje enviado es recibido eventualmente y que una máquina en un estado receptor eventualmente recibe el mensaje que espera, según la Def.~\ref{def:representabilidad}.


%!TEX root = ./main.tex


\section{Componentes Asincrónicas}


\subsection{Autómatas finitos de comunicación asincrónica}
Las CFSMs son un tipo de autómata que modela únicamente la comunicación externa con otros participantes del sistema. Necesitábamos un tipo de autómata que además representara otro además comportamientos internos. Para esto primero recurrimos a los Input Output Atomata (Ref IO). 
Un I/O automaton modela un componente de un sistema distribuido que puede interactuar con otros sistemas componentes. Es un tipo de máquina de estados en la cual las transiciones están asociadas con determinadas acciones específicas. Existen tres tipos de acciones: input, output, y acciones internas. El autómata usa sus acciones de input y output para comunicarse con el enotrno, mientras que las acciones internas son sólo visibles para el autómata. A diferencia de las acciones internas y las de output que son seleccionadas y resueltas por el autómata, las de entrada, que llegan del entorno, no están bajo control del autómata. Para componer estos autómatas se requiere que sincronicen las acciones compartidas. Es decir cuando un autómata ejecuta una transición con la etiqueta $\pi$, todo autómata que tenga esa acción $\pi$ la ejecuta en simultáneo.
El problema es que los IO no representan bien la comunicación asincrónica. Las operaciones de comunicación externa sincronizan a todos los participantes del sistema. Es decir que si dos autómatas componentes tienen una misma transición, se ejecutan en simultáneo. Esto va en contra de la idea de asincronsidad que buscamos.

\begin{definition}[Cola]
\label{def:cola}
Una cola es un tipo de estructura de datos caracterizada por ser una secuencia de elementos first in-first out (FIFO) debido a que el primer elemento en entrar es el primer elemento en salir. Para manejarla se definen dos operaciones, una para agregar y otra para sacar elementos de la una cola. Para los usos de este trabajo vamos a definir colas en las cuales se depositan y de las cuales se retiran mensajes, con las operaciones:

\begin{itemize}
\item Cola vacía: denotado $[\ ]$
\item Encolar mensaje: denotado como $b \ll m$, si $b$ es una cola y $m$ es un mensaje.
\item Desencolar mensaje: denotado como $b \gg m$, si $b$ es una cola y $m$ es un mensaje.
\item Sea una cola vacía $b= [\ ]$ al aplicarle la operación $b \ll m$ queda $b= [m]$
\item Sea una cola no vacía $b= [m_1,..,m_n]$ al aplicarle la operación $b \ll m$ queda $b= [m,m_1,..,m_n]$ del mismo modo al aplicarle $b \gg m_n$ queda $b= [m,m_1,..]$
\end{itemize}
\end{definition}

\begin{definition}[Autómata finitos de comunicación asincrónica]
\label{def:aa}
Sea $\mathcal{P}$ un conjunto de participantes y $\mathcal{M}$ un conjunto de mensajes. Un \emph{autómata finitos de comunicación asíncrona} es una estructura $A_\mathcal{P} = \langle Q, B, \mathcal{C}, \Sigma, \delta, q_0, F\rangle$ tal que:

\begin{itemize}
\item $Q$ es un conjunto finito de estados,
\item $B \subseteq \{pq_n \ | \ pq \in \mathcal{P}^2, n \in \mathbb{N}, p \neq q \} $ es un conjunto finito de buffers (i.e. colas, ver Def.~\ref{def:cola}),
\item $\mathcal{C} \subseteq \{pq_n \ | \ pq \in \mathcal{P}^2, n \in \mathbb{N}, p \neq q \}$ es un conjunto de canales tales que $B \cap \mathcal{C} = \emptyset$ 
\item $\Sigma = \{ \Sigma_\mathit{Int} \cup \Sigma_\mathit{Ex} \cup \Sigma_\mathit{Buff}\} $, $\Sigma \cap \mathcal{M} = \emptyset$ es el conjunto de etiquetas del autómata, siendo
\begin{inparaenum}[1)]

\item $\Sigma_\mathit{Int}$ las acciones internas del autómata 

\item $\Sigma_\mathit{Ex}$ un conjunto de etiquetas de la forma $\mathit{In}(c,m), \mathit{Out}(c,m)$ dónde $c \in \mathcal{C}$ es el canal a través del cual se resuelve la misma y $m \in \mathcal{M}$ es el mensaje .

\item $\Sigma_\mathit{Buff}$ es el conjunto de etiquetas de las acciones sobre los buffers de la forma $b \ll m$ o $b \gg m$, dónde $b \in B$ y $m \in \mathcal{M}$.
\end{inparaenum}
\item $\delta = (\delta_\mathit{Int} \cup \delta_\mathit{Ex} \cup \delta_\mathit{Buff})$ siendo:

\begin{inparaenum}[1)]
\item $\delta_\mathit{Int} \subseteq Q \times \Sigma_\mathit{Int} \times Q$ es la relación de transicion por acciones internas de $A_\mathcal{P}$, %transiciones internas

\item $\delta_\mathit{Ex} \subseteq Q \times \Sigma_\mathit{EX}  \times Q$ es la relación de transición de comunicación externa de $A_\mathcal{P}$, %comunicación externa

\item $\delta_\mathit{Buff} \subseteq Q \times \Sigma_\mathit{Buff} \times Q$ %comunicación interna
\end{inparaenum}
\item $q_0 \in Q$ es el estado inicial, y
\item $F \subseteq Q$ es el conjunto de estados finales. 
\end{itemize}

Se denota $P(A)$, al conjunto de participantes que integran el autómata A, y se define como $P(A) = \left(p \in \mathcal{P} \ | \left(\exists\ \langle p1,p2,c\rangle \in \Sigma_\mathit{Ex}\right)\left(p1=p \lor p2=p\right) \right) $ 
\end{definition}

$\Sigma$ es el conjunto de todas las etiquetas del autómata y lo dividimos en tres subconjuntos disjuntos que la forman $\Sigma_{\mathit{Int}}$, $\Sigma_{\mathit{Ex}}$ y $\Sigma_{\mathit{Buff}}$ que corresponden a las acciones de procesamiento interno, las de comunicación externa con otros participantes y las de comunicación interna vía buffers.

$\Sigma_{Ex}$ es el conjunto de etiquetas correspondientes a la comunicación externa del autómata. Como la comunicación es dirigida punto a punto, cada etiqueta incluye a dos participantes $p_1, P_2 \in \mathcal{P}$ (emisor y receptor) y un canal $c \in \mathcal{C}$. Necesariamente en todas las etiquetas vale que $p_1$ y $p_2$ son distintos y .

$\Sigma_{Buff}$ es el conjunto de etiquetas de acciones de comunicación interna vía buffers. Se especifican las acciones de encolar y desencolar como $b \ll m$, $b \gg m$, respectivamente, donde $b \in B$ y $m$ es el mensaje que se inserta/extrae del mismo.

$\delta$ es el conjunto de transciciones que se compone de $\delta_{Int}$, las acciones internas, $\delta_{Ex}$ las transiciones de comunicación con otros agentes y $\delta_{Buf}$ las acciones de buffer.

$\delta_{Ex}$ se compone de dos tipos de acciones denotadas \textit{In(c,m)} y \textit{Out(c,m)}, con $c \in \Sigma_{Ex}$ y $m \in \mathcal{M}$ . Las acciones de entrada o input que representan la recepción de un mensaje de algún proceso externo al correspondiente al autómata,  y las acciones de output o salida que representan el envío de un mensaje y se denotan. Estas acciones son la parte externa de la comunicaión asincrónica. Como los canales son unidireccionales decimos que para todo par de participantes existen dos canales, uno por cada sentido.

$\delta_{Buff}$ es la relación de transición de comunicación interna mediante buffers.\\

Formalmente, F $\subseteq$ Q. Es el conjunto de estados finales, los posibles estados a los que una ejecución llega y es aceptada como válida. Para los autómatas asincrónicos pedimos también que al llegar a este estado final los buffers se encuentren vacíos. Para esto definimos lo siguiente:

\begin{definition}[Configuración instantánea]
Dados $\mathcal{P}$ un conjunto de participantes y $\mathcal{M}$ un conjunto de mensajes. A un autómata con comunicación asincrónica $A_\mathcal{P} = \langle Q, B, \mathcal{C}, \Sigma, \delta, q_0, F\rangle$, definimos:

Una configuración instantánea del autómata como $\langle q, \Omega \rangle \in Q \times \{ [m_b]_{b \in B} | m_b \in \mathcal{M}^* \}$ donde $\mathcal{M}^* = \{ [s_o, \ldots, s_n] \  | \ \forall i \in [0, \ldots, n], s_i \in \mathcal{M} \}$ y $[m_b]_{b \in B} $

\begin{itemize}
\item  q es es el estado actual
\item $\Omega$ es el conjunto de buffers (con sus respectivos contenidos hasta el momento). Decimos que $\Omega$ es de la forma ${ \omega_{b_1}, \omega_{b_2},..., \omega_{b_n} }$ con $b_i \in B$. 
\item Decimos que una configuración es inicial si $q = q_0$ y $ \Omega = \{ [], \ldots, [] \}$
\item Decimos que una configuración es final si $q \in F$ y $ \Omega = \{ [], \ldots, [] \}$
\end{itemize}
\end{definition}

\begin{definition}[Relación de transición entre configuraciones] Sean $A_\mathcal{P} = \langle Q, B, \mathcal{C}, \Sigma, \delta, q_0, F\rangle$, $q_1, q_2 \in Q$, $ m \in \mathcal{M}$ definimos $\vdash$, relación de transición entre configuraciones, como:  
\begin{enumerate}
\item $\langle q_1, \Omega \rangle \vdash \langle q_2, \Omega \rangle \iff \langle q_1,r, q_2 \rangle \in \delta \land r \in \Sigma \setminus \Sigma_\mathit{Buff}$

\item $\langle q_1 ,\{\omega_{b_1},\ldots,\omega_{b_i},\ldots,\omega_{b_n}\} \rangle \vdash \langle q_2,\{\omega_{b_1},\ldots,m : \omega_{b_i},\ldots,b_n\}\rangle \iff \langle q_1, b_i \ll m, q_2 \rangle  \in \delta $ 

\item $\langle q_1, \{\omega_{b_1},\ldots,\omega_{b_i} : m,\ldots,\omega_{b_n}\} \rangle \vdash \langle q_2, \{\omega_{b_1},\ldots,\omega_{b_i},\ldots,\omega_{b_n} \}\rangle \iff \langle q_1, b_i \gg m, q_2 \rangle \in \delta$
\end{enumerate}

\end{definition}

% \begin{definition}[Estados y configuraciones alcanzables]
% \label{def:estadosyconAFCA}
%   Una configuración $c' = \langle q', \Omega' \rangle$ es \emph{alcanzable} desde otra  configuración $c = \langle q, \Omega \rangle$ a través de la \emph{ejecución de la transición} $l$ (escrito $c \overset{l}{\rightarrow} c'$) si existe un $m \in \mathcal{M}$ tal que ocurre una de las siguientes alternativas:
% 	\begin{enumerate}
% 		\item $l = b_i \ll m$, $\langle q, l,  q'\rangle \in \delta$, $\Omega = [\omega_{b_1},\ldots, \omega_{b_i},\ldots,b_n]$ y $\Omega' = [\omega_{b_1},\ldots,m : \omega_{b_i},\ldots,b_n]$
% 		\item $l = b_i \gg m$, $\langle q, l,  q'\rangle \in \delta$, $\Omega = [\omega_{b_1},\ldots,m : \omega_{b_i},\ldots,b_n]$ y $\Omega' = [\omega_{b_1},\ldots, \omega_{b_i},\ldots,b_n]$
% 	\end{enumerate}
% La clausura reflexoEscribimos $ s_1 \overset{t1...tm}{\rightarrow} s_{m+1}$ cuando para algún $s_2,...,s_m, s_1\overset{t}{\rightarrow} s_2...s_m\overset{t_m}{\rightarrow} s_{m+1} $ . El conjunto de configuraciones alcanzables de S es $RS(S) = \{s | s_0 \rightarrow^*S \}$
% \end{definition}

\begin{definition}[Traza de un AFCA] Llamamos traza a una secuencia posible de acciones de un autómata. Se define como una secuencia finita de etiquetas de estado y transición altenradas, que comienza y termina con un estado. Dado un autómata $A_\mathcal{P} = \langle Q, B, \mathcal{C}, \Sigma, \delta, q_0, F\rangle$, una traza tiene las siguientes caracterísitcas

\begin{itemize}
\item Tiene la forma $[q_0, \sigma_1, q_1,...,q_{n-1}, \sigma_n, q_n] $ donde 
\item $q_0$ es el estado inicial del autómata
\item $q_i \in Q$,
\item $ \sigma_i \in \Sigma$ y
\item $\langle q_{i-1}, \sigma_i, q_i \rangle \in \delta $ 
\end{itemize}

El comportamiento de un AFCA es el conjunto de todas las trazas posibles.
\end{definition}




% \begin{definition}[Condición de aceptación] Sea $\sigma$ una secuencia de transiciones sobre $A_{\mathcal{P}}$, decimos que es una ejecución válida si se cumple lo siguiente:

% \centering
% $\sigma \in E(A_{\mathcal{P}}) \iff \exists q_f \in F \ | \  (q_0, \sigma, [[],..,[]]) \vdash^*(q_f, \emptyset, [ [],..,[] ])$ 
% \end{definition}


% Entonces decimos que una ejecución aceptable es aquella que cumple que en una cantidad finita de pasos llegó a un estado final con los buffers vacíos, asegurándonos la ausencia de mensajes huérfanos y deadlocks.

\begin{definition}[Deadlock]Sean $A_\mathcal{P} = \langle Q, B, \Sigma, \delta, q_0, F\rangle$, $q_{i} \in Q$, $\sigma \in \Sigma^*$,  $\Omega = \{ \omega_{b_1},..,\omega_{b_i},..,\omega_{b_n} \}$, decimos que una configuración está en deadlock cuando ocurre:
\begin{itemize}
\item En la configuración $\langle q_i, \Omega \rangle$,

\item $ \forall \langle q_i, s, q_j \rangle \in \delta$, $s \in \Sigma_\mathit{Buff} \land (s= b \gg m \Rightarrow \omega_b = [\ ])$
\end{itemize}
\end{definition}

Decimos que ocurre deadlock cuando partiendo de un estado la única transición posible hacia un estado siguiente es consumiendo un mensaje de algún buffer y esos buffers se encuentran vacíos.  \\

\begin{definition}[Mensajes huérfanos]Sean $A_\mathcal{P} = \langle Q, B, \Sigma, \delta, q_0, F\rangle$, $q_{i} \in Q$, $\sigma \in \Sigma^*$,  $\Omega = \{ \omega_{b_1},..,\omega_{b_i},..,\omega_{b_n} \}$, decimos que una configuración tiene mensajes huérfanos cuando ocurre:
\begin{itemize}
\item En la configuración $\langle q_i, \Omega \rangle$ ,

\item $ \exists \omega_{b} \in \Omega$ tal que $ \omega_b \neq [\ ] $

\item $ \delta(q_i) = \emptyset  $
\end{itemize}
\end{definition}

Decimos que la configuración tiene mensajes huérfanos si, en un estado $q_i$  quedan mensajes sin consumir en algún buffer y no hay ninguna transición saliente. 


\begin{definition}[Receptor no especificado]Sean $A_\mathcal{P} = \langle Q, B, \Sigma, \delta, q_0, F\rangle$, $q_{i} \in Q$, $\sigma \in \Sigma^*$,  $\Omega = [ \omega_{b_1},..,\omega_{b_i},..,\omega_{b_n} ]$ decimos que una configuración es de receptor no especificado cuando ocurre:
\begin{itemize}\item En la configuración $\langle q_i, \Omega \rangle$ ,

\item $ \exists \omega_{b} \in \Omega$ tal que $\omega_b \neq [\ ] $

\item  $ \left(\forall \langle q_i, b \gg m, q_j \rangle \in \delta \right) \left(\omega_b = [ \ldots, m' ] \right)$, con $m' \neq m$
\end{itemize}
\end{definition}
Decimos que una configuración está en modo Receptor no Especificado si la única transición saliente de un estado $q_i$ consume un mensaje $m$ y el primer mensaje de la cola es un mensaje $m'$ distinto de $m$.

\begin{definition}[Ejecución de un AFCA]. Una ejecución es un $ \tau = \tau_0, \ldots,\tau_n $ donde $\tau_0$ es la configuración inicial, $\langle \tau_i, \tau_{i+1} \rangle \in \vdash$, $\tau_n$ es una configuración final y $ \forall i \in [1, \ldots, n], \tau_i$ es una configuración. Una ejecución es una configuración que no entra en deadlock, mensajes huérfanos ni receptor no especificado.
\end{definition}


\begin{example}[Ejemplo de un Autómata Finito de Comunicación Asincrónica]
\label{ex:AFCA}
Considere el AFCA $S= \langle \{q_0,..,q_5\},[b_0],\{sr_1,sr_2,rs_1\}, \Sigma_A, \delta_A, q_0, \{q_4\} \rangle$. Donde $\Sigma =\{wait, b_0 \ll m_1,b_0 \gg m_1, Out(sr_1,a),Out(sr_2,c), In(rs_1,b),In(rs_2,d) \}$
\begin{center}
\begin{tikzpicture}[->, thick]
 \node[state,initial] (q_0)   {$q_0$}; 
 \node[state] (q_1) [right= 1.5cm of q_0 ] {$q_1$};
 \node[state] (q_5) [right= 1.8cm of q_1 ] {$q_5$};
 \node[state] (q_2) [below= of q_0 ] {$q_2$};
 \node[state] (q_3) [right= 1.5cm of q_2 ] {$q_3$};
 \node[state,accepting] (q_4) [right= 1.8cm of q_3 ] {$q_4$};
 \draw[]        
        (q_0) edge[above] node{$b_0 \ll m_1$} (q_1)
        (q_0) edge[left] node{$Out(sr_1,a)$} (q_2)
        (q_1) edge[above] node{$Out(sr_2,c)$} (q_5)
        (q_2) edge[above] node{$In(rs_1,b)$} (q_3)
        (q_2) edge[bend right, below] node{$wait$} (q_4)
        (q_3) edge[above] node{$In(rs_2,d)$} (q_4)
        (q_5) edge[right] node{$b_0 \gg m_1$} (q_4)
        ;
\end{tikzpicture}
\end{center}
En el ejemplo podemos ver los tres tipos de transiciones que los AFCA pueden realizar. 

\begin{itemize}
    \item Transiciones de buffer interno: $b_0 \ll m_1$ y $b_0 \gg m_1$
    \item Transiciones de comunicación externa: $Out(sr_1,a),Out(sr_2,c), In(rs_1,b)$ y $In(rs_2,d) $
    \item Transiciones internas: $wait$
\end{itemize}


\end{example}


\subsection{Aspectos comunicacionales de los AFCA}

Una de las desventajas principales de utilizar CFSM como lenguaje de especificación de contratos en Service Oriented Computing es que la naturaleza del binding en SOC choca con el método de establecer si un conjunto de participantes pueden interacturar libres de errores de comunicación. En SOC el descubrimiento y binding de servicios se realiza por demanda, eso significa que a pesar de que se necesiten muchos servicios para resolver una tarea, se obtienen uno por uno según se van necesitando. Por el otro lado las CFSM requieren el conjunto total de participantes de un protocolo para poder determinar si el protocolo puede llevarse a cabo sin errores. La idea de los AFCA es intentar reducir la distancia entre ambas naturalezas. Por lo tanto una de las tareas principales que queremos lograr con AFCAs es obtener, vía una proyección, la inferfaz de comunicación en la forma de una mCFSM.
Al principio de esta sección introdujimos los AFCA para modelar el comportamiento deseado en SOC de binding incremental. Ahora nos queremos enfocar en los aspectos puramente comunicacionales, es decir la interacción entre distintos participantes. Para hacer énfasis en estos aspectos existen las CFSMs, pero los AFCA introducen con la composición la posibilidad de que haya múltiples canales de comunicación entre dos participantes. En esta sección atendemos dicha problemática introduciendo las Multichannel Communicating Finite State Machines. Se definen, del mismo modo que una CFSM sobre $\mathcal{M}$ de la siguiente manera.

\begin{definition}[mCFSM] Una multichannel communicating finite state machine (mCFSM) sobre $\mathcal{M}$ es una un sistema finito de transición $(Q, \mathcal{C}, q_0, \mathcal{M}, \delta)$ donde
\begin{itemize}
  \item $Q$ es un conjunto finito de estados
  \item $\mathcal{C} = \{ pq_n \mid pq \in \mathcal{P}^2, n \in \mathbb{N}, p \not= q\}$ es un conjunto de canales
  \item $q_0 \in Q$ es el estado inicial;
  \item $\delta \subseteq Q \times (\mathcal{C} \times \{!,?\} \times \mathcal{M}) \times
    Q$ es un conjunto finito de \emph{transiciones}.
  \end{itemize}

Un communicating system es un mapa S que asigna un mCFSM S(p) a cada $p \in \mathcal{P}$. Escribimos $q \in S(p)$ cuando q es un estado de la máquina S(p) y $\tau$ es una transición de S(p).
  
Al igual que antes la semántica de un communicating system se obtiene considerando configuraciones. Estas configuraciones son iguales a las de las CFSM puras con la salvedad que ahora los canales no están restringidos a un único par entre cada par de participantes.

 \end{definition}
 
% * <ivissani@gmail.com> 2018-01-28T13:14:43.225Z:

% De esta manera lo que tenés es una enumeración de canales entre cada par de participantes. Lo único que hay que hacer es ajustar la semántica (las reglas) a esta nueva notación para que sea coherente.
% 
% Notar que la cantidad de canales existentes entre cada par de participantes ahora es infinita (los naturales) PERO la cantidad de ellos que vas a usar siempre va a ser finita porque de acuerdo a la notación que tenemos hay que fijar qué canal se usa en cada envío/recepción de mensaje y, por lo tanto, el hecho de que EXISTAN infinitos canales no afecta a la semántica ni al poder expresivo de las CFSMs
% 


\begin{definition}[Semántica de una mCSFSM]La configuración de un multichannel communicating system se define en términos de transiciones entre configuraciones como se ve a continuación:

  La configuración de un communicating system S es un par
  $s = \sysconf{q}{w}$ donde
  $\vectorize{q} = \left(q_\p\right)_{\p \in \mathcal{P}}$ donde
  $q_\p \in S(p)$ para cada $p \in \mathcal{P}$ y
  $\vectorize{w} = \left(w_{pq}\right)_{pq \in \C}$ con
  $w_{pq} \in \mathcal{M} $.
  
  Una configuración $s' = \sysconf{q'}{w'}$ es \emph{alcanzable} desde otra
  configuración $s = \sysconf{q}{w}$ a través de la \emph{ejecución de la
    transición}  $\tau $ (escrito $s \overset{t}{\rightarrow} s'$) si existe un
   $m \in \mathcal{M}$ tal que ocurre una de las siguientes alternativas:
	\begin{enumerate}
		\item $t = (q_p, pq_n!m,  q'_p) \in \delta_p$ y 
			\begin{enumerate}
				\item $q'_{p'} = q_{p'}$ for all $p' \neq p$; y
				\item $w'_{pq_n} = w_{pq_n} \cdot m$ and $w'_{p'q'_m} = w_{p'q'_m}$ for all $p'q'_m \neq pq_n$; \\
			\end{enumerate}
                O bien
		\item $t = (q_q, pq?m,  q'_q) \in \delta_q$ y 
			\begin{enumerate}
				\item $q'_{p'} = q_{p'}$ for all $p' \not= q$; y
				\item $m \cdot w'_{pq_n} = w_{pq_n}$ and $w'_{p'q'_m} = w_{p'q'_m}$ for all $p'q'_m \neq pq_n$
			\end{enumerate}
	\end{enumerate}
\end{definition}

Multichannel CFSMs  con un único canal para cada par ordenado de participantes son equivalentes a CFSMs puras.

\begin{definition}[Interfaz de comunicación de un AFCA] Es la mCFSM que resulta de aplicarle el siguiente procedimiento a un AFCA.
\begin{enumerate}
    \item Transformamos toda acción interna y de buffer en una transición $\epsilon$. Esto resulta en una autómata no determinístico con transiciones $\epsilon$
    \item Transformamos el autómata no determinístico rsultante del paso anterior en uno determinístico.
\end{enumerate}
\end{definition}

En la sección 2.3 explicamos como funciona la condición de Generalized Multiparty Compatibility para CFSMs. Necesitamos ver que la misma condición es aplicable a las Multichannel CFSMs, para esto la solución más práctica que se encontró es ver que podemos emular una Multichannel CFSM con una CFSM pura. Mostrando esa equivalencia vemos que si podemos aplicar GMC al emulador, la condición aplica al emulado. Dado que un multichannel communicating system es libre de errores de comunicación si y solo si el sistema emulado también lo es. El procedimiento consiste en generar un nuevo participante para cada canal entre dos otros participantes. Este nuevo participante es un simple repetidor de mensajes de uno de los participantes a otro. De este modo una comunicación multichannel entre un par de participantes es reemplazada por múltiples comunicaciones de un canal con un repetidor en el medio. Una applicación de este procedimiento se ve en la Figura X. El aspecto clave de esta emulación es que preserva el orden de los mensajes.

\begin{definition}[Sistema emulado] Dado un multichannel communicating system $(M_p)_{p \in \mathcal{P}}$, agrandamos el conjunto $ \mathcal{P}$ agregando un participante adicional por cada canal en el sistema original $\mathcal{P}'=\mathcal{P} \cup \bigcup_{p \in \mathcal{P}} \{p^{pq_n} \ | \ pq_n \in \mathcal{C}_p \} \cup \bigcup_{p \in \mathcal{P}} \{p^{qp_n} \ | \ qp_n \in \mathcal{C}_p \}$. Cada participante nuevo $p^{sr_n} \in \bigcup_{p \in \mathcal{P}} \{p^{pq_n} \ | \ pq_n \in \mathcal{C}_p \} \cup \bigcup_{p \in \mathcal{P}} \{p^{qp_n} \ | \ qp_n \in \mathcal{C}_p \}$ se define con la siguiente mCFSM:
\begin{itemize}
\item $Q_{p^{sr_n}} = \{q_0\} \cup \bigcup_{m \in \mathcal{M}} \{q_m\}$
\item $\mathcal{C}_{p^{sr_n}} = \{sp_n^{sr_n}, p^{sr_n}r_n,  p^{sr_n}s_n, rp_n^{sr_n} \}$
\item $q_{0_{p^{sr_n}}} = q_0 $
\item $ \delta_{p^{sr_n}} = \bigcup_{m \in \mathcal{M}} \{ (q_0,sp_n^{sr_n}?m,q_m),(q_m,p^{sr_n}r_n!m,q_0)  \} $
\end{itemize}
Cada viejo participante $q \in \mathcal{P}$ se reemplaza por un nuevo participante $q'$ donde:
\begin{itemize}
\item $\mathcal{C}_{q'} = \{qp_n^{qr_n} \ | \ qr_n \in \mathcal{C}_{q} \} \cup \{p^{sq_n}q_n \ | \ sq_n \in \mathcal{C}_{q} \}$ 
\item $ \delta_{q'} = \bigcup_{m \in \mathcal{M}} \{ (q, qp_n^{qr_n}!m, q') \ | \ (q, qr_n!m, q') \in \delta_q \} \cup \{q, p^{sq_n}q_n?m,q' \ | \ (q, sq_n?m, q') \in \delta_q \}$

\end{itemize}

Queda claro que si bien transformamos cada canal (buffer) en dos canales nuevos, el orden de los mensajes está garantizado como resultado del modo en que $\delta$ está definido para los repetidores. Nótese que los repetidores respetan ese orden porque al consumir un mensaje desde el canal de 'entrada' (el canal que sirve para recibir mensajes del emisor original) la única acción posible del repetidor es mandar el mensaje a través del canal de salida (el canal que sirve para enviar mensajes al receptor original). Como todo canal es FIFO y los repetidores se comportan del mismo modo, el orden entre dentro de cada canal queda garantiado. Por oro lado, no hay garantía respecto al orden de la comunicación entre canales del modelo original. Por lo tanto podemos decir que los repetidores no introducen más concurrencia que la que estaba en el modelo original y debido a eso los dos sistemas, a pesar de no ser bisimilares y no tener las mismas trazas, son equivalentes respecto a la ausencia de deadlock, recepciones no especificadas y mensajes huérfanos. Esta observación es importante porque provee una forma de chequear multichannesl communication systems recurriendo nuevamente al chequeo de GMC del sistema emulado.

\end{definition} 

\begin{figure}
    \centering
    \begin{example}
    
    \end{example}
    \caption{Caption}
    \label{fig:my_label}
\end{figure}


\begin{prop} El sistema emulado preserva los errores comunicacionales

\begin{proof} Primero notemos que los repetidores no pueden generar errores de comunicación dado que en el estado inicial pueden recibir el rango completo de mensajes y luego de ocnsumir un mensaje su única acción posible es reenviarlo. Recordemos también que el orden de los mensajes se preserva. Esto significa que si en el sistema original los mensajes $ m $ y $m'$ fueron enviados en ese orden sobre el canal $sr$ en el sistema emulado se envían en el mismo orden sobre el canal $sp^{sr}$ y por lo tanto son enviados en ese mismo orden a través del canal $p^{sr}r$. Entonces se puede mostrar que para cada error de configuración alcanzable en el sistema original, hay una configuración que presenta el mismo error que es alcanzable en el sistema emulado y viceversa.
\begin{itemize}
\item[$\Rightarrow$] Consideremos cualquier configuración de error $e$ alcanzable en el sistema original y consideremos cualquier camino que alcance $e$. Luego reemplacemos toda acción $sr!m$ con la secuencia de acciones $sp^{sr}!m, sp^{sr}?m, p^{sr}r!m$ y cada acción $sr?m$ con $p^{sr}?m$, este nuevo camino está presente en el sistema emulado y alcanza el estado donde las configuraciones de buffer y transiciones permitidas son las mismas para el conjunto de máquinas compartidas (es decir, todas las máquinas de sistema emulado menos los repetidores) y los repetidores están en su estado inicial con buffers vacíos.
\item[$\Leftarrow$] Consideremos cualquier error de configuración $e$ alcanzable en el sistema emulado y ocnsideremos cualquier camino que alcance $e$. Ya establecimos que el error no puede ser culpa de los repetidores dado que siempre pueden progresar. Por lo tanto sin importar el estado de los buffers en $e$ hay una configuración $e'$ que presenta el mismo error de comunicación y donde el estado de los buffers de los repetidores está vacío y están en su estado inicial. Luego consideremos que cualquier camino que alcance $e'$ está formado por secuencias de la forma $sp^{sr}!m,\ldots, sp^{sr}?m, \ldots, p^{sr}r!m$ y $p^{sr}r?m$ (donde los puntos suspensivos denota la posibilidad de interleaving de otras acciones). Entonces alcanza con reemplazar cada secuencia $sp^{sr}!m,\ldots_0, sp^{sr}?m, \ldots_1, p^{sr}r!m$ con la secuencia $\ldots_0,\ldots_1, sr!m$ y cada $p^{sr}r?m$ con $sr?m$ para obtener un camino que existe en el sistma original y aclanza una configuración que tiene sus buffers en el mismo estado con las mismas transiciones permitidas.
\end{itemize}

\end{proof}

\end{prop}

%!TEX root = ./main.tex

\section{Composición de Componentes Asincrónicas}
\label{composicion}
Como ya hemos mencionado, en SOC (Service-Oriented Computing) los sistemas son concebidos como objetos dinámicos construidos en \emph{run-time} en la medida que su ejecución llega a un estado en el que la intervención de servicios externos se hace necesaria. Es decir un sistema de este tipo utilizará distintos servicios según las necesidades que se manifiesten a lo largo de una ejecución particular. 
 
Una aplicación que se encuentra ejecutando se conecta con los servicios que le son necesarios a través de canales de comunicación por los cuales se envían o reciben mensajes. Estos canales pueden establecer una comunicación entre un número fijo (para cada canal particular) pero no acotado a priori de servicios. En la sección anterior hemos detallado un conjunto de propiedades que garantizan una comunicación sin errores a través de estos canales (i.e. ausencia de deadlock, ausencia de mensajes huérfanos y ausencia de situaciones en las que el receptor no se encuentra a la espera de un mensaje que le fue enviado) y un procedimiento para garantizarlas. Estas condiciones y su procedimiento de análisis parten de la hipótesis de que las CFSMs correspondientes a todos y cada uno de los participantes de la comunicación sobre dicho canal se encuentran disponibles.

Ahora bien, que la aplicación arribe a un estado en el que un servicio se hace necesario sobre un canal particular, no implica que todos los participantes también lo sean en ese mismo instante y por ello, con el objeto de profundizar esta concepción incremental, a demanda, que se tiene sobre los sistemas de software surge, más o menos naturalmente, la idea de poder dotar al \emph{middleware} de la capacidad de realizar un \emph{binding} parcial sobre los canales. A esta práctica la llamaremos \emph{binding incremental}.
 
Esta percepción parcial del \emph{binding} sobre un canal requiere la utilización de un un lenguaje de descripción que soporten dichos mecanismos de composición. Por ejemplo, al componer dos CFSMs puede ocurrir que cada máquina se comunique a través de un canal con un tercer participante, estos dos canales son independientes y, además, una vez que se ha realizado la composición, deben ser percibidos por este tercer participante como canales de comunicación separados. Por lo tanto, para preservar la semántica de la comunicación, es necesario que el CFSM resultante de la composición tenga dos canales con este tercer participante. Este fenómeno será el eje rector de las modificaciones que introduciremos en esta sección. Esta característica no solo será necesario a nivel de CFSM (interfaz de comunicación de un servicio) sino también de los autómatas que caracterizan el cómputo, cuya interfaz de comunicación es expresada a través de una CFSM.

Para modelar este comportamiento introducimos los Autómatas Finitos de Comunicación Asincrónica (AFCA). Estos autómatas tienen tres tipos de transiciones: 
\begin{inparaenum}[1)]
\item internas, que sirven el propósito de representar cómputo realizado por la componente; 
\item de buffer, que representan comunicación asincrónica interna entre elementos de la componente, y que permiten representar la comunicación entre dos participantes luego de una composición; y por último 
\item de antrada / salida, que modelan acciones de comunicación con otras componentes del sistema.
\end{inparaenum}

Como la idea es que los autómatas con comunicación asincrónica representen procesos, o servicios, que pueden formar parte de un sistema más grande, necesitamos definir la operación de composición. Para que un par de autómatas $E$ y $R$ sean plausibles de ser compuestos se deben satisfacer las siguientes condiciones:
 \begin{enumerate}
\item $\Sigma_E \cap \Sigma_R = \emptyset$ el conjunto de etiquetas debe ser disjunto, tanto internas, como de entrada/salida y de bufffer
\item $ B_E \cap B_R = \emptyset$ el conjunto de Buffers de ambos autómatas debe ser disjunto  
\end{enumerate}
\begin{definition}[Composición]
Dados $\mathcal{P}$ un conjunto de participantes y $\mathcal{M}$ un conjunto de mensajes. Denominamos al conjunto de autómatas participantes como $E_\mathcal{P} = \{E_{p_i} \ | p_i \in \mathcal{P}\}$, dónde $E_{p_i}= \langle Q_{p_i}, B_{p_i}, \mathcal{C}_{p_i}. \Sigma_{p_i}, \delta_{p_i}, q_{0{p_i}}, F_{p_i} \rangle$. Dado un conjunto de autómatas se exige que todo par de integrantes sean compatibles (i.e. se satisface que para todos $p_i, p_j \in \mathcal{P}$, $\Sigma_{p_i} \cap \Sigma_{p_j}= \emptyset$ y $B_{p_i} \cap B_{p_j} = \emptyset$). Luego, definimos la composición $E_\mathcal{P} = ||_{1..n} E_{p_i}$ como sigue:
\begin{itemize}
    \item $Q_\mathcal{P}= \Pi_{p_i \in \mathcal{P}} Q_{p_i}$ (i.e. el conjunto de estados de la composición es el producto cartesiano de los estados de los autómatas componentes),    
    \item $B_\mathcal{P} = \bigcup_{p_i \in \mathcal{P}} B_{p_i} \cup \{ \omega_c \ | \ \mbox{there exists } p_i, p_j \in \mathcal{P} \mbox{ such that } c \in \mathcal{C}_{p_i} \cap \mathcal{C}_{p_j} \}$ (i.e. el conjunto de nombres de buffers del autómata resultante se compone de los buffers de cada autómata participante en la composición, junto con uno nuevo por cada canal compartido entre cada par de autómatas, dichos canales son aquellos mediante los cuales los autómatas a componer intercambian mensajes),    
    \item $\mathcal{C}_\mathcal{P} = \bigcup_{p_i \in \mathcal{P}} \mathcal{C}_{p_i} \setminus \{ c_k \ | \mbox{for all } p_i, p_j \in \mathcal{P}, c \in \mathcal{C}_{p_i} \cap \mathcal{C}_{p_j} \}$,    
    \item $\Sigma_\mathcal{P} = \Sigma_{\mathcal{P}\mathit{Int}} \cup \Sigma_{\mathcal{P}\mathit{Ex}} \cup \Sigma_{\mathcal{P}\mathit{Buff}}$ tal que:    
    \begin{inparaenum}[1)]
        \item $\Sigma_{\mathcal{P}\mathit{Int}} = \bigcup_{p_i \in \mathcal{P}} \Sigma_{p_i\mathit{Int}}$,
        \item $\Sigma_{\mathcal{P}\mathit{Ex}} = \bigcup_{p_i \in \mathcal{P}} \Sigma_{p_i\mathit{Ex}} \setminus \Sigma_{p_i \mapsto p_j}$ y 
        \item $\Sigma_{\mathcal{P}\mathit{Buff}} = \bigcup_{p_i \in \mathcal{P}} \Sigma_{p_i\mathit{Buff}} \cup \Sigma_{p_i \mapsto p_j}$,
    \end{inparaenum}
donde $\Sigma_\mathit{p_i \mapsto p_j} =\{ \langle p_i,p_j,c \rangle \ | \ p_i, p_j \in \mathcal{P}, \  c \in \mathcal{C}_{p_i} \cap \mathcal{C}_{p_j}\}$
\item $\delta_{\mathcal{P}} = \delta_{\mathcal{P}\mathit{Int}} \cup \delta_{\mathcal{P}\mathit{Ex}} \cup \delta_{\mathcal{P}\mathit{Buff}}$ tal que:
\begin{enumerate}
\item $\delta_{\mathcal{P}\mathit{Int}} \subseteq Q_{\mathcal{P}} \times \Sigma_{\mathcal{P} Int} \times Q_{\mathcal{P}}$  es la relación de transición interna, tal que que se satisface la siguiente fórmula:
$$(\forall \langle q, \sigma, q' \rangle \in \delta_{\mathcal{P}\mathit{Int}})(\exists p_i \in \mathcal{P})(\exists q_{p_i}, q'_{p_i} \in Q_{p_i}, \sigma \in \Sigma_{p_i})(\langle q_{p_i}, \sigma, q'_{p_i} \rangle \in \delta_{p_i Int})$$

\item $\delta_{\mathcal{P}\mathit{Ex}} \subseteq Q_{\mathcal{P}} \times \Sigma_{\mathcal{P}\mathit{Ex}} \times Q_{\mathcal{P}}$ es la relación de transición externa, tal que que se satisface la siguiente fórmula: $$(\forall \langle q, \sigma, q' \rangle \in \delta_{\mathcal{P}\mathit{Ex}}) (\exists p_i \in \mathcal{P})(\exists q_{p_i},q'_{p_i} \in Q_{p_i}, \sigma \in \Sigma_{p_i}) (\langle q_{p_i},\sigma,q'_{p_i} \rangle \in \delta_{p_i Ex})$$

\item $\delta_{\mathcal{P}\mathit{Buff}} \subseteq Q_{\mathcal{P}} \times \Sigma_{\mathcal{P} \mathit{Buff}} \times Q_{\mathcal{P}}$ es la relación de transición por comunicación asincrónica interna tal que se satisface la siguiente fórmula:
$$
\begin{array}{l}
(\forall \langle q, \sigma, q' \rangle \in \delta_{\mathcal{P}\mathit{Buff}})(q = \langle q_k \rangle_{p_k \in \mathcal{P}} \land q' = \langle q'_{k'} \rangle_{p_{k'} \in \mathcal{P}} \land\\
\qquad
\begin{array}{l}
((\exists p_i \in \mathcal{P})(\exists \omega_b \in B_{p_i}, \omega_b \{\gg, \ll\} m \in \Sigma_{{p_i} \mathit{Buff}})\\
\qquad ((\forall p_k \in \mathcal{P})(p_k \neq p_i \implies q_k = q'_k) \land \langle q_i, \omega_b \{\gg, \ll\} m, q'_i \rangle \in \delta_{p_i \mathit{Buff}}) \lor\\
(\exists p_i, p_j \in \mathcal{P})(p_i \neq p_j \land (\forall p_k \in \mathcal{P})((p_k \neq p_i \land p_k \neq p_j) \implies q_k = q'_k) \land \\ 
\qquad (\exists \widehat{q}, \widehat{q'} \in Q_{p_i}) \land (q_i = \widehat{q} \land q'_i = \widehat{q'}) \land (\\ 
\qquad\qquad (\exists \mathit{Out}(c_{{p_i, p_j}_n}, m) \in \Sigma_{p_i \mathit{Ex}})(\sigma = \omega_{c_{{p_i, p_j}_n}} \ll m) \lor\\
\qquad\qquad (\exists \mathit{In}(c_{{p_j, p_i}_n}, m) \in \Sigma_{p_i \mathit{Ex}})(\sigma = \omega_{c_{{p_j, p_i}_n}} \gg m)))))
\end{array}
\end{array}
 $$
%Además vale que $\forall q_i,q_j \in Q_\mathit{\mathcal{P}}, \ m \in \mathcal{M} \ |	 \ \langle q_i, [\omega_\mathit{b}]_{b \in B_\mathcal{P}} \gg m, q_j \rangle$ y $\langle q_{p_i},\sigma,q'_{p_i} \rangle \in \delta_{p_i Buff}$ $ \delta_\mathit{\mathcal{P}Buff} \iff \exists$ las configuracioes $c_i = \langle q_i, [\omega_\mathit{b},...,\omega_\mathit{b}:m, ..., \omega_\mathit{b}] \rangle, c_j =\langle q_j, [\omega_\mathit{b},...,\omega_\mathit{b}, ..., \omega_\mathit{b}] \rangle$ y $c_i \vdash c_j$
\end{enumerate}
\item $q_0 = \langle {q_0}_{p_1}, \ldots, {q_0}_{p_n} \rangle$, y
\item $F_{\mathcal{P}} = \bigcup_{1..n} F_{p_i}$.
\end{itemize}

Cada autómata es un sistema independiente que cumple una función (o una serie de funciones), y se relaciona con otros a través del envío de mensajes. Decimos que si un conjunto de autómatas tienen una acción con la misma etiqueta, al componerlos, ambas transiciones se ejecutarían a la vez. Como estos autómatas son de comunicación asincrónica, queremos evitar que las transiciones se sincronicen de ese modo. Para asegurarnos esto pedimos que para todo par de autómatas a componerse tengan conjuntos de etiquetas de acciones que sean disjuntos. 

Del mismo modo, cada autómata tiene su propio conjunto de buffers que pedimos sean disjuntos, para distinguir la comunicación interna de cada autómata componente de la que ocurra entre componentes o con participantes externos. Para modelar la comunicación interna entre componentes, agregamos dos buffers por cada par de integrantes, uno para cada sentido de la comunicación, de $E_{p_i}$ a $E_{p_j}$ y viceversa.

Como los conjuntos de acciones son disjuntos podemos decir que las acciones internas, de comunicación externa y de buffer, de cada componente se preservan, siempre y cuando tenga sentido con la composición de estados. Existe un caso particular que ocurre cuando existían envíos de mensaje de un autómata componente a otro. En ese caso dado que ambos ahora son parte un mismo autómata, la comunicación pasa a ser envío de mensajes interno. Para representar este tipo de comunicación es que utilizamos buffers. De este modo el intercambio que antes era $In((E_{p_i},(E_{p_j},c),m)$ y $Out(((E_{p_i},(E_{p_j},c),m)$ ahora es $b_{\mathcal{P}}  \ll m$ y $b_{\mathcal{P}}  \gg  m$, donde $ b_{\mathcal{P}} \in \{ \mathcal{C}_{p_i} \cap \mathcal{C}_{p_j} \} $ son los buffers exclusivos del autómata compuesto. 

Al componer autómatas la comunicación que antes era externa y ahora es de buffer puede generar problemas. Puntualmente pueden aparecer transiciones de consumo de un buffer (que antes eran envío de mensajes) donde antes no había. De este modo pueden aparecer secuencias de estados y transiciones donde se consume un mensaje antes de que este sea depositado en el buffer correspondiente. Para esto pedimos que $\delta_\mathit{\mathcal{P}Buff}$ cumpla con una condición especial. Solo pueden haber transiciones de consumo saliendo de un estado si en alguna secuencia de acciones que termina en ese estado, hay transiciones de producción (es decir se encola un mensaje en el buffer).

\end{definition}


La Figura~\ref{fig:ejemplo-aa} muxestra un ejemplo de una composición de dos autómatas. 

\begin{figure}[ht]
\begin{center}
Dibujo
\end{center}
\caption{Ejemplo de composición autómata asíncrono de comunicación}
\label{fig:ejemplo-aa}
\end{figure} 

\begin{definition}[Determinismo] Decimos que un autómata es determinístico cuando cumple que no hay dos transiciones con la misma etiqueta que partan de un mismo estado y vayan a estados distintos. Es decir 

\begin{centering}
Sea un autómata $ \Lambda = \langle Q, \Sigma, \delta, q_0, F \rangle$ se cumple
$ \forall \  q_i, q_j, q_k \in Q_{j \neq k}, \  \nexists \ \delta_i, \delta_j \in \delta, t \in \Sigma \ \| \  \delta_1 = \langle q_i, t, q_j \rangle, \ \delta_2 = \langle q_i, t, q_k \rangle$ \\
\end{centering} 

Decimos que la composición de estos autómatas preserva el determinismo. Esto es un resultado directo de que ambos autómatas no comparten acciones y de la definición de la composición de $\delta$.

\end{definition}


% \begin{definition}[Autómata proyectado] Dados, un conjunto de participantes $\mathcal{P}$, un conjunto de mensajes $\mathcal{M}$, un autómata $A = \langle Q, \mathcal{C}, B, \Sigma, \delta, q_0, F\rangle$, y una configuración instantánea del mismo $\langle q_{i}, \Omega_i \rangle$, $q_i \in Q, \Omega_i \in B^*$ llamamos autómata proyectado $A_{\pi}= \langle Q_{\pi}, \beta, \mathcal{C}_{\pi}, \Sigma_{\pi}, \delta_{\pi}, q_i, F\rangle$ al autómata resultante de recortar A a partir de $q_i$. Donde:
% \begin{itemize}
%     \item $Q_{\pi} = \{q \in Q \ | \  \exists \omega \in B^*,$ tq $\langle q_{i}, \omega_i \rangle \vdash^* \langle q, \omega \rangle  \}$ son aquellos estados alcanzables desde $q_i$
%     \item $ \beta \subseteq B $ son los buffers del autómata. Los buffers retienen aquellos mensajes que fueron almacenados pero aún no han sido retirados en $q_i$
%     \item $\mathcal{C}_{\pi}$ es el conjunto de canales de comunicación externa
%     \item $\Sigma_{\pi}=  \{ \gamma \in \Sigma \ | \ \exists q_k, q_l \in Q' \land \langle q_k ,\gamma, q_l \rangle \in \delta \}$ es el conjunto de etiquetas del autómata proyectado y se compone de aquellas etiquetas 
% %     $\{ \Sigma_\mathit{Int} \cup \Sigma_\mathit{Ex} \cup \Sigma_\mathit{Buff}\} $, $\Sigma \cap \mathcal{M} = \emptyset$ es el conjunto de etiquetas del autómata, siendo
% %     \begin{inparaenum}[1)]

% %         \item $\Sigma_\mathit{Int}$ las acciones internas del autómata 

% %         \item $\Sigma_\mathit{Ex}$ un conjunto de etiquetas de la forma $\langle p_1,p_2,c\rangle$ dónde $p_1,p_2 \in \mathcal{P}$ son, respectivamente, el emisor y el receptor de la comunicación y $c \in \mathcal{C}$ es el canal a través del cual se resuelve la misma.

% %         \item $\Sigma_\mathit{Buff}$ es el conjunto de etiquetas de las acciones sobre los buffers de la forma $b \ll m$ o $b \gg m$, dónde $b \in B$ y $m \in \mathcal{M}$.
% % \end{inparaenum}
% \item $\delta_{\pi} = \{  \}$
% % \item $\delta = (\delta_\mathit{Int} \cup \delta_\mathit{Ex} \cup \delta_\mathit{Buff})$ siendo:

% % \begin{inparaenum}[1)]
% %     \item $\delta_\mathit{Int} \subseteq Q \times \Sigma_\mathit{Int} \times Q$ es la relación de transicion por acciones internas de $A_\mathcal{P}$, %transiciones internas

% %     \item $\delta_\mathit{Ex} \subseteq Q \times \{\mathit{In}(c,m), \mathit{Out}(c,m) | c \in \Sigma_{Ex} \land m \in \mathcal{M} \} \times Q$ es la relación de transición de comunicación externa de $A_\mathcal{P}$, %comunicación externa

% %     \item $\delta_\mathit{Buff} \subseteq Q \times \Sigma_\mathit{Buff} \times Q$ %comunicación interna
% % \end{inparaenum}    
% \item $q_i \in Q'$ es el estado inicial del autómata proyectado, 
%     \item $F \subseteq Q$ es el conjunto de estados finales del autómata.
% \end{itemize}

% \end{definition}

\subsection{Composición parcial vs composición total}
En esta sección definimos los Autómatas Finitos de Comunicación Asincrónica para modelar la composición parcial de CFSMs. Ahora necesitamos demostrar que esta composición parcial es equivalente a una composición total.

Como sabemos que todas las CFSMs a componer se encuentran en $\mathcal{P}$ podemos decir que conocemos a priori todos los componentes de la composición final. Dado un conjunto finito $\mathcal{P}$ de CFSMs denominamos $ p_1, p_2, p_3, \ldots, p_n$. 

Si componemos $p_1$ y $p_2$ nos quedaría el conjunto $\mathcal{P}_1 =\{ p_{12} \}, p_3, \ldots, p_n \} $, podemos hacer un paso siguiente componiendo $p_{n-1}$ y $p_n$. De esto obtenemos $\mathcal{P_2}_2= \{ p_{12}, p_3, \ldots, p_n-1 \} $. Podemos continuar este proceso hasta llegar a tener un único autómata. %Esto de acá capaz vuela cuando lo dibuje

Decimos que en cada paso de la composición parcial tengo una función suryectiva, pero no inyectiva que garantiza que cada punto del codominio es la composición de al menos dos puntos de la preimagen.

Llamemos TM1 y TM2 a los autómatas resultantes de la composición de todos los elementos de $\mathcal{P}$ por sucesión de composiciones parciales y por composición total respectivamente. Queremos ver que $ q \in Q_{TM1} \iff q \in Q_{TM2} $ y en ambos casos está entre los estados alcanzables. 

Queremos demostrar que
\begin{enumerate}
\item Las configuraciones alcanzables entre tm1 y tm2 son las mismas
\item tm1 es bisimilar a tm2, es decir que para cada configuración las acciones realizables son las mismas
\end{enumerate}

\begin{definition}[Bisimulación]
Dado un sistema de transición con etiquetas $ S =\langle Q, \Sigma, \delta \rangle $, una bisimulación es una relación binaria $R \subseteq Q \times Q$, tal que tanto $R$ como su transpuesta $R^T$ son simulaciones. Equivalentemente $R$ es una bisimulación si para cada par de elementos $p, q \in Q$ vale $\langle p, q \rangle \in R$ y para todo $\sigma \in \Sigma $ vale:
\begin{itemize}
    \item $(\forall p' \in \Sigma \  | \  p \ \overset{\sigma}{\rightarrow} p' \implies \exists q'\in Q \ \mathit{tal que} \ q \overset{\sigma}{\rightarrow} q' \land \langle p', q' \rangle \in R )$ y, simétricamente vale 
    \item $(\forall q' \in \Sigma \  | \  q \overset{\sigma}{\rightarrow} q' \implies \exists p'\in Q \ \mathit{tal que} \ p \overset{\sigma}{\rightarrow} p' \land \langle p', q' \rangle \in R ) $
\end{itemize}

Dados dos estados $p, q \in Q $, p es bisimilar a q, se denota $p \sim q $, si existe una bisimulación R tal que $\langle p,q \rangle \in R$. La relación de bisimilaridad $\sim$ es una relación de equivalencia. Además es la relación de bisimulación más grande sobre un sistema dado.

\end{definition}








\section{Trabajo futuro} 



\begin{definition} Sea $\mathcal{P}$ un conjunto de participantes, $\mathcal{C}$ un conjunto de canales de comunicación unidireccionales y $\mathcal{M}$ un conjunto de mensajes. Un autómata es una es una estructura $A_\mathcal{P} = \langle Q, B, \Sigma, \delta, q_0, F\rangle$ tal que:

\begin{itemize}
\item $Q$ es un conjunto finito de estados,
\item $B$ es el conjunto de nombres de buffers,
\item $\Sigma = \{ \Sigma_\mathit{Int} \cup \Sigma_\mathit{Ex} \cup \Sigma_\mathit{Buff}\} $ es el conjunto de etiquetas del autómata, siendo:
\begin{inparaenum}[1)]

\item $\Sigma_\mathit{Int}$ las acciones internas del autómata 

\item $\Sigma_\mathit{Ex}$ un conjunto de etiquetas de la forma $\langle p_1,p_2,c\rangle$ dónde $p_1,p_2 \in \mathcal{P}$ son, respectivamente, el emisor y el receptor de la comunicación y $c \in \mathcal{C}$ es el canal a través del cual se resuelve la misma.
%\item $\forall (p_1,P_2,c), (p'_1,P'_2,c') \in \Sigma_{Ex} c = c' \iff p_1= P'_1 \land p_2= P'_2 $

\item $\Sigma_\mathit{Buff}$ es el conjunto de etiquetas de las acciones sobre los buffers de la forma $b \ll m$ o $b \gg m$, dónde $b \in B$ y $m \in \mathcal{M}$.
\end{inparaenum}
\item $\delta = (\delta_\mathit{Int} \cup \delta_\mathit{Ex} \cup \delta_\mathit{Buff})$ siendo:

\begin{inparaenum}[1)]
\item $\delta_\mathit{Int} \subseteq Q \times \Sigma_\mathit{Int} \times Q$ es la relación de transicion por acciones internas de $A_\mathcal{P}$, %transiciones internas

\item $\delta_\mathit{Ex} \subseteq Q \times \{\mathit{In}(c,m), \mathit{Out}(c,m) | c \in \Sigma_{Ex} \land m \in \mathcal{M} \} \times Q$ es la relación de transición de comunicación externa de $A_\mathcal{P}$, %comunicación externa

\item $\delta_\mathit{Buff} \subseteq Q \times \Sigma_\mathit{Buff} \times Q$ %comunicación interna
\end{inparaenum}
\item $q_0 \in Q$ es el estado inicial, y
\item $F \subseteq \mathit{Partes(Q)}$ es el conjunto de conjuntos de estados finales. 
\end{itemize}

Se denota $P(A)$, al conjunto de participantes que integran el autómata A, y se define como $P(A) = \{ p \in \mathcal{P} \ | (\exists\ \langle p1,p2,c\rangle \in \Sigma_\mathit{Ex})(p1=p \lor p2=p) \} $ 
\end{definition}

Los autómatas de asincrónicos reactivos se comportan como los autómatas asincrónicos de comunicación en casi todo sentido. La diferencia principal es que con estos queremos representar el comportamiento de sistemas que no tienen una ejecución finita. Para esto tomamos comportamiento de los autómatas de Muller que nos genera una condición de aceptación doble. En este caso el conjunto de estados finales es un conjunto de conjuntos y decimos que el sistema terminó su ejecución cuando pasa al menos una cantidad infinita de veces por alguno de los conjuntos que componen a F. Para asegurarnos de que la ejecución sea correcta, pedimos también que en ese ciclo infinito los buffers estén vacíos.


\section{Conclusiones}

\section{Bibliografía}
\begin{thebibliography}{}
\bibitem{CFSM}D. Brand and P. Zafiropulo. On communicating finite-state machines. Journal of the ACM, 30(2):323-342, 1983.
\bibitem{Communicating System}Julien Lange, Emilio Tuosto, Nobuko Yoshida. From Communicating Machines to Graphical ChoreographiesIn S. K. Rajamani and D. Walker, editors, Proceedings of 42rd. Annual ACM SIGPLAN-SIGACT Symposium on Principles of Programming Languages, POPL ’15, pages 221–232, New York, NY, USA, 2015. ACM.

\end{thebibliography}


\end{document}