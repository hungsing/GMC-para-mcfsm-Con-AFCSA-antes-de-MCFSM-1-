%!TEX root = ./main.tex
\cnote{Este texto parece muy introductorio, habla de la motivación para introducir AFCAs y mCFSM pero eso ya pasó en las secciones anteriores. Pasar este texto como introducción del capítulo 3 diciendo para cada cosa dónde lo presentás.}

Como ya hemos mencionado, en SOC los sistemas son concebidos como objetos dinámicos construidos en \emph{run-time} en la medida que su ejecución llega a un estado en el que la intervención de servicios externos se hace necesaria. Es decir un sistema de este tipo utilizará distintos servicios según las necesidades que se manifiesten a lo largo de una ejecución particular. 
 
Una aplicación que se encuentra ejecutando se conecta con los servicios que le son necesarios a través de canales de comunicación por los cuales se envían o reciben mensajes. Estos canales pueden establecer una comunicación entre un número fijo (para cada canal particular) pero no acotado a priori de servicios. En la sección anterior hemos detallado un conjunto de propiedades que garantizan una comunicación sin errores a través de estos canales (i.e. ausencia de deadlock, ausencia de mensajes huérfanos y ausencia de situaciones en las que el receptor no se encuentra a la espera de un mensaje que le fue enviado) y un procedimiento para garantizarlas. Estas condiciones y su procedimiento de análisis parten de la hipótesis de que las CFSMs correspondientes a todos y cada uno de los participantes de la comunicación sobre dicho canal se encuentran disponibles.

Ahora bien, que la aplicación arribe a un estado en el que un servicio se hace necesario sobre un canal particular, no implica que todos los participantes también lo sean en ese mismo instante y por ello, con el objeto de profundizar esta concepción incremental, a demanda, que se tiene sobre los sistemas de software surge, más o menos naturalmente, la idea de poder dotar al \emph{middleware} de la capacidad de realizar un \emph{binding} parcial sobre los canales. A esta práctica la llamaremos \emph{binding incremental}.
 
Esta percepción parcial del \emph{binding} sobre un canal requiere la utilización de un un lenguaje de descripción que soporten dichos mecanismos de composición. Por ejemplo, al componer dos CFSMs puede ocurrir que cada máquina se comunique a través de un canal con un tercer participante, estos dos canales son independientes y, además, una vez que se ha realizado la composición, deben ser percibidos por este tercer participante como canales de comunicación separados. Por lo tanto, para preservar la semántica de la comunicación, es necesario que el CFSM resultante de la composición tenga dos canales con este tercer participante. Este fenómeno será el eje rector de las modificaciones que introduciremos en esta sección. Esta característica no solo será necesario a nivel de CFSM (interfaz de comunicación de un servicio) sino también de los autómatas que caracterizan el cómputo, cuya interfaz de comunicación es expresada a través de una CFSM.

Para modelar este comportamiento introducimos los Autómatas Finitos de Comunicación Asincrónica (AFCA). Estos autómatas tienen tres tipos de transiciones: 
\begin{inparaenum}[1)]
\item internas, que sirven el propósito de representar cómputo realizado por la componente; 
\item de buffer, que representan comunicación asincrónica interna entre elementos de la componente, y que permiten representar la comunicación entre dos participantes luego de una composición; y por último 
\item de entrada / salida, que modelan acciones de comunicación con otras componentes del sistema.
\end{inparaenum} \\

\cnote{Abrir una sección sobre equivalencia del comportamiento de la composición de los AFCAs y el del communicating systems formado por sus interfaces de comunicación.}
Esta sección va a estar enfocada en la definición de composición total de autómatas y la noción de que dado un conjunto $\{ \mathcal{A}_i\}_{i \in I}$ de AFCA weak deterministic sin transiciones de buffer la semántica de la composición es equivalente a la del conjunto de mCFSM proyectadas de los elementos de $\{ \mathcal{A}_i\}_{i \in I}$. 

\begin{figure}[ht]
$$
\xymatrix{   
	\{\mathcal{A}_i\}_{i \in I} \ar[rrr]_{\Pi} \ar[d]_{||} & & & \{C_i\}_{i \in I}  \ar[d]_{semantica}  \\
	  {A} \ar[r]_{\Pi'} \ar[d]_{\Pi} & \mathcal{L}' & = & {\mathcal{L}}  \\
	  \emptyset
}
$$
\caption{La composición de componentes asincrónicas preserva la semántica asincrónica de su interfaz de comunicación.}
\label{fig:preservacion}
\end{figure}

En general, cuando se trata de formalismos que modelan simultáneamente aspectos del comportamiento interno de una componente y elementos de interfaz observable por otras componentes, la proyección de dicha interfaz no resulta bisimilar con el comportamiento real de la componente. A continuación se muestra un ejemplo de este fenómeno recurriendo a la clase de autómatas presentada en la \ref{def:afca}, tomando como comportamiento interno las transiciones internas y de buffer y como comportamiento observable las acciones de comunicación, tanto de entrada como de salida, con otras componentes.

\cnote{Armar una sola figura y usar subfloats para esta figura. El ejemplo no está bien, te mando una foto.}
\begin{figure}[ht]
\begin{tikzpicture}[->, thick]
 \node[state,initial] (q_0)   {$q_0$}; 
 \node[state] (q_1) [below = of q_0 ] {$q_1$};
 \node[state] (q_2) [right = of q_1 ] {$q_2$};
 \node[state] (q_3) [below = of q_1 ] {$q_3$};
 \node[state] (q_4) [below = of q_2 ] {$q_4$};
 \node[state, accepting] (q_5) [below = of q_3 ] {$q_5$};
 \draw[]        
        (q_0) edge[left] node{int1} (q_1)
        (q_0) edge[right] node{int2} (q_2)
        (q_1) edge[left] node{in(sp,a)} (q_3)
        (q_3) edge[left] node{in(sp,b)} (q_5)
        (q_2) edge[right] node{in(sp,b)} (q_4)
        ; 
\end{tikzpicture}
\qquad
\begin{tikzpicture}[->, thick]
 \node[state,initial] (q_0)   {$q_0$}; 
 \node[state] (q_1) [below = of q_0 ] {$q_1$};
  \node[state, accepting] (q_2) [below = of q_1 ] {$q_2$};
 \draw[]        
        (q_0) edge[left] node{out(sp,a)} (q_1)
        (q_1) edge[left] node{out(sp,b)} (q_2)
        ; 
\end{tikzpicture}

\caption{Los AFCA P y S, donde P no es Weak deterministic}
\label{fig:weak determinacy afca}
\end{figure}

\begin{figure}[ht]
    \centering
    
\begin{tikzpicture}[->, thick]
 \node[state,initial] (q_0)   {$q_0$}; 
  \node[state] (q_1) [below = of q_0 ] {$q_1$};
 \node[state] (q_2) [right = of q_1 ] {$q_2$};
 \node[state, accepting] (q_3) [below = of q_1 ] {$q_3$};
 \draw[]        
        (q_0) edge[left] node{sp?a} (q_1)
        (q_0) edge[right] node{sp?b} (q_2)
        (q_1) edge[left] node{sp?b} (q_3)
        ; 
\end{tikzpicture}
\qquad
\begin{tikzpicture}[->, thick]
 \node[state,initial] (q_0)   {$q_0$}; 
 \node[state] (q_1) [below = of q_0 ] {$q_1$};
  \node[state, accepting] (q_2) [below = of q_1 ] {$q_2$};
 \draw[]        
        (q_0) edge[left] node{sp!a} (q_1)
        (q_1) edge[left] node{sp!b} (q_2)
        ; 
\end{tikzpicture}

   \caption{Las mcfsm correspondientes a los autómatas S y P}
    \label{fig:weak determinacy mcfsm}
\end{figure}

\cnote{Explicar acorde al nuevo ejemplo.} En la Fig 3.4\cnote{Usar referencia!} podemos ver al sistema compuesto de los AFCA $S$ y $P$. En la misma podemos observar como $S$ tiene un único camino lineal donde envía los mensajes $a$ y $b$ al autómata $P$. Por su lado $P$ puede tomar dos caminos, ejecutando una de sus dos transiciones internas. Esas transiciones no afectan directamente el comportamiento del autómata $S$. Pero cual de ellas elija ejecutar $P$, sí lo afecta. Al tomar el primer camino el sistema se ejecuta de manera correcta, pero si ejecuta la transición interna $int2$, no puede recibir el mensaje $a$ y por lo tanto el sistema entra en deadlock. El autómata $P$ es un autómata determinístico, pero esa noción no alcanza para que el sistema funciones correctamente. En la figura 3.5 se ve como la mCFSM proyectada preserva los errores comunicacionales.  

Para prevenir este tipo de problemas surgen la noción de determinismo débil o weak determinacy. La noción de determinismo es importante dado que está relacionada con la predecibilidad del sistema. Si realizamos el mismo experimento dos veces sobre un sistema determinista, comenzando ambas veces del estado inicial, esperamos obtener el mismo resultado o comportamiento cada vez. Además, la predecibilidad es algo que frecuentemente requerimos de sistemas. Por ende si elegimos especificar un sistema $S$ dándole un participante abstracto $Spec$ al que $S$ debería ser equivalente, entonces $Spec$ va a ser determinista. La idea de \emph{Weak Determinacy} \cite[Chap.~11, Def.~3]{milner89}\cnote{subir esta cita al primer lugar en que lo mencionás.} es una forma más amplia de determinismo que nos permite contemplar una equivalencia observacional entre participantes. Esta equivalencia basada en la definición de bisimulación tiene en cuenta que para un participante externo, las transiciones $\epsilon$ son invisibles y por lo tanto una serie de estados conectados por transiciones consecutivas de este tipo serían bisimilares. En este caso tomamos las transiciones internas o de comunicación interna como observacionalmente bisimilares para un participante externo.  

%\begin{definition}[Determinacy] 
%Dado un autómata $ \Lambda = \langle Q, \Sigma, \delta, q_0, F \rangle$ decimos que es \emph{determinate} si vale que $(\forall q, q', q'' \in Q)(\nexists t \in \Sigma)(q \xrightarrow{t} q' \land q \xrightarrow{t} q''$.
%\end{definition}

\begin{definition}[Weak Determinacy]
Sea $A= \langle Q, \mathcal{C}, B, \Sigma, \delta, q_0, F \rangle$ un AFCA y $\Sigma' \subseteq \Sigma$ un conjunto de etiquetas a silenciar. Se dice que $A$ es \emph{weak determinate sobre $\Sigma'$} si para todo $q, q', q'' \in Q$ y para todo $s, s' \in \Sigma^*$ tal que $\widehat{s}_{\Sigma'} = \widehat{s'}_{\Sigma'}$, $q \xrightarrow{s} q'$ y $q \xrightarrow{s'} q''$, $q' \sim_{\Sigma'} q''$.
\end{definition}

\begin{prop}
Dado un AFCA $A = \langle Q, \mathcal{C}, B, \Sigma, \delta, q_0, F \rangle$ weak determinate tal que $\Sigma_\mathit{Int} \subseteq \Sigma$ denota el conjunto de transiciones internas, entonces dados dos estados $q, q' \in Q$ tales que $q \xrightarrow{\sigma} q'$ con $\sigma \in \Sigma_{\mathit{Int}}^*$,  $q \sim_{\Sigma_\mathit{Int}} q'$.
\end{prop}
\begin{proof}
Supongamos que $A$ es weak determinate sobre $\Sigma_\mathit{Int}$ y existen $q, q' \in Q$ tales que $q \xrightarrow{\sigma} q'$ con $\sigma \in \Sigma_{\mathit{Int}}^*$ y no vale que $q \sim_{\Sigma_\mathit{Int}} q'$. Si $q \sim_{\Sigma_\mathit{Int}} q'$ no vale, entonces no existen $q'' \in Q$,  $s, s' \in \Sigma^*$ tal que $\widehat{s}_{\Sigma'} = \widehat{s'}_{\Sigma'}$, $q \xrightarrow{s} q'$ y $q \xrightarrow{s'} q''$, lo que es absurdo pues se puede tomar $q'' = q$, $s = \epsilon$ y $s' = s\sigma$ satisfaciendo que $\widehat{s}_{\Sigma_\mathit{Int}} = \widehat{s'}_{\Sigma_\mathit{Int}}$
\end{proof}

\begin{definition}[Equivalencia por comunicación interna]
\cnote{Modificada.}Dado un AFCA $A = \langle Q, \mathcal{C}, B, \Sigma, \delta, q_0, F \rangle$ weak determinate tal que $\Sigma_\mathit{Int} \subseteq \Sigma$ denota el conjunto de transiciones internas, definimos $q \Rightarrow q'$ si y solo si existe $\sigma \in \Sigma_\mathit{Int}$ tal que $q \xrightarrow{\sigma} q'$. Luego, $[q]_m$, \emph{la clase de equivalencia por comunicación interna} de $q$, se define como $[q]_m = \{q'\ |\ q \Rightarrow^\bullet q'\}$. 
\end{definition}

Intuitivamente, la clase de equivalencia por comunicación interna de un estado $q$, $[q]_m$ son aquellos estados $q'$ relacionados en cero o más pasos de transiciones internas con $q$.

Dado un AFCA $A = \langle Q, \mathcal{C}, B, \Sigma, \delta, q_0, F \rangle$, $\Sigma = \Sigma_\mathit{Int} \cup \Sigma_\mathit{Buff}$ y $q \in Q$, . Decimos que dos clases de equivalencia están relacionadas $[q]_m \xrightarrow{\sigma} [q']_m$ sii $\exists q_i \in [q]_m, q'_i \in [q']_m$ tales que $q_i \xrightarrow{\sigma} q'_i \in \delta_{\text{Buff}}$.

\begin{definition}[Proyección de comunicación interna]\label{def:pci}\ \\
Dado un AFCA $\mathcal{A} = \langle Q, B, \mathcal{C}, \Sigma, \delta, q_0, F \rangle$, definimos $\Tau(\mathcal{A})$ como la proyección de la semántica de comunicación interna de $\mathcal{A}$. Llamamos $\widehat{\mathcal{A}}$ al sistema de transición etiquetado resultante y decimos que $\Tau (\langle Q, B, \mathcal{C}, \Sigma, q_0, \delta, F \rangle)= \langle \widehat{Q}, B, \widehat{q_0}, \widehat{\Delta} \rangle$ tal que:
\begin{itemize}
    \item $ \widehat{Q} = \{ \langle [q]_m, \overrightarrow{\Omega} \rangle | q \in Q \land \overrightarrow{\Omega} = (\Omega_b)_{b \in B}$ con $\Omega_b \in \mathcal{M}^* \}$ Cada estado del sistema de transición resultante es una configuración de $\mathcal{A}$ para una clase de equivalencia por comunicación interna,
    \item $B$ es el conjunto de buffers del LTS, que son los mismos de comunicación interna de $\mathcal{A}$,
    \item $\widehat{q_0}= \langle q_0, \Omega \rangle$ es el estado inicial de  donde $q_0$ es el estado inicial de $\mathcal{A}$ y $\Omega = \langle [], \ldots, [] \rangle$ es el conjunto de buffers en su estado inicial sin mensajes, y
    \item $\widehat{\Delta}= \{\langle \langle q_i, \Omega \rangle, \widehat{\sigma},\langle q_j, \Omega' \rangle \rangle | q_i, q_j \in Q, \widehat{\sigma} \in \{B \times \{\gg, \ll\} \times \mathcal{M}\} \land \exists \sigma \in \Sigma$ tal que  $[q_i]_m \xrightarrow{\sigma} [q_j]_m \}$. Definimos $\widehat{\Delta}$ como la relación de transición de $\widehat{\mathcal{A}}$. Un estado de $\widehat{q_j}$ es alcanzable desde otro $\widehat{q_i}$ si la clase de equivalencia $[q_j]_m$ es alcanzable desde $[q_i]_\mathcal{M}$.
\end{itemize}
\end{definition}

Antes de pasar a la demostración de equivalencia entre la semánticas de un conjunto compatible de AFCAs

\begin{lemma}[Equivalencia de estados discretos] Sea $\widehat{\mathcal{A}}=\langle \widehat{Q}, B, \widehat{q_0}, \widehat{\delta} \rangle$, la proyección de comunicación interna del AFCA $\mathcal{A}$ que surge de la composición del conjunto de autómatas  $\{\mathcal{A}\}_{1 \leq i \leq n}$  y $M= \langle [Q], \mathcal{C}, [q_0], [\delta] \rangle$, el LTS que representa la semántica del CS correspondiente a $\{M_i\}_{1 \leq i \leq n}$, el conjunto de mCFSM de $\{\mathcal{A}\}_{1 \leq i \leq n}$. Para todo $\langle [\overrightarrow{q}]_m, \overrightarrow{\Omega} \rangle \in \widehat{Q}$, existe $\langle \overrightarrow{p}, \overrightarrow{\Omega_{\mathcal{C}}} \rangle \in P$ tal que $q_i \in \overrightarrow{q} \iff q_i \in \overrightarrow{p}$ y viceversa. 
\end{lemma}
\begin{proof} 
\begin{itemize}
    \item[$\implies$] Dado $\widehat{q} \in \widehat{Q}$, un estado de la proyección $\widehat{\mathcal{A}}$. Por definición de $\widehat{\mathcal{A}}$, $\widehat{q}= \langle [\overrightarrow{q}]_m, \overrightarrow{\Omega} \rangle$. Por definición de equivalencia por comunicación interna $[\overrightarrow{q}]_m = \left\{\overrightarrow{q'}\ |\ \overrightarrow{q} \xrightarrow{\sigma}\overrightarrow{q'}\right\}$  con $\overrightarrow{q},\overrightarrow{q'} \in Q$ y $\sigma \in \Sigma_\mathit{Int}^*$ o sea son los estados alcanzables por transiciones internas desde $\overrightarrow{q}$. Por lo tanto $\langle [\overrightarrow{q}]_m, \overrightarrow{\Omega} \rangle \in \widehat{Q}$ implica $\overrightarrow{q} \in Q$. Por definición de composición de AFCA (Def.~\ref{def:composicion}) $\overrightarrow{q} = \langle q_1, \ldots, q_i, \ldots, q_n \rangle$ donde $q_i \in Q_i$ es un estado del AFCA $\mathcal{A}_i$. Como el conjunto de AFCA $\{\mathcal{A}_i\}_{1 \leq i \leq n}$ no posee comunicación interna, la comunicación interna de $\mathcal{A}$ es la comunicación externa de sus componentes internalizada a través de la composición. Por lo tanto, $q_i$ es un estado alcanzable por una sucesión de transiciones internas y externas de $\mathcal{A}_i$. Entonces $\overrightarrow{q} \in Q$ si y solo si $q_i \in Q_i$, para todo $1 \leq i \leq n$, y por definición de interfaz de comunicación de AFCA (ver Def.~\ref{def:interfazAFCA}), existe $[q_i] \in [Q_i]$, el conjunto de estados de la mCFSM $M_i$. Por lo tanto, existe $\overrightarrow{p} = \langle [q_1], \ldots, [q_i], \ldots, [q_n] \rangle$ tal que, para algún $\overrightarrow{\Omega_\mathcal{C}}$, $\langle \overrightarrow{p}, \overrightarrow{\Omega_\mathcal{C}} \rangle \in P$.
    
    \item[$\impliedby$] Sea $p \in P$, un estado del LTS $M$ que representa la semántica el CS. Por definición de CS vale $p = \langle \langle [q_1],\ldots,[q_i], \ldots, [\q_n] \rangle, \overrightarrow{\Omega} \rangle \in P \iff [q_i] \in [Q_i]$, donde $[Q_i]$ es el conjunto de estados de la mCFSM $M_i$. Sea $\mathcal{A}_i$ el AFCA tal que $M_i$ es su interfaz de comunicación. Por definición de interfaz de comunicación existe un estado alcanzable por una transición de comunicación externa $q_i \in Q_i$. Por definición de composición $q_i \in Q_i \iff \langle q_1, q_2, \ldots, q_i, \ldots, q_n \rangle \in Q$ y la transición de comunicación externa es internalizada. Por lo tanto $\overrightarrow{q}$ es alcanzable por una transición de comunicación interna y tiene su clase de comunicación interna $[\overrightarrow{q}]_m$. Entonces existe $\widehat{q} = \langle [\overrightarrow{q}]_m, \overrightarrow{\Omega} \rangle \in \widehat{Q}$
\end{itemize}
\end{proof}

\begin{lemma}[Equivalencia entre buffers y canales]\label{lemma2} Sean $\widehat{\mathcal{A}}=\langle \widehat{Q}, B, \widehat{q_0}, \widehat{\delta} \rangle$, la proyección de comunicación interna del AFCA $\mathcal{A}$ que surge de la composición del conjunto de autómatas  $\{\mathcal{A}\}_{1 \leq i \leq n}$  y $M= \langle P, \mathcal{C}, p_0, [\delta] \rangle$, el LTS que representa la semántica del CS correspondiente a las mCFSM del conjunto. Dados $\langle [\overrightarrow{q}]_m, \overrightarrow{\Omega} \rangle \in \widehat{Q}$ y $\langle \overrightarrow{p}, \overrightarrow{\Omega_{\mathcal{C}}} \rangle \in P$. Queremos ver que el conjunto de buffers $B$ de $\mathcal{A}$ es el conjunto de canales internalizados de $\mathcal{C}$ de $M$. Es decir que para todo canal $\omega$ vale que $\omega \in \mathcal{C} \iff \omega \in B$.
\end{lemma}
\begin{proof} Sea $\langle \overrightarrow{p}, \overrightarrow{\Omega_{\mathcal{C}}} \rangle$ un estado del LTS $M$. Por definición de CS $\overrightarrow{\Omega_{\mathcal{C}}}$ es el conjunto de canales $\mathcal{C}$ con su contenido en el estado discreto $\overrightarrow{p}$. Estos canales son los canales de las mCFSM $\{M_i\}_{1 \leq i \leq n}$ que componen el CS. Entonces vale $(\omega \in \overrightarrow{\Omega_{\mathcal{C}}}) \iff (\omega \in \mathcal{C}) \iff (\exists M_i \in \{M_i\}_{1 \leq i \leq n} \land  \omega \in \mathcal{C}_i)$. Sea $\mathcal{A}_i$ el AFCA tal que $M_i$ es su interfaz de comunicación. Por definición de interfaz de comunicación de AFCA $\omega \in \mathcal{C}_i \iff \omega \in {\mathcal{C}}_{\mathcal{A}_i}$. Por definición de composición de AFCA, la comunicación externa se internaliza. Entonces como $\mathcal{A}_i$ no tienen buffers ni comunicación interna, vale $\omega \in {\mathcal{C}}_{\mathcal{A}_i} \iff \omega \in B$ donde $B$ es el conjunto de buffers de comunicación interna de $\mathcal{A}$. Luego por definición de proyección de comunicación interna el conjunto de buffers de $\mathcal{A}$ es el mismo que en $\widehat{\mathcal{A}}$. Por lo tanto $\omega \in B \iff \exists \langle [\overrightarrow{q}]_m, \overrightarrow{\Omega} \rangle \in \widehat{Q}$ tal que $\omega \in \overrightarrow{\Omega}$.
\end{proof}


\begin{theorem}[Equivalencia de las semánticas de las mFCSMs y los AFCAs] Sea $\{\mathcal{A}\}_{1 \leq i \leq n}$ un conjunto de weak deterministic AFCA sin transiciones de comunicación interna, tal que su composición $\mathcal{A} = \langle Q, B, \mathcal{C},\Sigma, \delta, q_0, F \rangle$, también lo es. Llamamos $\widehat{\mathcal{A}}=\langle \widehat{Q}, B, \widehat{q_0}, \widehat{\delta} \rangle$ al sistema de transición etiquetado (LTS) que resulta de la proyección $\Tau(\mathcal{A})$. Sea el conjunto $\{M_i\}_{1 \leq i \leq n} \langle [Q_i], \mathcal{C}, [{q_0}_i], [\delta_i] \rangle$ de conformado por las mCFSM correspondientes al conjunto AFCA. Llamamos $M = \langle P, \mathcal{C}, {p_0}, [\delta] \rangle$ al LTS que representa la semántica del CS. Luego $\widehat{\mathcal{A}}$ y $M$ son bisimilares. 
\end{theorem}

\begin{proof}
Queremos probar que la relación $R \subseteq \widehat{Q} \times P$, definida de la siguiente manera $R = \{ \langle \langle [\overrightarrow{q}]_m, \overrightarrow{\Omega} \rangle, \langle \overrightarrow{p}, \overrightarrow{\Omega_{\mathcal{C}}} \rangle \rangle \ | \ [\overrightarrow{q}]_mR'\overrightarrow{p} \land  \overrightarrow{\Omega}=\overrightarrow{\Omega_{\mathcal{C}}} \}$ es una bisimulación, con $R'=\{\langle [\overrightarrow{q}]_m, \overrightarrow{p} \rangle \ | \ q_i \in \overrightarrow{q} \iff q_i \in \overrightarrow{p} \}$. Donde $P$ es el conjunto de estados de $M$ y para todo $\langle [\overrightarrow{q}]_m, \overrightarrow{\Omega} \rangle \in \widehat{Q}$, $\langle \overrightarrow{p}, \overrightarrow{\Omega_{\mathcal{C}}} \rangle \in P$, $\langle \langle [\overrightarrow{q}]_m, \overrightarrow{\Omega} \rangle, \langle \overrightarrow{p}, \overrightarrow{\Omega_{\mathcal{C}}} \rangle \rangle \in R$ si y solo si $[\overrightarrow{q}]_mR'\overrightarrow{p}$ y $\overrightarrow{\Omega}=\overrightarrow{\Omega_{\mathcal{C}}}$. \\

Es decir que para cada par de elementos $\widehat{q} \in \widehat{Q}$, $p \in P$, vale $\langle \widehat{q}, p \rangle \in R$ y $\forall \widehat{\sigma} \in \widehat{\Sigma} | \widehat{q} \xrightarrow{\widehat{\sigma}} \widehat{q}' \implies \exists p' \in P$ tal que $p \xrightarrow{\sigma_M} p' \land \langle \widehat{q}', p', \rangle \in R$, y simétricamente $\forall \sigma_M \in \Sigma_M | p \xrightarrow{\sigma_M} p \implies \exists \widehat{q}' \in \widehat{Q}, \widehat{\sigma} \in \widehat{\Sigma}$ tal que $\widehat{q} \xrightarrow{\widehat{\sigma}} \widehat{q}' \land \langle \widehat{q}', p'\rangle \in R$. Donde $\widehat{\Sigma} \subseteq B \times \{\gg,\ll\} \times \mathcal{M}$, $\Sigma_M \subseteq \mathcal{A} \times \{!,?\} \times \mathcal{M}$ y el conjunto de buffers $B$ son los canales internalizados del conjunto $\mathcal{C}$. 

Tomamos inicialmente como etiquetas $\widehat{\sigma} = (a_ka_j)_l \gg m$ y $\sigma_M = (a_ka_j)_l?m$, 

Primero demostraremos la condición $\forall (a_ka_j)_l \gg m \in \widehat{\Sigma} | \widehat{q} \xrightarrow{(a_ka_j)_l \gg m} \widehat{q}' \implies \exists p' \in P$ tal que $p \xrightarrow{(a_ka_j)_l?m} p' \land \langle \widehat{q}', p', \rangle \in R$. 
 
\begin{itemize}
    
    \item Dado el par $\langle [\overrightarrow{q}]_m, \overrightarrow{\Omega} \rangle \in \widehat{Q}, \langle \overrightarrow{p}, \overrightarrow{\Omega_\mathcal{C}} \rangle \in P$ tal que $\langle \langle [\overrightarrow{q}]_m, \overrightarrow{\Omega} \rangle, \langle \overrightarrow{p}, \overrightarrow{\Omega_\mathcal{C}} \rangle \rangle \in R$. Sean $(a_ka_j)_l \gg m \in \widehat{\Sigma}$ y $\langle [\overrightarrow{q}']_m, \overrightarrow{\Omega}' \rangle \in \widehat{Q}$ tales que $\langle [\overrightarrow{q}]_m, \overrightarrow{\Omega} \rangle \xrightarrow{(a_ka_j)_l \gg m} \langle [\overrightarrow{q}']_m, \overrightarrow{\Omega}' \rangle \in \widehat{\Delta}$.  Esto vale si y solo si la clase de equivalencia por comunicación interna $[\overrightarrow{q}']_m$ es alcanzable desde $[\overrightarrow{q}]_m$ vía $(a_ka_j)_l \gg m$. Es decir $[\overrightarrow{q}]_m \xrightarrow{(a_ka_j)_l \gg m} [\overrightarrow{q}']_m$. Entonces $\overrightarrow{\Omega} = [(a_1a_2), \ldots, (a_ka_j)_l \cdot m \ldots (a_na_m)]$ y $\overrightarrow{\Omega}' = [(a_1a_2), \ldots, (a_ka_j)_l \ldots (a_na_m)] \rangle$.
    
    \item Sean $\overrightarrow{q_{\sigma}} \in [\overrightarrow{q}]_m, \overrightarrow{q_{\sigma}}' \in [\overrightarrow{q}']_m$, por definición de proyección de comunicación vale $\overrightarrow{q_{\sigma}} \xrightarrow{(a_ka_j)_l \gg m} \overrightarrow{q_{\sigma}}' \in \delta$. Por definición de composición sabemos que vale $\overrightarrow{q_{\sigma}} =\langle q_0, q_1, \ldots, q_j, \ldots,q_n \rangle$ y $\overrightarrow{q_{\sigma}}' =\langle q_0, q_1, \ldots, q'_j, \ldots,q_n \rangle$ donde $q_j,q'_j$ son estados de $\mathcal{A}_j$, componente de $\mathcal{A}$. Por definición de AFCA, $(a_ka_j)_l \gg m$ es una operación de desencolamiento del mensaje $m$ sobre el buffer $(a_ka_j)_l$. Como los AFCA del conjunto no tienen acciones internas esto corresponde a una recepción de un mensaje por parte de $\mathcal{A}_j$ vía el canal $(a_ka_j)_l$. Entonces podemos afirmar que se cumple $\overrightarrow{q_{\sigma}} \xrightarrow{(a_ka_j)_l \gg m} \overrightarrow{q_{\sigma}}'\in \delta \iff q_j,q'_j \in Q_j \land q_j \xrightarrow{(a_ka_j)_l?m} q'_j \in \delta_j$.
    
    \item Por la interfaz de comunicación de AFCA podemos obtener $M_j$, la mCFSM correspondiente a $\mathcal{A}_j$, y sabemos que $q_j \xrightarrow{(a_ka_j)_l?m} q'_j \in \delta_j \iff$ se cumple que $[q_j], [q'_j] \in [Q_j] \land [q_j] \xrightarrow{(a_ka_j)_l?m} [q'_j] \in [\delta_j]$. 
    
    \item Entonces, por definición del LTS M, existen $p'=\langle \overrightarrow{p'}, \overrightarrow{\Omega_\mathcal{C}} \rangle \in P, (a_ka_j)_l?m \in \Sigma_M$ tal que $\overrightarrow{p'}=\langle [q_1], [q_2], \ldots, [q'_j], \ldots [q_n] \rangle$, $\overrightarrow{\Omega'_\mathcal{C}}=[\ldots, (a_ka_j)_l, \ldots]$, y $p \xrightarrow{(a_ka_j)_l?m} p'$. Donde $\overrightarrow{\Omega_\mathcal{C}}$ y $\overrightarrow{\Omega_\mathcal{C}}'$ son los canales de comunicación externa para los estados $\overrightarrow{p}$ y $\overrightarrow{p'}$, y vale $\overrightarrow{\Omega_\mathcal{C}} = [(a_1a_2), \ldots, (a_ka_j)_l \cdot m \ldots (a_na_m)]$ y $\overrightarrow{\Omega_\mathcal{C}}' = [(a_1a_2), \ldots, (a_ka_j)_l \ldots (a_na_m)] \rangle$. Por lo tanto vale $\langle \widehat{q'}, p' \rangle \in R$.
    
    \item Por lo tanto vale que $\forall (a_ka_j)_l \gg m \in \widehat{\Sigma} | \widehat{q} \xrightarrow{(a_ka_j)_l \gg m} \widehat{q}' \implies \exists p' \in P$ tal que $p' \xrightarrow{(a_ka_j)_l?m} p' \land \langle \widehat{q}', p', \rangle \in R$ donde $\widehat{\Sigma} \subseteq B \times \{\gg,\ll\} \times \mathcal{M}$
  \end{itemize}
  
  Demostraremos ahora la condición simétrica $\forall (a_ka_j)_l?m \in \Sigma_M | p \xrightarrow{(a_ka_j)_l?m} p' \implies \exists \widehat{q}' \in \widehat{Q}, (a_ka_j)_l \gg m \in \widehat{\Sigma}$ tal que $\widehat{q} \xrightarrow{(a_ka_j)_l \gg m} \widehat{q}' \land \langle \widehat{q}', p'\rangle \in R$ 
  \begin{itemize}

     \item Dado el par $\langle q, p \rangle = [\overrightarrow{q}]_m, \overrightarrow{\Omega} \rangle \in \widehat{Q}, \langle \overrightarrow{p}, \overrightarrow{\Omega_\mathcal{C}} \rangle \in P$ tal que $\langle \langle [\overrightarrow{q}]_m, \overrightarrow{\Omega} \rangle, \langle \overrightarrow{p}, \overrightarrow{\Omega_\mathcal{C}} \rangle \rangle \in R$. Sean $(a_ka_j)_l?m$ y $p' = \langle \overrightarrow{p}', \overrightarrow{\Omega_\mathcal{C}}' \rangle \in P$ tales que $\langle \overrightarrow{p}, \overrightarrow{\Omega_\mathcal{C}} \rangle \xrightarrow{(a_ka_j)_l?m} \langle \overrightarrow{p}', \overrightarrow{\Omega_\mathcal{C}}' \rangle \in \delta_P$. Por definición del LTS $M$, $(a_ka_j)_l?m$ es una operación sobre el canal $(a_ka_j)_t$ de la mCFSM $M_j$, componente del CS. Como el LTS $M$ representa la semántica del CS compuesto por el conjunto de mCFSMs, vale que $\overrightarrow{p}=\langle [q_1], [q_2], \ldots [q_j], \ldots [q_n] \rangle$, $\overrightarrow{p}'= \langle [q_1], [q_2], \ldots, [q'_j], \ldots [q_n] \rangle$, donde $[q_j]$ y $[q'_j]$ son estados de la mCFSM $M_j$, y vale $[q_j] \xrightarrow{(a_ka_j)_l?m} [q'_j] \in [\delta]_j$
    
    \item Por definición de intefaz de comunicación de AFCA sabemos que $M_J$ es la mCFSM correspondiente a $\mathcal{A}_j$, componente de $\mathcal{A}$. Por lema 1, como $[q_j] \xrightarrow{(a_ka_j)_l?m} [q'_j] \in [\delta_j]$ entonces $q_j \xrightarrow{(a_ka_j)_l?m} q'_j \in \delta_j$.
    
    \item Por definición de composición $q_j$ y $q'_j$ son componentes de dos estados compuestos que llamamos $\overrightarrow{q}_{\sigma}$ y $\overrightarrow{q}_{\sigma}'$, respectivamente. De modo tal que $\overrightarrow{q}_{\sigma} \xrightarrow{(a_ka_j)_l \gg m} \overrightarrow{q}_{\sigma}' \in \delta$. Donde $(a_ka_j)_l$ es el canal internalizado en forma de buffer.
    
    \item Por definición de equivalencia por comunicación interna como $\overrightarrow{q}_{\sigma} \xrightarrow{(a_ka_j)_l \gg m} \overrightarrow{q}_{\sigma}' \in \delta$ entonces existen las clases de equivalencia $[\overrightarrow{q}]_m$, $[\overrightarrow{q}']_m$, tales que $\overrightarrow{q}_{\sigma} \in [\overrightarrow{q}]_m \land \overrightarrow{q}_{\sigma}' \in [\overrightarrow{q}']_m$ y vale $[\overrightarrow{q}]_m \xrightarrow{(a_ka_j)_l \gg m} [\overrightarrow{q}']_m$. Por proyección de comunicación entonces existen $\widehat{q}' = \langle [\overrightarrow{q}']_m, \overrightarrow{\Omega}' \rangle \in \widehat{Q}, (a_ka_j)_l \gg m \in \widehat{\Sigma}$, tales que $\langle [\overrightarrow{q}]_m, \overrightarrow{\Omega} \rangle \xrightarrow{(a_ka_j)_l \gg m} \langle [\overrightarrow{q}']_m, \overrightarrow{\Omega}' \rangle$. Donde, $\overrightarrow{\Omega}$, y $\overrightarrow{\Omega}'$ son los buffers de $\mathcal{A}$ para los estados $\overrightarrow{q}_{\sigma}$ y  $\overrightarrow{q}_{\sigma}'$, respectivamente, y valen $\overrightarrow{\Omega} = [(a_1a_2), \ldots, (a_ka_j)_l \cdot m \ldots (a_na_m)]$ y $\overrightarrow{\Omega}' = [(a_1a_2), \ldots, (a_ka_j)_l \ldots (a_na_m)] \rangle$. Por lo tanto $\langle \widehat{q}', p' \rangle \in R$ 
    
    % \item Por lo tanto vale en forma simétrica que $\forall (a_ka_j)_l?m \in \Sigma_M | p \xrightarrow{(a_ka_j)_l?m} p \implies \exists \widehat{q}' \in \widehat{Q}, (a_ka_j)_l \gg m \in \widehat{\Sigma}$ tal que $\widehat{q} \xrightarrow{(a_ka_j)_l \gg m} \widehat{q}' \land \langle \widehat{q}', p'\rangle \in R$ donde $\Sigma_M \subseteq \mathcal{C} \times \{!,?\} \times \mathcal{M}$
    
\end{itemize}
\end{proof}


\begin{figure}[h]
\begin{example}[Transformación de Autómatas Finitos de Comunicación Asincrónica]
\label{ex:Transformación}
Consideremos los siguientes AFCA 
\begin{center}
\begin{tikzpicture}[->, thick]
 \node[state,initial] (q_0)   {$q_0$}; 
 \node[state] (q_1) [right= 1.5cm of q_0 ] {$q_1$};
 \node[state] (q_2) [below= of q_0 ] {$q_2$};
 \node[state] (q_3) [right= 1.5cm of q_2 ] {$q_3$};
 \node[state, accepting] (q_4) [below= of q_3 ] {$q_4$};
	
 \draw[]        
        (q_0) edge[above] node{out(pr,a)} (q_1)
        (q_0) edge[left] node{in(sp,b)} (q_2)
        (q_1) edge[right] node{in(sp,b)} (q_3)
        (q_2) edge[above] node{out(pr,a)} (q_3)
        (q_3) edge[right] node{$int_p$} (q_4)
        ; 
\end{tikzpicture} 
\qquad
\begin{tikzpicture}[->, thick]
 \node[state,initial] (q_0)   {$q_0$}; 
 \node[state] (q_1) [below= of q_0 ] {$q_1$};
 \node[state, accepting] (q_2) [below= of q_1 ] {$q_2$};
 \draw[]        
        
        (q_0) edge[right] node{$int_r$} (q_1)
        (q_1) edge[right] node{in(pr,a)} (q_2)
        ;
\end{tikzpicture} 
\qquad
\begin{tikzpicture}[->, thick]
 \node[state,initial] (q_0)   {$q_0$}; 
 \node[state] (q_1) [below= of q_0 ] {$q_1$};

 \draw[]        
        
        (q_0) edge[right] node{out(sp,b)} (q_1)
        
        ;
\end{tikzpicture} 

\end{center}

El autómata compuesto resultante es el AFCA $\mathcal{A}$  \\  
\begin{tikzpicture}[->, thick]
 \node[state,initial] (q_0)   {$q_{000}$}; 
 \node[state] (q_1) [right= 1.5cm of q_0 ] {$q_{100}$};
 \node[state] (q_2) [right = 1.5 of q_1 ] {$q_{110}$};
 \node[state] (q_3) [right = 1.5 of q_2 ] {$q_{120}$};
  \node[state] (q_4) [right = 1.5 of q_3 ] {$q_{121}$};
  \node[state] (q_5) [below=  of q_0] {$q_{001}$};
  \node[state] (q_6) [right = 1.5cm of q_5] {$q_{201}$};
  \node[state] (q_7) [right = 1.5 of q_6 ] {$q_{301}$};
  \node[state] (q_8) [right = 1.5 of q_7 ] {$q_{311}$};
  \node[state] (q_9) [right = 1.5 of q_8 ] {$q_{321}$};
 \node[state,accepting] (q_10) [right= 1.5cm of q_9 ] {$q_{421}$};
 \draw[]        
        (q_0) edge[above] node{$PR \ll a$} (q_1)
        (q_1) edge[above] node{$int_r$} (q_2)
		(q_2) edge[above] node{$PR \gg a$} (q_3)
        (q_3) edge[above] node{$SP \ll b$} (q_4) 
        (q_4) edge[right] node{$SP \gg b$} (q_9)
        (q_0) edge[left] node{$SP \ll b$} (q_5)
        (q_5) edge[below] node{$SP \gg b$} (q_6)
        (q_6) edge[below] node{$PR \ll a$} (q_7)
        (q_7) edge[below] node{$int_r$} (q_8)
        (q_8) edge[below] node{$PR \gg a$} (q_9)
        (q_9) edge[below] node{$int_p$} (q_10)
        ;
\end{tikzpicture}
    \caption{Transformación}
    \label{fig:transformacion}
\end{example}
\end{figure}


\begin{figure}[h]
\begin{example}[CS de transformación]
\label{ex:cmst} El CS correspondiente al conjunto de autómatas es el siguiente
\centering

\begin{tikzpicture}[->, thick]
 \node[state,initial] (q_0)   {$q_0$}; 
 \node[state] (q_1) [right= of q_0 ] {$q_1$};
  \node[state] (q_2) [below= of q_0 ] {$q_2$};
 \node[state] (q_3) [right= of q_2 ] {$q_3$};

 \draw[]        
        (q_0) edge[above] node{pr!a} (q_1)
        (q_0) edge[right] node{sp?b} (q_2)
        (q_1) edge[right] node{sp?b} (q_3)
        (q_2) edge[above] node{pr!a} (q_3)
        ; 
\end{tikzpicture} 
\qquad
\begin{tikzpicture}[->, thick]
 \node[state,initial] (q_0)   {$q_0$}; 
 \node[state] (q_1) [below= of q_0 ] {$q_2$};

 \draw[]        
        
        (q_0) edge[right] node{pr?a} (q_1)
        
        ;
\end{tikzpicture} 
\qquad
\begin{tikzpicture}[->, thick]
 \node[state,initial] (q_0)   {$q_0$}; 
 \node[state] (q_1) [below= of q_0 ] {$q_1$};

 \draw[]        
        
        (q_0) edge[right] node{sp!b} (q_1)
        
        ;
\end{tikzpicture} 

La semántica del CS es el sistema de transición etiquetado $\mathcal{L}$
\\
\begin{tikzpicture}[->, thick][scale=0.20]
 \node[state,initial] (q_0)   {$q_{000},\epsilon,\epsilon$}; 
 \node[state] (q_1) [right= 3cm of q_0 ] {$q_{100},a, \epsilon$};
 \node[state] (q_2) [right = 3cm of q_1 ] {$q_{120},\epsilon, \epsilon$};
 \node[state] (q_3) [below = of q_2 ] {$q_{121}, \epsilon,b$};
 \node[state,accepting] (q_4) [below=  of q_3] {$q_{321}, \epsilon,\epsilon$};
 \node[state] (q_5) [below = of q_0] {$q_{001}, \epsilon, b$};
 \node[state] (q_6) [below = of q_5 ] {$q_{201}, \epsilon, \epsilon$};
 \node[state] (q_7) [right= 3cm of q_6 ] {$q_{301}, a, \epsilon$};
 \draw[]        
        (q_0) edge[above] node{$\langle q_{0p},PR!a,q_{1p} \rangle$} (q_1)
        (q_1) edge[above] node{$\langle q_{0r},PR?a,q_{1r} \rangle$} (q_2)
        (q_2) edge[right] node{$\langle q_{0s},SP!b,q_{1s} \rangle$} (q_3)
        (q_3) edge[right] node{$\langle q_{1p},SP?b,q_{3p} \rangle$} (q_4)
        (q_0) edge[left] node{$\langle q_{0s},SP!b,q_{1s} \rangle$} (q_5)
        (q_5) edge[left] node{$\langle q_{0p},SP?b,q_{2p} \rangle$} (q_6)
        (q_6) edge[below] node{$\langle q_{2p},PR!a,q_{3p} \rangle$} (q_7)
        (q_7) edge[below] node{$\langle q_{0r},PR?a,q_{1r} \rangle$} (q_4)
        ;
\end{tikzpicture}
    \caption{Transformación}
    \label{fig:semántica mcfsm}
\end{example}
\end{figure}

\begin{figure}[ht]
    \begin{centering}
      \begin{tikzpicture}[->, thick]
         \node[state,initial] (q_0) {$[q_{000}],\epsilon, \epsilon$}; 
         \node[state] (q_1) [right= 3.5cm of q_0 ] {$[q_{100}], a, \epsilon$};
         \node[state] (q_2) [right = 3.5cm of q_1 ] {$[q_{120}],\epsilon,\epsilon$};
         \node[state] (q_3) [below= of q_2 ] {$[q_{121}],\epsilon, b$};
         \node[state] (q_4) [below=  of q_0] {$[q_{001}],\epsilon, b$};
         \node[state] (q_5) [below = of q_4] {$[q_{201}],\epsilon, \epsilon$};
         \node[state] (q_6) [right = 3.5cm of q_5 ] {$[q_{301}], a, \epsilon$};
         \node[state,accepting] (q_7) [below = of q_3 ] {$[q_{321}],\epsilon, \epsilon$};
    \draw[]        
         (q_0) edge[above] node{$\langle q_{000}, PR \ll a, q_{100} \rangle$} (q_1)
    	 (q_1) edge[above] node{$\langle q_{110}, PR \gg a, q_{120} \rangle$} (q_2)
         (q_2) edge[right] node{$\langle q_{120}, SP \ll b, q_{121} \rangle$} (q_3) 
         (q_3) edge[right] node{$\langle q_{121}, SP \gg b, q_{321} \rangle$} (q_7)
         (q_0) edge[left] node{$\langle q_{000}, SP \ll b, q_{001} \rangle$} (q_4)
         (q_4) edge[left] node{$\langle q_{001}, SP \gg b, q_{201} \rangle$} (q_5)
         (q_5) edge[below] node{$\langle q_{201}, SP \gg b, q_{301} \rangle$} (q_6)
        (q_6) edge[below] node{$\langle q_{311}, PR \ll a, q_{321} \rangle$} (q_7)
         ;
        \end{tikzpicture}
    \caption{Caption}
    \label{fig: transformación loca}
 \end{centering}
\end{figure}

\newpage


