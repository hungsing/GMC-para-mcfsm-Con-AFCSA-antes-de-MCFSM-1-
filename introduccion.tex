%!TEX root = ./main.tex
\chapter{Introducción}
La tendencia hacia los sistemas distribuidos genera la necesidad de mecanismos de comunicación más complejos. Para manejar esta complejidad se han introducido lenguajes de especificación y métodos formales de análisis que permiten asegurar ciertas propiedades de dichos mecanismos como por ejemplo, las \emph{Communicating Finite State Machines} \cite{brand:jacm-30_2}, los \emph{Global Graphs} \cite{castagna:lmcs-8_1}, los \emph{Session types} en su gran diversidad de variantes \cite{honda:esop98,honda:popl08} y los \emph{Interface automata} \cite{dealfaro:esec-fse-01}. Tomando esto en consideración, los autómatas finitos \cite[Def.~2.2.1]{hopcroft01} resultan útiles como lenguaje primitivo de modelado, subyacente en muchos de los formalismos mencionados anteriormente, que permite representar algunos de estos aspectos de dichos sistemas.

Service oriented computing (SOC) es un paradigma de computación distribuida que cambió el modo en que los sistemas de software son concebidos. El corazón del paradigma son servicios que proveen elementos computacionales autónomos, independientes de la plataforma y que se ejecutan sobre una infraestructura de cómputo y comunicación existente. Estos pueden ser descritos, publicados, descubiertos y programados usando protocolos estándar para construir redes de aplicaciones que colaboran entre sí, incluso distribuidas dentro de distintas fronteras organizacionales, con la misión de, colectivamente, alcanzar un objetivo de negocios. Uno de los elementos centrales de este paradigma es que dichos elementos computacionales son procurados en tiempo de ejecución y bajo demanda; una demanda, resulta local a cada ejecución, lo que implica que no todos los servicios necesarios son al mismo tiempo y algunos, puede que ni siquiera lo sean.

Esta mirada sobre cómo un sistema de software evoluciona reconfigurándose en tiempo de ejecución pone de relieve la necesidad de contar con un lenguaje de descripción que posibilite un enfoque para la composición que sea parcial, y que el resultado de dicha composición resulte una descripción legítima de una componente; una característica que la gran mayoría de los lenguaje utilizados en la descripción de sistemas distribuidos no posee. Volveremos sobre esto en la Sec.~\ref{trabajo-relacionado} donde discutiremos otros lenguajes formales relacionados, que hemos mencionado más arriba, al final de la presente.\\

En este trabajo buscaremos definir un lenguaje formal de modelado de componentes de software con el objeto de satisfacer las necesidades mencionadas anteriormente. Este lenguaje debe satisfacer las siguientes propiedades:
\begin{inparaenum}[1.]
\item debe posibilitar la convivencia de elementos computacionales internos de una componente (transiciones que expresan cambios locales de estado) con su interfaz de comunicación (transiciones que expresan envío o recepción de mensajes),
\item debe poseer un mecanismo claro de composición que no requiera que todos los participantes, y
\item debe tener semántica de comunicación asincrónica, compatible con la semántica de los lenguajes formales conocidos como los mencionados anteriormente).
\end{inparaenum}

Para satisfacer estos objetivos definiremos una clase de autómatas finitos, a la que llamaremos \emph{Autómatas Finitos de Comunicación Asincrónica} (AFCA), que cuentan con transiciones internas (denotando cambios locales de estado) y transiciones de comunicación sobre canales de comunicación (que expresan la comunicación con otras componentes), adicionalmente, estos autómatas pueden ser compuestos internalizando la comunicación a través de la creación de buffers dedicados que permiten reemplazar los canales de comunicación. Adaptaremos el lenguaje de las Communicating Finite State Machines, a las llamaremos \emph{multichannel Communicating Finite State Machines} (mCFSM), con el objeto de que sean capaces de reflejar la interfaz de comunicación de estos autómatas. Por último, probaremos la equivalencia entre la semántica de la composición de una familia de AFCAs y la del \emph{communicating system} obtenido a partir de la familia de CFSM correspondientes a cada uno de dichos AFCAs (ver Fig.~\ref{fig:preservacion}).\\


Esta tesis se divide en cuatro capítulos. Primero en el capítulo~\ref{preliminares} describimos las CFSM como lenguaje y la noción de Generalized Multiparty Compatibility (GMC, \cite{lange:popl15}) como condición suficiente para que un conjunto de CFSMs formen un communicating system seguro. En el capítulo~\ref{AFCA} definimos los AFCA, su composición, y la proyección de su interfaz de comunicación. También definimos mCFSM como extensión de CFSM y proyección de la interfaz de comunicación de un AFCA, y adaptamos la propiedad de GMC para este nuevo modelo. En el capítulo~\ref{resultados} exploramos la equivalencia de la semántica de un conjunto de AFCA y el communicating system resultante de su mCFSM proyectadas, y demostramos la validez de esta propiedad. Por último, en el capítulo~\ref{conclusiones} cerramos este trabajo con algunas conclusiones que hemos podido derivar del trabajo, presentamos brevemente distintos trabajos relacionados y algunas ideas para trabajo a futuro.


