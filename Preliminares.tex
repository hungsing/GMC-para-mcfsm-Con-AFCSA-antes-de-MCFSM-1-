%!TEX root = ./main.tex
\chapter{Preliminares}
\label{preliminares}
A continuación presentaremos algunas definiciones y resultados preliminares que serán de utilidad en las restantes secciones de esta tesis.

%\subsection{Definiciones básicas}

\begin{definition}[Sistema de transición etiquetado, \cite{keller:cacm-19_7}] \label{def:LTS}
Una estructura $S = \langle Q, \Sigma, \longrightarrow \rangle$ se dice que es un \emph{sistema de transición etiquetado} si satisface las siguientes propiedades:
\begin{itemize}
\item $Q$ es un conjunto finito llamado \emph{conjunto de estados},
\item $\Sigma$ es un conjunto finito llamado \emph{conjunto de etiquetas}, y
\item $\longrightarrow \subseteq Q \times \Sigma \times Q$.
\end{itemize}
Sean $q, q' \in Q$, $a \in \Sigma$, cuando $\langle q, a, q' \rangle \in \longrightarrow$ lo denotaremos como $q \xrightarrow{a} q'$. Denotaremos $\Sigma^*$ al conjunto de secuencias (finitas o infinitas) $\sigma_0, \sigma_1, \sigma_2, \ldots, \sigma_n, \ldots$ tal que para todo $i \in \nat$, $\sigma_i \in \Sigma$. A su vez, dado $q \in Q$, definiremos $\Sigma^* [q]$, llamado el lenguaje de $q$, como $\{ \sigma_0, \sigma_1, \ldots, \sigma_n, \ldots \in \Sigma^*\ |\ \text{existe } \{q_i\}_{i \in \nat} \text{ tal que } q = q_0 \text{ y para todo } i \in \nat \text{, } q_i \xrightarrow{\sigma_i} q_{i+1}\}$. Dados $q, q' \in Q$ y $\sigma \in \Sigma^*$, $q \xrightarrow{\sigma} q'$ si y solo si $\sigma = \sigma_0, \sigma_1, \ldots, \sigma_n$ y existen $\{q_i\}_{0 \leq i \leq n} \subseteq Q$ tal que:
\begin{inparaenum}[1-]
\item $q_0 = q$,
\item $q_n = q'$, y
\item para todo $0 \leq i < n$, $q_i \xrightarrow{\sigma_i} q_{i+1}$.
\end{inparaenum}
\end{definition}

\begin{definition}[Relaciones y clausuras] 
Sea $Q$ un conjunto y $R, S \subseteq Q \times Q$:
\begin{itemize}
\item $R^0 = \{\langle q, q \rangle \in Q \times Q\}$ (denominada \emph{relación identidad}),
\item $R^{-1} = \{\langle q', q \rangle \in Q \times Q\ |\ \langle q, q' \in R\}$ (denominada \emph{relación inversa}), 
\item $R \circ S = \{\langle q, q'' \rangle \in Q \times Q\ |\ \text{existe} q' \in Q \text{ tal que } \langle q, q' \rangle \in R \text{ y } \langle q', q'' \rangle \in S\}$ (denominada \emph{relación composición}), y
\item dado $i \in \nat$, $R^{i+1} = R^{i} \circ R$ (denominada \emph{$i$-ésima iteración}).
\end{itemize}
\begin{itemize}
    \item $S$ es la \emph{clausura reflexiva} de $R$ si $S = R^0 \cup R$,
    \item $S$ es la \emph{clausura simétrica} de $R$ si $S = R \cup R^{-1}$, y
    \item $S$ es la \emph{clausura transitiva} de $R$ si $S = \bigcup_{i \in \nat \text{ e } i > 0} R^i$ (denotada $R^+$).
\end{itemize}
$S$ se dice que es la \emph{clausura reflexiva y transitiva} si $S = R^0 \cup R^+$ (denotada como $R^*$) y se dice que es la \emph{clausura reflexiva, simétrica y transitiva} si $S = R^0 \cup R^{-1} \cup R^+$ (denotada como $R^\bullet$). Cuando una relación es reflexiva, simétrica y transitiva se dice que es una \emph{equivalencia}.
\end{definition}

\begin{definition}[Clases de equivalencia y cociente]
Sea $Q$ un conjunto y $R \subseteq Q \times Q$ una relación de equivalencia:
$Q|_R = \{s \in \wp(Q)\ |\ \text{para todo } q, q' \in s \text{, } \langle q, q' \rangle \in R \text{ y no existe } q'' \in \overline{s} \text{ tal que } \langle q, q'' \rangle \in R\}$. Dado $q \in Q$, denotaremos $[q] \in Q|_R$ al conjunto $s \in Q|_R$ tal que $q \in s$. 
\end{definition}

En lo subsiguiente introduciremos la noción de equivalencia entre estados que usaremos a lo largo del presente trabajo. Intuitivamente, solo deseamos que dos estados sean distinguibles si tienen la capacidad ser diferenciados por alguna continuación a partir de ellos. A continuación daremos una primera aproximación a esta noción de equivalencia.

\begin{definition}[Bisimulación \cite{milner89}, Chap.~4, Def.~1] \label{def:bisimulación} Dado un sistema de transición etiquetado $ S = \langle Q, \Sigma, \delta \rangle $, una \emph{bisimulación} es una relación binaria $R \subseteq Q \times Q$, tal que: para cada par de elementos $p, q \in Q$ tal que $\langle p, q \rangle \in R$ vale:
\begin{itemize}
    \item $(\forall \sigma \in \Sigma)(p \xrightarrow{\sigma} p' \implies (\exists q' \in Q)(q \xrightarrow{\sigma} q' \land \langle p', q' \rangle \in R))$ y, simétricamente, vale 
    \item $(\forall \sigma \in \Sigma)(q \xrightarrow{\sigma} q' \implies (\exists p' \in Q)(p \xrightarrow{\sigma} p' \land \langle p', q' \rangle \in R))$
\end{itemize}
Dados dos estados $p, q \in Q $, $p$ es bisimilar a $q$, si existe una bisimulación $R$ tal que $\langle p, q \rangle \in R$.
\end{definition}

Si observamos la definición anterior, dos estados que pertenecen a la relación de bisimulación no tienen posibilidad de ser distinguidos bajo ninguna posible sucesión de transciones comenzando en ellos. Si bien esta definición caracteriza una relación de equivalencia muy útil para razonar sobre sistemas de software, en nuestro caso no será suficiente dado que el tipo de autómatas con los que trabajaremos poseen etiquetas de diferente naturaleza. Para poder razonar sobre ellos se introducen las siguientes definiciones.

En \cite[Sec.~5.1, Def.~5]{milner89} Milner nos da una definición de \emph{bisimulación débil} basada en la eliminación de las transiciones silenciosas, o transiciones $\epsilon$ (ver \cite[Defs.~1 a~4]{milner89}). En nuestro caso, adaptaremos dicha definición para hacerla paramétrica en el conjunto de transiciones que deseamos silenciar a los efectos de determinar bisimilaridad débil.

\begin{definition}
Sea $\Sigma$ un conjunto de etiquetas, $\Sigma' \subseteq \Sigma$ y $\sigma \in \Sigma^*$, se define $\widehat{\sigma}_{\Sigma'}$ como la secuencia de etiquetas obtenida a partir de eliminar de $\sigma$ toda ocurrencia de etiquetas en $\Sigma'$.
\end{definition}

\begin{definition}[Bisimulación débil] Dado un sistema de transición etiquetado $ S = \langle Q, \Sigma, \delta \rangle $ y $\Sigma' \subseteq \Sigma$ el conjunto de etiquetas a silenciar, una \emph{bisimulación débil sobre $\Sigma'$} es una relación binaria $R \subseteq Q \times Q$, tal que: para cada par de elementos $p, q \in Q$ tal que $\langle p, q \rangle \in R$ vale:
\begin{itemize}
    \item $(\forall \sigma \in \Sigma \setminus \Sigma')(p \xrightarrow{\sigma} p' \implies (\exists q' \in Q)(\exists \sigma^* \in \Sigma^*)(q \xrightarrow{\sigma^*} q' \land \widehat{\sigma^*}_{\Sigma'} = \sigma \land \langle p', q' \rangle \in R))$ y, simétricamente, vale 
    \item $(\forall \sigma \in \Sigma \setminus \Sigma')(q \xrightarrow{\sigma} q' \implies (\exists p' \in Q)(\exists \sigma^* \in \Sigma^*)(p \xrightarrow{\sigma^*} p' \land \widehat{\sigma^*}_{\Sigma'} = \sigma \land \langle p', q' \rangle \in R))$
\end{itemize}
Dados dos estados $p, q \in Q $, $p$ es debilmente bisimilar sobre $\Sigma'$ a $q$, denotado como $p \sim_{\Sigma'} q$, si existe una bisimulación débil $R$ tal que $\langle p, q \rangle \in R$.
\end{definition}


\section{Comunicación Asincrónica}
Como se dijo en la primera sección, buscamos una implementación del paradigma de SOC donde la comunicación entre distintos componentes se realiza en forma asincrónica. Esto no es un requerimiento del paradigma propiamente dicho pero resulta una implementación más eficiente. El intercambio de mensajes en forma asincrónica permite a los componentes maximizar su uso de la cpaacidad de cómputo.  El intercambio de mensajes con cada componente es independiente entre sí, salvo que exista alguna necesidad de lo contrario. Para representar este comportamiento utilizamos un tipo de autómata finito denominado \emph{Communicating Finite State Machines} (llamado a partir de ahora CFSM). El concepto de CFSM fue introducido en \cite{brand:jacm-30_2} con el objetivo modelar y estudiar el comportamiento de sistemas distribuidos constituidos por un conjunto de procesos secuenciales que ejecutan concurrentemente y se comunican a partir de intercambiar mensajes a través de canales de comunicación previamente declarados. Las CFSMs son autómatas que modelan únicamente la comunicación externa entre participantes. De este modo cada CFSM representa los distintos componentes del sistema distribuido, pero solo el comportamiento que es relevante a la interacción entre los mismos. Una CFSM se puede ver como la proyección de este comportamiento específico de un autómata más complejo que tenga otro tipo de transiciones adicionales.

La naturaleza dinámica no pre programada de los sistemas SOC hace que se requiera algún mecanismo para verificar que todos los componentes puedan funcionar entre sí en forma correcta. Para esto en esta sección detallamos el concepto de \emph{Generalised multiparty compatibility} (GMC), definido originalmente en \cite{lange:popl15}. GMC se define a partir de una serie de condiciones que debe cumplir el conjunto de CFSMs participantes del sistema. Esto condiciona al sistema (o a quienes lo diseñen) a tener un conocimiento previo de los posibles sistemas que vayan a interactuar entre sí. Esto no necesariamente rompe con la idea del descubrimiento y binding en tiempo de ejecución dado que se pueden generar métodos para hacer estas comprobaciones a medida que el sistema va conectándose con los distintos participantes. %(existen ya?)%

\begin{definition}[Communicating Finite State Machines] Sea $\mathcal{M}$ un conjunto finito de mensajes y $\mathcal{P}$ un conjunto finito de participantes, definimos una CFSM sobre $\mathcal{M}$ como un sistema de transición finito $\langle Q, C, q_0, \mathcal{M}, \delta \rangle$ donde
\begin{itemize}
  \item $Q$ es un conjunto finito de estados;
  \item $C = \left\{ pq \in \mathcal{P}^2 \left|\right. p \not= q\right\}$ es un conjunto de canales
  \item $q_0 \in Q$ es el estado inicial;
  \item $\delta \subseteq Q \times (C \times \{!,?\} \times \mathcal{M}) \times Q$ es un conjunto finito de \emph{transiciones}.
  \end{itemize}
\end{definition} 

A partir de la introducción de las CFSMs, la siguiente definición introduce el concepto de \emph{CS} como un conjunto de CFSMs que cumplen determinadas propiedades.

\begin{definition}[Communicating System, \cite{lange:popl15}, Def.~2.2]\label{def:CS} Dado un conjunto de mensajes $\mathcal{M}$, una CFSM $\textit{M}_p = \langle Q_p, \mathcal{C}_p, q_{0_p}, \delta_p \rangle$ para cada participante $p \in \mathcal{P}$, la tupla $S=\langle M_p \rangle_{p \in \mathcal{P}}$ es un communicating system (CS).
\end{definition}

La semántica de los CSs está dada por un sistema de transición etiquetado cuyos estados y transiciones determinan las posibles ejecuciones del conjunto de procesos sobre el que está definido el sistema.

\begin{definition}[Estados y configuraciones alcanzables]
\label{def:estadosyconf} Dado un conjunto de mensajes $\mathcal{M}$, una CFSM $\textit{M}_p = \langle Q_p, \mathcal{C}_p, q_{0_p}, \delta_p \rangle$ para cada participante $p \in \mathcal{P}$ y $S=\langle M_p \rangle_{p \in \mathcal{P}}$ un CS, el sistema de transición etiquetado que determinala semántica de $S$ (denotado como $M_S = \langle Q_S, 
{q_0}_S, \delta_S \rangle$ una configuración de S es un par $s = \langle \overrightarrow{q} ; \overrightarrow{\omega} \rangle$ donde $\overrightarrow{q} = (q_p)_{p \in \mathcal{P}}$ con $q_p \in Q_p$ y donde $\overrightarrow{\omega} = (\omega_{pq})_{pq \in C}$ con $\omega_{pq} \in \mathcal{M}^*$. La componente $\overrightarrow{q}$ es el estado de control y $q_p \in Q_p$ es el estado local de la máquina $ M_p$. La configuración inicial de S es ${q_0}_S = \langle \overrightarrow{q_0} ; \overrightarrow{\epsilon} \rangle$ con $\overrightarrow{q_0} = (q_{0_p})_{p \in \mathcal{P}}$ y $\overrightarrow{\epsilon} = (\epsilon_p)_{p \in \mathcal{P}}$.

Dadas $c' = \langle \overrightarrow{q'},\overrightarrow{\omega'} \rangle$, $c = \langle \overrightarrow{q},\overrightarrow{\omega} \rangle$ y $l \in \mathcal{C}_s \times \{!,?\} \times \mathcal{M}$, decimos que $c'$ es alcanzable desde $c$ a través de la etiqueta $l$, denotado como $\langle c, l, c' \rangle \in \delta_S$, si y sólo si:
    \begin{enumerate}
		\item $l=sr!m$ y $\langle q_s,l,q'_s\rangle \in \delta_s$ y 
			\begin{enumerate}
				\item $q'_p = q_p$ para todo $\p \neq s$; y
				\item $\omega'_{sr} = \omega_{sr} \cdot m$ y  $\omega'_{pq} = \omega_{pq}$ para todo $pq \neq sr$; o bien
			\end{enumerate}
		\item $l=sr?m$ y $\langle q_r,l,q'_r\rangle \in \delta_r$ y 
			\begin{enumerate}
			\item $q'_{p} = q_{p}$ para todo $p \neq r$; y
				\item $\omega_{sr} = m \cdot \omega'_{pq}$ y $\omega'_{pq} = \omega_{pq}$ para todo $pq' \neq sr$
			\end{enumerate}
	\end{enumerate}
El conjunto de configuraciones alcanzables de S es $RS(S) = \{q \in Q_S | \langle {q_0}_S,l_0 \ldots l_n,q \rangle \in \delta_S^* \}$ donde $\delta_S^*$ es la clausura reflexo-transitiva de $\delta_S$
\end{definition}

En adelante presentaremos definiciones y resultados que permiten expresar propiedades de los CSs introducidos en la definición anterior.

\begin{definition}[Deadlock]Sea $S$ un CS, una transición $t$ del mismo y $s= \langle \overrightarrow{q} ; \overrightarrow{\omega} \rangle$ con $\overrightarrow{q}= \langle q_1, \ldots, q_n \rangle$ y sea $\overrightarrow{\omega}= \langle \omega_1, \ldots, \omega_n \rangle$ una de sus configuraciones. Decimos que $s$ es una \textit{configuración de deadlock} si $\overrightarrow{\omega} = \overrightarrow{\epsilon}$, existe $r \in \mathcal{P}$ tal que $qr$ es un estado receptor y $\langle q_r,sr?a,q'_r \rangle \in \delta_r$ y para todo $p \in \mathcal{P}$, $q_p$ es un estado receptor o final. Es decir todos los canales están vacíos, hay al menos una máquina esperando un mensaje y todas las otras máquinas están en un estado final o receptor.
\end{definition}

\begin{definition}[Recepción no especificada]Sea $S$ un CS, una transición $t$ del mismo y $s= \langle \overrightarrow{q} ; \overrightarrow{\omega} \rangle$ con $\overrightarrow{q}= \langle q_1, \ldots, q_n \rangle$ y sea $\overrightarrow{\omega}= \langle \omega_1, \ldots, \omega_n \rangle$ una de sus configuraciones. Decimos que $s$ es una \textit{configuración de recepción no especificada} si existe $r \in \mathcal{P}$ tal que $qr$ es un estado receptor y $\langle q_r,sr?a,q'_r \rangle \in \delta_r$ implica que $|\omega_sr| > 0$ y $\omega_sr \notin a\mathcal{M}*$. Una configuración de recepción no especificada corresponde a una configuración en la que la máquina está bloqueada definitivamente por su incapacidad para recibir el mensaje que se encuentra en sus canales.
\end{definition}

\begin{definition}[Mensaje Huérfano] Sea $S$ un CS, una transición $t$ del mismo y $s= \langle \overrightarrow{q} ; \overrightarrow{\omega} \rangle$ con $\overrightarrow{q}= \langle q_1, \ldots, q_n \rangle$ y sea $\overrightarrow{\omega}= \langle \omega_1, \ldots, \omega_n \rangle$ una de sus configuraciones. Decimos que $s$ es una \textit{configuración de mensaje huérfano} si todos los $q_p \in \overrightarrow{q}$ son finales pero $\overrightarrow{\omega} \neq \overrightarrow{\epsilon}$. Es decir quedó un mensaje en algún canal pero no hay ninguna transición que lo reciba.
\end{definition}

\begin{definition}[CS seguro] Sea $S$ un CS; se dice que $S$ es \emph{seguro} si para cada $s \in RS(S)$:
\begin{enumerate}
\item $s$ no es una configuración de deadlock 
\item $s$ no posee recepciones no especificadas, y
\item $s$ no posee mensajes huérfanos  
\end{enumerate} 

Para poder expresar esta condición de seguridad es necesario identificar conjuntos de acciones que pueden ser llevadas a cabo concurrentemente. Para esto definimos las siguientes relaciones sobre el conjunto de transiciones de una CFSM. Dados $q, q' \in Q$, se define $\mathit{act}(q,q') = \left\{\ell \ \left|\right. \ (q,\ell,q') \in \delta \right\}$ y $\Diamond, \blacklozenge \subseteq \delta \times \delta$ como las relaciones de equivalencia más pequeñas que contienen $\romboeqb$ y $\underline{\blacklozenge}$ donde:
\begin{itemize}
\item $(q_1, \ell, q_2) \underline{\Diamond} (q'_1, \ell, q'_2)$ sii $ l \notin \mathit{act}(q_1, q'_1) \land \mathit{act}(q_1, q'_1) = \mathit{act}(q_2, q'_2) \land \mathit{act}(q_2, q'_2) \neq \emptyset $
\item  $(q_1, \ell, q_2) \underline{\blacklozenge} (q'_1, \ell, q'_2)$ sii $ (q_1, \ell, q_2) \romboeqb (q'_1, \ell, q'_2) $ y $\forall(q,\ell,q') \in [(q_1, \ell, q_2) ]^{\Diamond}, \ \mathit{act}(q_1,q) = \mathit{act}(q_2,q') \land \mathit{act}(q'_1,q) = \mathit{act}(q'_2,q')$  
\end{itemize}
donde $[(q_1, \ell, q_2) ]^{\Diamond}$ es la clase de equivalencia de $(q, \ell, q')$ respecto de la relación $\Diamond$ (resp. $\blacklozenge$). Intuitivamente dos transiciones están $\blacklozenge$-relacionadas si se refieren a la misma acción aún teniendo en cuenta el interleaving.
\end{definition}

\begin{example}[Relaciones $\underline{\Diamond}$ y $\underline{\blacklozenge}$]
\label{ex:relaciones}
Consideremos la siguiente CFSM:
\begin{center}
\begin{tikzpicture}[->, thick]
 \node[state,initial] (q_0)   {$q_0$}; 
 \node[state] (q_1) [right= of q_0 ] {$q_1$};
 \node[state] (q_5) [right= of q_1 ] {$q_5$};
 \node[state] (q_2) [below= of q_0 ] {$q_2$};
 \node[state] (q_3) [right= of q_2 ] {$q_3$};
 \node[state,accepting] (q_6) [right= of q_3 ] {$q_6$};
 \draw[]        
        (q_0) edge[above] node{sr!a} (q_1)
        (q_0) edge[right] node{sr'!b} (q_2)
        (q_1) edge[right] node{sr'!b} (q_3)
        (q_1) edge[above] node{sr!a} (q_5)
        (q_2) edge[above] node{sr!a} (q_3)
        (q_2) edge[bend right, below] node{sr!c} (q_6)
        (q_3) edge[above] node{sr!a} (q_6)
        (q_5) edge[right] node{sr'!b} (q_6)
        ;
\end{tikzpicture} 
\end{center}

\begin{enumerate}
    \item \label{ex:cond1} $(q_0, sr!a,q1) \underline{\Diamond} (q_2,sr!a,q3)  $ %dado que está entrelazada con $sr'!b$
    \item \label{ex:cond2} $(q_0, sr!a,q1) \underline{\blacklozenge} (q_2,sr!a,q3) $ %por la misma razón que 1
    \item \label{ex:cond3} No vale $ ((q_0, sr!a,q1) \underline{\Diamond} (q_1,sr'!b,q5))$ %porque la transición entre $q_0$ y $q_1$ pasa a través de $sr!a$. Es decir que las dos transiciones son secuenciales no concurrentes. 
    \item \label{ex:cond4} $ (q_0, sr'!b,q2) \underline{\Diamond} (q_1, sr'!b,q3)$ %la relación e 
    \item \label{ex:cond5} No vale $((q_0, sr'!b,q2) \underline{\blacklozenge} (q_1,sr'!b,q_3)) $
\end{enumerate}

Las relaciones en Ej.~\ref{ex:relaciones}.\ref{ex:cond1}--~\ref{ex:relaciones}.\ref{ex:cond2} se sostienen dado que ambas transiciones están entrelazadas con $sr'!b$. La relacion en Ej.~\ref{ex:relaciones}.\ref{ex:cond3} no se sostiene debido a que la transición entre el origen de una $(q_0)$ y el del otro $(q_1)$ pasa por $sr!a$. Ambas transiciones en Ej.~\ref{ex:relaciones}.\ref{ex:cond3} son secuenciales, no concurrentes. La relación en Ej.~\ref{ex:relaciones}.\ref{ex:cond4} se sostiene, pero en Ej.~\ref{ex:relaciones}.\ref{ex:cond5} no porque $(q_5,sr'!b,q_6)$ está en la clase de $\Diamond$-equivalencia de $(q_0,sr'!b,q_2)$ para la cual la condición no se sostiene (debido a la transición con la etiqueta $sr!c$).
\end{example}

\begin{definition}[Eventos] Dados un conjunto de participantes $\mathcal{P}$ y un conjunto de mensajes $\mathcal{M}$ definimos un evento $e$ como es una tupla $\langle q_s, q_r, s, r, a \rangle)$ (también escrita como $\langle q_s, q_r, s \rightarrow r:a \rangle$), tal que $s,r \in \mathcal{P}$, indicando que $s$ y $r$ pueden intercambiar el mensaje a, cuando están en el estado $q_s$ y $q_r$, respectivamente.
\end{definition}

Tomando en consideración la definición anterior, para distinguir el paralelsimo a nivel máquina introducimos una relación de equivalencia sobre eventos que identifica eventos cuyas transiciones son $\blacklozenge$-equivalentes.

\begin{definition}[Equivalencia entre eventos] Definimos la equivalencia entre eventos como $\bowtie = \bowtie_s \cap \bowtie_r \subseteq \mathcal{E} \times \mathcal{E}$ donde $\mathcal{E}$ es el conjunto de eventos del sistema y se cumplen las siguientes condiciones:

\begin{itemize}
\item $(q_1, q_2, s \rightarrow r:a) \bowtie_s (q'_1, q'_2, s \rightarrow r:a) \iff$ \\ 
 $\forall (q_1, sr!a,q_3),(q'_1, sr!a,q'_3) \in \delta_s:(q_1, sr!a,q_3) \underline{\blacklozenge} (q'_1, SR!a,q'_3) $
\item $(q_1, q_2, s \rightarrow r:a) \bowtie_r (q'_1, q'_2, s \rightarrow r:a) \iff \\ 
\forall (q_2, sr?a,q_3),(q'_2, sr?a,q'_4) \in \delta_ r:(q_2, sr?a,q_4) \romboeqn (q'_2, sr?a,q'_4)$ \end{itemize}
\end{definition}
\begin{example}[Equivalencia entre eventos]
\label{ex:equiveventos}
Considere el siguiente CS

\begin{tikzpicture}[->, thick]
 \node[state,initial] (q_0)   {$q_0$}; 
 \node[state] (q_1) [right= of q_0 ] {$q_1$};
  \node[state] (q_2) [below= of q_0 ] {$q_2$};
 \node[state] (q_3) [right= of q_2 ] {$q_3$};

 \draw[]        
        (q_0) edge[above] node{pr!a} (q_1)
        (q_0) edge[right] node{sp?b} (q_2)
        (q_1) edge[right] node{sp?b} (q_3)
        (q_2) edge[above] node{pr!a} (q_3)
        ; 
\end{tikzpicture} 
\qquad
\begin{tikzpicture}[->, thick]
 \node[state,initial] (q_0)   {$q_0$}; 
 \node[state] (q_1) [below= of q_0 ] {$q_1$};

 \draw[]        
        
        (q_0) edge[right] node{pr?a} (q_1)
        
        ;
\end{tikzpicture} 
\qquad
\begin{tikzpicture}[->, thick]
 \node[state,initial] (q_0)   {$q_0$}; 
 \node[state] (q_1) [below= of q_0 ] {$q_1$};

 \draw[]        
        
        (q_0) edge[right] node{sp!b} (q_1)
        
        ;
\end{tikzpicture} 

En el sistema de arriba podemos ver los siguientes eventos $(q_{0p}, q_{0r}, p \rightarrow r:a)$, $(q_{2p}, q_{0r}, p \rightarrow r:a)$, $(q_{0s}, q_{0p}, s \rightarrow p:b)$ y $(q_{0s}, q_{3p}, s \rightarrow p:b)$. Queremos ver si se cumple que $(q_{0p}, q_{0r}, p \rightarrow r:a) \bowtie (q_{2p}, q_{0r}, p \rightarrow r:a)$, es decir que son eqivalentes tanto bajo $\bowtie_p$ como en $\bowtie_r$. 

Para el primero queremos ver que $(q_{0p}, pr!a, q_{1p}) \romboeqn (q_{2p}, pr!a, q_{3p})$
Para esto necesitamos que valga $(q_{0p}, pr!a, q_{1p}) \romboeqb (q_{2p}, pr!a, q_{3p})$. Esto se cumple dado que $pr!a \notin act(q_{0p},q_{2p})$ y $act(q_{0p},q_{2p}) = act(q_{1p},q_{3p}) \neq \emptyset$. Ahora tenemos que ver la clase de equivalencia $[(q_{0p}, pr!a, q_{1p})]^{\Diamond}$. La misma es el conjunto unitario $ \{(q_{2p}, pr!a, q_{3p})\}$. Por último queremos ver que  $act(q_{0p}, q_{2p}) = 
act(q_{1p}, q_{3p}) \land act(q_{2p}, q_{2p}) = act(q_{3p}, q_{3p})$. La segunda es trivial, la primera se cumple siendo el conjunto unitario $\{sp?b\}$. Con esto demostramos que $(q_{0p}, q_{0r}, p \rightarrow r:a) \bowtie_p (q_{2p}, q_{0r}, p \rightarrow r:a)$ 

Nos queda probar que vale $(q_{0p}, q_{0r}, p \rightarrow r:a) \bowtie_r (q_{2p}, q_{0r}, p \rightarrow r:a)$. Esta equivalencia es más sencilla de probar, dado que tenemos una única transición. Entonces vemos que es trivial que $(q_{0r}, pr?a, q_{1r}) \underline{\blacklozenge} (q_{0r}, pr?a, q_{1r})$. Con esto vemos que $(q_{0p}, q_{0r}, p \rightarrow r:a)$ y $(q_{2p}, q_{0r}, p \rightarrow r:a)$ son equivalentes en $\bowtie_p$ y $\bowtie_r$, por lo tanto están en $\bowtie = \bowtie_p \cap \bowtie_r$.
\end{example} 

A continuación definimos la noción de Sistema de Transición Sincrónico, para reflejar el comportamiento de un sistema cuando envíos y recepciones son emparejados para mostrar que ocurren al mismo tiempo.

\begin{definition}[Sistema de transición sincrónico] Dados un $S = (M_P)_{p \in \mathcal{P}}$ un CS, sea $\langle N,\hat{\delta}, E \rangle$, donde: 
\begin{itemize}
    \item[] $N = \{\overrightarrow{q} \ | \ (\overrightarrow{q}; \overrightarrow{\epsilon}) \in RS_1(S) \}$,
    \item[] $\hat{\delta}= \{(n, e, n') \ | \ (n;\overrightarrow{\epsilon}) s_1 \overset{sr!a}{\longrightarrow}\overset{sr?a}{\longrightarrow} (n';\overrightarrow{\epsilon})	\land e= n[s], n[r], s \rightarrow r:a \}$, y
    \item[] $ E = \{ \exists n, n' \in N : (n,e,n') \in \hat{\delta}\} \subseteq \mathcal{E}$,
\end{itemize}
   el \emph{Sistema de Transición Sincrónico} de S es $TS(S)= \langle N, n_0, E/ \bowtie,\rightrightharpoons \rangle$ donde $n_0= \overrightarrow{q_0} $ es el estado inicial, $n \overset{[e]}{\rightrightharpoons} n' \iff (n,e,n') \in \hat{\delta}$. Fijamos un conjunto $\hat{E}$ de elementos representativos de cada clase de equivalencia $\bowtie$ (ej: $\hat{E} \subseteq E$ y $\left(\forall e \in E\right)\left(\exists!e' \in \hat{E}\right)\left(e' \in [e] \right)$) y escribimos $n \overset{e'}{\rightrightharpoons} n'$ para $ n \overset{[e]}{\rightrightharpoons} n'$ cuando $ e' \in [e] \cap \hat{E} $. Las secuencias de eventos se notan con un símbolo $\pi$ y extendemos la notación de $ \rightarrow$ en la Def.~\ref{def:estadosyconf} a $\rightrightharpoons$ (ej: $si \pi = e_1 ...e_k, n_1 \overset{\pi}{\rightrightharpoons}n_{k+1} sii n_1 \overset{e_1}{\rightrightharpoons} n_2 \overset{e_2}{\rightrightharpoons}...\overset{e_k}{\rightrightharpoons} n_{k+1}$).

$TS(S)$ representa todas las posibles ejecuciones sincrónicas del sistema $S$; y cada transición es etiquetada con un evento $e$.
\end{definition}

\begin{example}[Sistema de Transición Sincrónico] 
\label{ex:STS}
Consideremos el Ej.~\ref{ex:equiveventos}, su Sistema de Transición Sincrónico es el siguiente

\begin{tikzpicture}[->, very thick]
\node[state] (q_0) {};
\node[state] (q_1) [right= 3.5cm of q_0 ] {};
\node[state] (q_2) [below= of q_0 ] {};
\node[state] (q_3) [right= 3.5cm of q_2 ] {};
%$(q_{0p}, q_{0r}, p \rightarrow r:a)$, $(q_{2p}, q_{0r}, p \rightarrow r:a)$, $(q_{0s}, q_{0p}, s \rightarrow p:)$ y $(q_{0s}, q_{3p}, s \rightarrow p:b)$
\draw[]        
        (q_0) edge[above] node{$(q_{0p}, q_{0r}, p \rightarrow r:a)$} (q_1)++
        (q_0) edge[left] node{$(q_{0s}, q_{0p}, s \rightarrow p:b)$} (q_2)
        (q_1) edge[right] node{$(q_{0s}, q_{3p}, s \rightarrow p:b)$} (q_3)
        (q_2) edge[below] node{$(q_{2p}, q_{0r}, p \rightarrow r:a)$} (q_3)
        ; 
\end{tikzpicture}

Tenemos $(q_{0p}, q_{0r}, p \rightarrow r:a) \bowtie (q_{2p}, q_{0r}, p \rightarrow r:a)$ y $(q_{0s}, q_{0p}, s \rightarrow p:b) \bowtie (q_{0s}, q_{3p}, s \rightarrow p:b)$. Podemos considerar equivalentes los eventos de las transiciones verticales por un lado y las de las transiciones horizontales por el otro. Esto nos permite identificar un par de interacciones concurrentes, pero seguir diferenciándolas de otras instancias de comunicación $p \rightarrow r:a$ y $s \rightarrow p:b$
\end{example}

\begin{definition}[Proyecciones] La proyección de un evento e sobre un participante p, denotado por $e \downharpoonright_p$ se define de la siguiente manera:
\begin{equation}
(q_s,q_r,s \rightarrow r:a) \downharpoonright_p = \begin{cases} 
pr!a & \mathit{if} s=p \\
sp?a & \mathit{if} r=p \\
\epsilon & \mathit{en \ otro \ caso} \\
\end{cases} 
\end{equation}

La proyección se define sobre secuencias de eventos en el modo evidente. La proyección $TS(S)= (N, n_0, \hat{E}, \rightrightharpoons)$ sobre el participante p, notada $ TS(S) \downharpoonright_p $, es el autómata $(Q, q_0, \Sigma, \delta)$ donde $Q=N, \ q_0 = n_0$, $\Sigma$ es el conjunto de etiquetas y $\delta \subseteq Q \times \Sigma \cup \{ \epsilon \} \times Q $ es el conjunto de transiciones, tal que $(n_1, e \downharpoonright_p, n_2) \in \delta \iff n_1 \overset{e}{\rightrightharpoons} n_2 $
\end{definition}

Introducimos el concepto de generalized multiparty compatibility (GMC) como una condición completa y sólida para construir CFSMs. 
% Usé sólida como traducción de sound. En el paper habla de construir global graphs, término que acá no usamos, no se si puse bien
A partir de este punto, fijamos un sistema $S =(M_p)_{p \in \mathcal{P}} $ con $TS(S)= (N, n_0, \hat{E}, \rightrightharpoons)$. GMC depende de dos condiciones, representabilidad y propiedad de ramificación.\\

La propiedad de representabilidad determina que cada máquina, traza y elección estén representadas en el sistema de transición.

\begin{definition}
Para un lenguaje $\mathcal{L}$, $hd(\mathcal{L})$ devuelve las primeras acciones de $\mathcal{L}$ si las tiene: 
\begin{center}
$hd(\mathcal{L})= \{\ell \ | \ \exists q \in Act*: \ell \cdot q \in \mathcal{L} \}$	\ \	 $hd(\{ \epsilon \}) = \{ \epsilon \}$
\end{center}
Dado $n \in N$, sea $ TS(S)\langle n \rangle$ el sistema de transición TS(S) donde se reemplaza al estado inicial $n_0$ por $n$. Escribimos LT(S,n,p) para $\mathcal{L}(TS(S)\langle n \rangle)\downharpoonright_p $, es decir LT(S,n,p) es el lenguaje que se obtiene estableciendo a n como nodo inicial de TS(S) y proyectando el nuevo sistema de transición sobre p.
\end{definition}

\begin{definition}[Representabilidad]
\label{def:representabilidad}
Un sistema S es representable si
\begin{enumerate}
\item $\mathcal{L}(M_p) = LT(S,n_0,p) $ y
\item $\forall q \in Q_p \ \exists n \in N: n[p] = q \land  \cup_{(q,\ell, q') \in \delta_p} \{ \ell \} \subseteq hd (LT(S,n,p))$
\end{enumerate}
para todo $p \in \mathcal{P} $ \\

La primera condición asegura que cada traza de cada máquina esté en $TS(S)$, a su vez, la segunda condición es necesaria para asegurar que cada elección en cada máquina esté representada en $TS(S)$.
\end{definition}

La propiedad de ramificación estimula que cada vez que hay una decisión en un sistema $TS(S)$, una única máquina toma esa decisión y cada uno de los otros participantes es notificado de la rama que se tomó o no participa en esa elección. 
 
\begin{definition}[Propiedad de ramificación] Un sistema S posee la propiedad de ramificación si para todo $n \in N$ y para todo $e_1 \neq e_2 \in \hat{E} \ \mathit{tq} \ n \overset{e_1}{\rightrightharpoons} n_1 \ \mathit{y} \  \ n_1 \overset{e_2}{\rightrightharpoons} n_2$ luego tenemos
\begin{enumerate}
\item o bien existe $n'\in N$ tal que $n_1 \overset{e_2}{\rightrightharpoons} n' \ \mathit{y} \ n_2 \overset{e_1}{\rightrightharpoons} n'$, o 
\item para cada $(n'_1, n'_2) \in ln(n, e_1, e_2)$ quedando 

$L_p^i = hd (\{e_i \downharpoonright_p \cdot \phi \  | \ \phi \in LT(S,n'_i,p) \} ) $ con $i \in \{1,2\}$ y $p \in P$, y se cumplen las condiciones 2a, 2b y 2c definidas abajo 	

\begin{enumerate}[(a)]
\item Choice awareness: $\forall p \in \mathcal{P}$ valen \begin{enumerate}[i.]
\item $L_p^1 \cap	L_p^2 \subseteq \{ \epsilon \} \ \mathit{y} \  \epsilon \in L_p^1 \iff \epsilon \in L_p^2$, o 
\item $\exists n' \in N, \pi_1, \pi_2$: $n'_1 \overset{\pi_1}{\rightrightharpoons} n' \land  n'_2 \overset{\pi_2}{\rightrightharpoons} n' \land (e_1 \cdot \pi_1) \downharpoonright_p= (e_2 \cdot \pi_2)\downharpoonright_p = \epsilon$
\end{enumerate}
\item selector único: $\exists!s \in \mathcal{P}: L_s^1 \cap L_s^2 = \emptyset \land \exists sr!a \in L_s^1 \cup L_s^2 $
\item no race: $\forall r \in \mathcal{P}: L_r^1\cap L_r^2 = \emptyset \Rightarrow \forall s_1r?a_1 \in L_r^1, \forall s_2r?a_1 \in L_r^2:\forall i \neq j \in \{1,2\}: n'_i \overset{\pi_i}\rightrightharpoons  $ \\
$\Rightarrow dep (s_i \rightarrow r: a_i, e_i \cdot \pi_i, s_j \rightarrow r: a_j) $
\end{enumerate}
\end{enumerate} 	
\end{definition}

Representabilidad garantiza que TS(S) contiene suficiente información para decdir propiedades seguras de cualquier ejecución asincrónica de S. La propiedad de ramificación asegura que si una rama en TS(S) representa una elección esta está "bien formada".\\
 
Con esto definimos que toda ramificación es o bien (1) la ejecución concurrente de dos eventos; o, para cada participante p (2(a)I) si p no termina antes de n entonces las primeras dos acciones de p en dos ramas distintas son disjuntas; o (2(a)II) p no está involucrado en la elección, o sea la ramas se juntan antes de que p realice ninguna acción; (2(b)) hay un único participante s tomando la decisión; y (2(c)) para cada participante r involucrado en la elección, no puede haber race condition entre los mensajes que puede recibir r. La no race condition asegura que en ninguna ejecución (asincrónica) de S si una máquina tiene más de un buffer no vacío, entonces puede leerlos en cualquier orden (interleaving es posible). Notar que si una máquina r recibe todos sus mensajes de un mismo emisor, entonces hay una $\triangleleft$-relación entre todas sus acciones.

\begin{definition}[Generalised Multiparty Compatibilty] Un CS S es generalised multiparty compatible (GMC) si es representable y posee la propiedad de ramifiación. 
\end{definition}

\begin{theorem}[Solvencia]Si S es GMC entonces es seguro (no tiene configuraciones de deadlock, recepción no especificada o mensaje huérfano)
\end{theorem}

El teorema dice que ninguna ejecución asincrónica de S va a resultar en una configuración de mensaje huérfano, deadlock o recepción no especificada. Basándose en la propiedad de representabilidad (toda transición y ramificación de cada máquina está representada en TS(S)), la demostración \cite{lange:popl15} muestra que todo mensaje enviado es recibido eventualmente y que una máquina en un estado receptor eventualmente recibe el mensaje que espera, según la Def.~\ref{def:representabilidad}.

