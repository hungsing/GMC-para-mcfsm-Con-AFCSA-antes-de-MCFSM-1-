%!TEX root = ./main.tex
\newpage
\chapter{Conclusiones y Trabajo futuro} 
\label{conclusiones}

En esta sección resumimos las contribuciones de esta tesis junto con conclusiones y posibles futuras líneas de investigación.


\section{Conclusiones} 

El objetivo de esta tesis era el estudio y posible extensión de los modelos formales para servicios, dado que consideramos que son necesarios para la implementación de SOC en toda su capacidad. Partiendo del ideal del paradigma SOC, es decir, un contexto que permita la reconfiguración dinámica de un artefacto de software a partir de posibilitar el binding en tiempo de ejecución, el objetivo principal fue crear un formalismo que soporte binding parcial. Con este objetivo en mente desarrollamos los AFCA que permiten internalizar el envío y la recepción de mensajes como comportamiento interno de una componente resultante de la composición. Dicho mecanismo de composición, además, preserva la comunicación externa de las componentes involucradas.

Para expresar la interfaz externa de comunicación de ese tipo de autómatas propusimos las mCFSMs, una extensión de las CFSMs con múltiples canales entre cada par de participantes. esto facilita la unificación de las interacciones de dos componente a ser compuestas, con una tercera. Como método para garantizar que los distintos participantes puedan interoperar en forma segura extendimos la noción de GMC para mCFSMs. Por último pudimos demostrar la equivalencia entre el comportamiento de la composición de un conjunto de AFCA y el de sus respectivas mCFSMs.\\
Todos estos elementos proporcionan los fundamentos de un lenguaje formal que provee un mecanismo de composición parcial, en el sentido de lo explicado en la Sec.~\ref{afca-compo-parcial}, y los vincula con resultados conocidos del campo de los sistemas distribuidos que permiten garantizar una interoperación libre de errores.


\section{Trabajo futuro}
Si bien el lenguaje soporta binding parcial, queda por estudiar en mayor profundidad si el lenguaje cumple con las condiciones esperadas. Para esto es necesario definir formalmente la composición parcial de AFCAs y ver la relación entre un AFCA obtenido por una serie de composiciones parciales y uno de una composición total. Esto permite vincular los crterios de corrección de la comunicación establecidos por el lenguaje formal de las CFSM con el comportamiento del resultado de la composición dinámica del sistema. \\
Una condición que se pide en este modelo es que el conjunto de autómatas y el autómata compuesto sean weak determinate. Esta condición es muy fuerte y restrictiva dado que la propiedad de weak determinacy no se preserva en la composición. De esta condición surge la pregunta de si es posible lingüísticamente generar autómatas weak determinate que al componerse den uno que también lo sea. Una forma posible de solucionar esto sería un modelo donde el comportamiento interno no esté representado en las aristas del autómata, sino de alguna manera encapsularlo dentro de los estados. De este modo potencialmente se evitaría el problema de que la composición de autómatas weak determinate pueda genera uno que no lo es.


