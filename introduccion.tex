%!TEX root = ./main.tex
\chapter{Introducción}
La tendencia hacia los sistemas distribuidos genera la necesidad de mecanismos de comunicación más complejos. Para manejar esta complejidad se han introducido lenguajes de especificación y métodos formales de análisis que permiten asegurar ciertas propiedades de dichos mecanismos como por ejemplo, las \emph{Communicating Finite State Machines} \cite{brand:jacm-30_2}, los \emph{Global Graphs} \cite{castagna:lmcs-8_1}, los \emph{Session types} en su gran diversidad de variantes \cite{honda:esop98,honda:popl08} y los \emph{Interface automata} \cite{dealfaro:esec-fse-01}. Tomando esto en consideración, los autómatas finitos resultan útiles como lenguaje primitivo de modelado, subyacente en muchos de los formalismos mencionados anteriormente, que permite representar algunos de estos aspectos de dichos sistemas.

Service oriented computing (SOC) es un paradigma de computación distribuida que cambió el modo en que los sistemas de software son concebidos. El corazón del paradigma son servicios que proveen elementos computacionales autónomos, independientes de la plataforma y que se ejecutan sobre una infraestructura de cómputo y comunicación existente. Estos pueden ser descritos, publicados, descubiertos y programados usando protocolos estándar para construir redes de aplicaciones que colaboran entre sí, incluso distribuidas dentro de distintas fronteras organizacionales, con la misión de, colectivamente, alcanzar un objetivo de negocios. Uno de los elementos centrales de este paradigma es que dichos elementos computacionales son procurados en tiempo de ejecución y bajo demanda; una demanda, resulta local a cada ejecución, lo que implica que no todos los servicios necesarios son al mismo tiempo y algunos, puede que ni siquiera lo sean.

Esta mirada sobre cómo un sistema de software evoluciona reconfigurándose en tiempo de ejecución pone de relieve la necesidad de contar con un lenguaje de descripción que posibilite un enfoque para la composición que sea parcial, y que el resultado de dicha composición resulte una descripción legítima de una componente; una característica que la gran mayoría de los lenguaje utilizados en la descripción de sistemas distribuidos no posee. Volveremos sobre esto en la Sec.~\ref{trabajo-relacionado} donde discutiremos otros lenguajes formales relacionados, que hemos mencionado más arriba, al final de la presente.\\

En este trabajo buscaremos definir un lenguaje formal de modelado de componentes de software con el objeto de satisfacer las necesidades mencionadas anteriormente. Este lenguaje debe satisfacer las siguientes propiedades:
\begin{inparaenum}[1.]
\item debe posibilitar la convivencia de elementos computacionales internos de una componente (transiciones que expresan cambios locales de estado) con su interfaz de comunicación (transiciones que expresan envío o recepción de mensajes),
\item debe poseer un mecanismo claro de composición que no requiera que todos los participantes, y
\item debe tener semántica de comunicación asincrónica, compatible con la semántica de los lenguajes formales conocidos como los mencionados anteriormente).
\end{inparaenum}

Para satisfacer estos objetivos definiremos una clase de autómatas finitos, a la que llamaremos \emph{Autómatas Finitos de Comunicación Asincrónica} (AFCA), que cuentan con transiciones internas (denotando cambios locales de estado) y transiciones de comunicación sobre canales de comunicación (que expresan la comunicación con otras componentes), adicionalmente, estos autómatas pueden ser compuestos internalizando la comunicación a través de la creación de buffers dedicados que permiten reemplazar los canales de comunicación. Adaptaremos el lenguaje de las Communicating Finite State Machines, a las llamaremos \emph{multichannel Communicating Finite State Machines} (mCFSM), con el objeto de que sean capaces de reflejar la interfaz de comunicación de estos autómatas. Por último, probaremos la equivalencia entre la semántica de la composición de una familia de AFCAs y la del \emph{communicating system} obtenido a partir de la familia de CFSM correspondientes a cada uno de dichos AFCAs (ver Fig.~\ref{fig:equivalencia}).\\

En la siguiente sección presentamos brevemente distintos trabajos relacionados y como esos modelos no cumplen del todo con lo que queríamos modelar. El resto de esta tesis se divide en cuatro capítulos. Primero en el capítulo~\ref{preliminares} describimos las CFSM como lenguaje y la noción de Generalized Multiparty Compatibility (GMC, \cite{lange:popl15}) como condición suficiente para que un conjunto de CFSMs formen un communicating system seguro. En el capítulo~\ref{AFCA} definimos los AFCA, su composición, y la proyección de su interfaz de comunicación. También definimos mCFSM como extensión de CFSM y proyección de la interfaz de comunicación de un AFCA, y adaptamos la propiedad de GMC para este nuevo modelo. En el capítulo~\ref{resultados} exploramos la equivalencia de la semántica de un conjunto de AFCA y el communicating system resultante de su mCFSM proyectadas, y demostramos la validez de esta propiedad. Por último, en el capítulo~\ref{conclusiones} cerramos este trabajo con las conclusiones resultantes y algunas ideas para trabajo a futuro.


\section*{Trabajo relacionado}
\label{trabajo-relacionado}
\cnote{Repasar.}Como se ha dicho anteriormente, la propuesta de la presente tesis es la presentación de un lenguaje que permita describir una componente de un sistema distribuido, con las siguientes particularidades: 
\begin{itemize}
\item debe posibilitar la convivencia de acciones vinculadas al cómputo que realiza localmente la componente, junto con primitivas de comunicación utilizadas para interactuar con las restantes componentes del sistema,
\item se espera que posea un mecanismo de composición parcial, en relación al conjunto esperado de participantes, y
\end{itemize}

Existe una gran variedad de lenguajes que permiten la formalización de la comunicación entre componentes de un sistema distribuido. Algunos ejemplos de interés en este campo son:

\emph{Communicating Finite State Machines} \cite{brand:jacm-30_2}: Las CFSM son un modelo para protocolos de comunicación, basado en máquinas de estado finitas que representan procesos que se comunican entre vía el intercambio asincrónico de mensajes a través de canales FIFO. Las mCFSM son una versión extendida de dicho modelo con múltiples canales entre cada par de participantes y la interfaz de comunicación de los AFCA se proyecta como una mCFSM. A su vez los AFCA surgen de la necesidad de poder representar en un mismo formalismo todo el comportamiento de cada componente, tanto el interno como el de comunicación externa.

\emph{Session types} \cite{honda:esop98,honda:popl08}: Session Types es un cálculo tipado para procesos móbiles que introduce una nueva noción de tipos en la que las interacciones que incluyen múltiples participantes se abstraen directamente como un escenario global. Un global type cumple el rol de un acuerdo compartido entre pares que se comunican y es la base de type checking eficiente a través de su proyección sobre participantes individuales. Las propiedades fundamentales de la disciplina de session types como seguridad communication safety, progress y session fidelity son establecidas para interacciones asincrónicas de n participantes. Esta noción de acuerdos entre participantes formalizada a través de los global types permite modelar interacciones entre participantes de una comunicación y demostrar ciertas propiedades que garanticen una comunicación segura.

\emph{Global Graphs} \cite{castagna:lmcs-8_1}: Global Graphs son una superclase de los Generalized Global Types de \cite{denielou:esop12} que a su vez son una versión de los multi-party session global types. Esta versión de global types presenta un lenguaje simplificado equipado con semántica basada en trazas, cuyas características y restricciones están justificadas semánticamente. La noción de GMC originalmente surge en \cite{lange:popl15} como condición para que un communicating system compuesto de un conjunto de CFSMs fuese seguro para poder construir un Global Graph que modele la comunicación del sistema.  

En el caso de todo ellos, el objetivo único es la caracterización formal de la comunicación entre las componentes participantes de forma que se posible garantizar que esta se lleva a cabo en forma correcta.


\emph{Interface automata} \cite{dealfaro:esec-fse-01}: Es un lenguaje basado en autómatas que captura el comportamiento asumido respecto al orden en el que los métodos de un componente son llamados y el orden en el que se llama a métodos externos. Similar a las CFSMs esto permite modelar la interacción entre distintos componentes y validar el correcto comportamiento. A diferencia de las CFSMs estos autómatas modelan no solo comportamiento de comunicación (a través de acciones de input y output) sino también internas, y pueden componerse dadas ciertas condiciones de compatibilidad. La desventaja respecto a las CFSMs es que las interacciones entre estos autómatas son sincrónicas esto se ve en la composición de dos autómatas donde una acción de salida de un autómata que coincide con la de entrada de otro se transforman en una única acción.

\emph{Global Types for Open Systems} \cite{barbanera:eptcs279}: Los formalismos basados en Global Types permiten describir el comportamiento general de un sistmea distribuido y al mismo tiempo hacer cumplir propiedades de seguridad para la comunicación entre los sistemas componentes. La visión centralizada de los global type es adecuada para describir sistemas cerrados. Esto impide que un sistema descrito en base a global types se pueda ver como un módulo independiente que puede conectarse a otros sistemas. De la necesidad de resolver esta problemática surgen los global types abiertos. En este enfoque un global type abierto, denominado "global type with interface roles" (GTIR), denota un número de sistemas abiertos de CFSMs donde ciertos participantes (roles en este contexto) son identificados como interfaces en vez de como participantes propiamente dichos. Para lo cual se introduce una sintaxis paramétrica que, dado un formalismo basado en global types, extiende su sintaxis, permitiéndole identificar algunos roles como interface roles y definir una composición de global types abiertos, interpretados semánticamente como sistemas de CFSMs. Esta sintaxis no depende de un formalismo particular, sino en general mientras los componentes individuales o end points puedan ser interpretados como  CFSMs. Esta nueva sintaxis además permite asegurar la preservación de las condiciones de seguridad bajo la composición bajo ciertas condiciones.

Los GTIRs, al igual que otros formalismos anteriormente mencionados, se enfocan únicamente en el aspecto comunicacional, es decir en la interacción entre componentes. Similar a nuestro trabajo utiliza las CFSMs para representar la interfaz de comunicación de los end points y los sistemas de CFSMs o Communicating Systems (ver Def.~\ref{def:CS}) para los sistemas abiertos que los end points componen. En este sentido dejan de lado comportamiento que habría que modelar por separado. A su vez esta nueva sintaxis opera sobre CS y requiere condiciones más estrictas de sus CFSMs componentes para asegurar que se cumplan las propiedades de seguridad (ver Def.~\ref{def:safeCS}).

\emph{Choreography automata} \cite{---}

