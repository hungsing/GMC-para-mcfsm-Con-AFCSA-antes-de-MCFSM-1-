%!TEX root = ./main.tex
\chapter{Equivalencia de comportamiento comunicacional}
\label{resultados}
Esta sección va a estar enfocada en la definición de composición total de autómatas y la noción de que dado un conjunto $\{ \mathcal{A}_i\}_{i \in I}$ de AFCA weak determinate sin transiciones de buffer la semántica de la composición es equivalente a la del conjunto de mCFSM proyectadas de los elementos de $\{ \mathcal{A}_i\}_{i \in I}$. 

\begin{figure}[ht]
$$
\xymatrix{   
	\{\mathcal{A}_i\}_{i \in I} \ar[rrr]_{\Pi} \ar[d]_{||} & & & \{C_i\}_{i \in I}  \ar[d]_{semantica}  \\
	  {A} \ar[r]_{\Pi'} \ar[d]_{\Pi} & \mathcal{L}' & = & {\mathcal{L}}  \\
	  \emptyset
}
$$
\caption{La composición de componentes asincrónicas preserva la semántica asincrónica de su interfaz de comunicación.}
\label{fig:preservacion}
\end{figure}

En general, cuando se trata de formalismos que modelan simultáneamente aspectos del comportamiento interno de una componente y elementos de interfaz observable por otras componentes, la proyección de dicha interfaz no resulta bisimilar con el comportamiento real de la componente. A continuación, en la Fig.~\ref{fig:weakdeterminacyviolation}, se muestra un ejemplo de este fenómeno recurriendo a la clase de autómatas presentada en la Def.~\ref{def:afca}, tomando como comportamiento interno las transiciones internas y de buffer, y como comportamiento observable las acciones de comunicación, tanto de entrada como de salida.

\begin{figure}[H]
\begin{center}
\label{fig:ejemplo-weak-determinacy}
\subfloat[1][Automatas Finitos de Comunicación Asincrónica S y P]{
\label{fig:weak determinacy afca}
\begin{tikzpicture}[->, thick]
 \node[state,initial] (q_0)   {$q_0$}; 
 \node[state] (q_1) [below = of q_0 ] {$q_1$};
 \node[state] (q_2) [right = of q_1 ] {$q_2$};
 \node[state] (q_3) [below = of q_1 ] {$q_3$};
 \node[state] (q_4) [below = of q_2 ] {$q_4$};
 \draw[]        
        (q_0) edge[left] node{int1} (q_1)
        (q_0) edge[right] node{int2} (q_2)
        (q_1) edge[left] node{in(sp,a)} (q_3)
        (q_2) edge[right] node{in(sp,b)} (q_4)
        ; 
\end{tikzpicture}
\qquad
\begin{tikzpicture}[->, thick]
 \node[state,initial] (q_0)   {$q_0$}; 
 \node[state] (q_1) [below = of q_0 ] {$q_1$};
  \node[state] (q_2) [right = of q_1] {$q_2$};
 \draw[]        
        (q_0) edge[left] node{out(sp,a)} (q_1)
        (q_0) edge[right] node{out(sp,b)} (q_2)
        ; 
\end{tikzpicture}
}

\subfloat[2][Las mCFSM $M_S$ y $M_P$, correspondientes a los autómatas $S$ y $P$, respectivamente]{
\label{fig:weak determinacy mCFSM}
\begin{tikzpicture}[->, thick]
 \node[state,initial] (q_0)   {$[q_0]$}; 
  \node[state] (q_1) [below = of q_0 ] {$[q_1]$};
 \node[state] (q_2) [right = of q_1 ] {$[q_2]$};
 \draw[]        
        (q_0) edge[left] node{sp?a} (q_1)
        (q_0) edge[right] node{sp?b} (q_2)
        ; 
\end{tikzpicture}
\qquad
\begin{tikzpicture}[->, thick]
 \node[state,initial] (q_0)   {$[q_0]$}; 
 \node[state] (q_1) [below = of q_0 ] {$[q_1]$};
  \node[state] (q_2) [right = of q_1 ] {$[q_2]$};
 \draw[]        
        (q_0) edge[left] node{sp!a} (q_1)
        (q_0) edge[right] node{sp!b} (q_2)
        ; 
\end{tikzpicture}
}
\end{center}
\caption{Sistema distribuido de dos componentes.}
\label{fig:weakdeterminacyviolation}
\end{figure}

En la Fig.~\ref{fig:weakdeterminacyviolation} podemos ver un sistema distribuido de dos componentes que a primera vista funciona correctamente pero visto en detalle encontramos que tiene errores en su comportamiento de comunicación y puede entrar en deadlock. Cada componente, expresado en la Fig.~\ref{fig:weak determinacy afca} por los autómatas $S$ y $P$ podemos ver que tiene sus respectivas acciones de comunicación externa y transiciones internas. En su estado inicial $P$ puede enviar tanto $a$ como $b$, indistintamente, pero observando $S$ podemos ver que puede haber deadlock. Esto se debe a que $S$ tiene que ejecutar una transición interna para llegar a un estado donde pueda recibir un mensaje. Dependiendo de qué transición ejecute, es un mensaje u otro. Si ejecuta $int1$, solo puede recibir $a$ y si $P$ envía $b$ el sistema entra en deadlock. Del mismo modo si toma el camino de $int2$ y recibe $a$, no puede seguir ejecutando porque espera el mensaje $b$. Pero en la Fig.~\ref{fig:weak determinacy mCFSM} se ve como las mCFSM proyectadas funcionan como un sistema, ya que en su estado incial ambas máquinas están en la misma posición. Es decir, $M_P$ en $[q_0]$ puede enviar tanto $a$ como $b$, y $M_S$ puede ejecutar correctamente al recibir el mensaje.\\

Para prevenir este tipo de problemas surge la noción de determinismo débil o \emph{weak determinacy} \cite[Chap.~11, Def.~3]{milner89}. La noción de determinismo es importante dado que está relacionada con la predecibilidad del sistema. Si realizamos el mismo experimento dos veces sobre un sistema determinista, comenzando ambas veces del estado inicial, esperamos obtener el mismo resultado o comportamiento cada vez. Además, la predecibilidad es algo que frecuentemente requerimos de sistemas. Por ende si elegimos especificar un sistema $S$ dándole un participante abstracto $Spec$ al que $S$ debería ser equivalente, entonces $Spec$ va a ser determinista. \emph{Weak Determinacy} es una forma más amplia de determinismo que nos permite contemplar una equivalencia observacional entre participantes. Esta equivalencia basada en la definición de bisimulación tiene en cuenta que para un participante externo, las transiciones $\epsilon$ son invisibles y por lo tanto una serie de estados conectados por transiciones consecutivas de este tipo serían bisimilares. En este caso tomamos las transiciones internas o de comunicación interna como observacionalmente bisimilares para un participante externo.  

\begin{definition}[Weak Determinacy]
Sea $A= \langle Q, \mathcal{C}, B, \Sigma, \delta, q_0, F \rangle$ un AFCA y $\Sigma' \subseteq \Sigma$ un conjunto de etiquetas a silenciar. Se dice que $A$ es \emph{weak determinate sobre $\Sigma'$} si para todo $q, q', q'' \in Q$ y para todo $s, s' \in \Sigma^*$ tal que $\widehat{s}_{\Sigma'} = \widehat{s'}_{\Sigma'}$, $q \xrightarrow{s} q'$ y $q \xrightarrow{s'} q''$, $q' \sim_{\Sigma'} q''$.
\end{definition}

\begin{prop}
Dado un AFCA $A = \langle Q, \mathcal{C}, B, \Sigma, \delta, q_0, F \rangle$ weak determinate tal que $\Sigma_\mathit{Int} \subseteq \Sigma$ denota el conjunto de transiciones internas, entonces dados dos estados $q, q' \in Q$ tales que $q \xrightarrow{\sigma} q'$ con $\sigma \in \Sigma_{\mathit{Int}}^*$,  $q \sim_{\Sigma_\mathit{Int}} q'$.
\end{prop}
\begin{proof}
Supongamos que $A$ es weak determinate sobre $\Sigma_\mathit{Int}$ y existen $q, q' \in Q$ tales que $q \xrightarrow{\sigma} q'$ con $\sigma \in \Sigma_{\mathit{Int}}^*$ y no vale que $q \sim_{\Sigma_\mathit{Int}} q'$. Si $q \sim_{\Sigma_\mathit{Int}} q'$ no vale, entonces no existen $q'' \in Q$,  $s, s' \in \Sigma^*$ tal que $\widehat{s}_{\Sigma'} = \widehat{s'}_{\Sigma'}$, $q \xrightarrow{s} q'$ y $q \xrightarrow{s'} q''$, lo que es absurdo pues se puede tomar $q'' = q$, $s = \epsilon$ y $s' = s\sigma$ satisfaciendo que $\widehat{s}_{\Sigma_\mathit{Int}} = \widehat{s'}_{\Sigma_\mathit{Int}}$
\end{proof}

Este lenguaje requiere de determinadas condiciones para asegurar que un sistema distribuido puedan funcionar correctamente. Una vez asegurado eso queremos ver que la composición total de un conjunto de componentes preserva correctamente el comportamiento comunicacional de los mismos. Para esto primero establecemos una propiedad de equivalencia sobre estados de un autómata que permite abstraerse de las acciones internas sin alterar la comunicación. Luego definimos un LTS que resulta de proyectar la semántica de comunicación de un AFCA utilizando la propiedad anterior sobre las transiciones internas. Por último definimos y demostramos.

\begin{definition}[Equivalencia de estados por transición interna] \label{def:eci}
Dado un AFCA $A = \langle Q, \mathcal{C}, B, \Sigma, \delta, q_0, F \rangle$ weak determinate tal que $\Sigma_\mathit{Int} \subseteq \Sigma$ denota el conjunto de transiciones internas, definimos $q \Rightarrow q'$ si y solo si existe $\sigma \in \Sigma_\mathit{Int}$ tal que $q \xrightarrow{\sigma} q'$. Luego, $[q]_m$, \emph{la clase de equivalencia por comunicación interna} de $q$, se define como $[q]_m = \{q'\ |\ q \Rightarrow^\bullet q'\}$. 
\end{definition}

Intuitivamente, la clase de equivalencia por comunicación interna de un estado $q$, $[q]_m$ son aquellos estados $q'$ relacionados en cero o más pasos de transiciones internas con $q$.

Dado un AFCA $A = \langle Q, \mathcal{C}, B, \Sigma, \delta, q_0, F \rangle$, $\Sigma = \Sigma_\mathit{Int} \cup \Sigma_\mathit{Buff}$ y $q \in Q$, . Decimos que dos clases de equivalencia están relacionadas $[q]_m \xrightarrow{\sigma} [q']_m$ sii $\exists q_i \in [q]_m, q'_i \in [q']_m$ tales que $q_i \xrightarrow{\sigma} q'_i \in \delta_{\text{Buff}}$.

\begin{definition}[Proyección de la comunicación de un AFCA]\label{def:pci}\ \\
Dado un AFCA weak determinate $\mathcal{A} = \langle Q, B, \mathcal{C}, \Sigma, \delta, q_0, F \rangle$, tal que su interfaz de comunicación externa es la mCFSM vacía, es decir $\Pi(\mathcal{A}) = \emptyset$, definimos $\Tau(\mathcal{A})$ como la proyección de la semántica de comunicación de $\mathcal{A}$. Llamamos $\widehat{\mathcal{A}}$ al sistema de transición etiquetado resultante y decimos que $\Tau (\langle Q, B, \mathcal{C}, \Sigma, q_0, \delta, F \rangle)= \langle \widehat{Q}, B, \widehat{q_0}, \widehat{\Delta} \rangle$ tal que:
\begin{itemize}
    \item $ \widehat{Q} = \{ \langle [q]_m, \overrightarrow{\Omega} \rangle | q \in Q \land \overrightarrow{\Omega} = (\Omega_b)_{b \in B}$ con $\Omega_b \in \mathcal{M}^* \}$ Cada estado del sistema de transición resultante es una configuración de $\mathcal{A}$ para una clase de equivalencia por comunicación interna,
    \item $B$ es el conjunto de buffers de $\mathcal{A}$ que se utilizan en las aristas del LTS para denotar de comunicación interna
    \item $\widehat{q_0}= \langle q_0, \Omega \rangle$ es el estado inicial de  donde $q_0$ es el estado inicial de $\mathcal{A}$ y $\Omega = \langle [], \ldots, [] \rangle$ es el conjunto de buffers en su estado inicial sin mensajes, y
    \item $\widehat{\Delta}= \{\langle \langle q_i, \Omega \rangle, \widehat{\sigma},\langle q_j, \Omega' \rangle \rangle | q_i, q_j \in Q, \widehat{\sigma} \in \{B \times \{\gg, \ll\} \times \mathcal{M}\} \land \exists \sigma \in \Sigma$ tal que  $[q_i]_m \xrightarrow{\sigma} [q_j]_m \}$. Definimos $\widehat{\Delta}$ como la relación de transición de $\widehat{\mathcal{A}}$. Un estado de $\widehat{q_j}$ es alcanzable desde otro $\widehat{q_i}$ si la clase de equivalencia $[q_j]_m$ es alcanzable desde $[q_i]_\mathcal{M}$.
\end{itemize}
\end{definition}

Partimos de la hipótesis que para poder extraer el LTS de semántica de comunicación interna es necesario que el autómata no tenga transiciones de comunicación externa. Esto es debido a que la comunicación asincrónica externa introduce otra complejidad a tener en cuenta a la hora de abstraernos correctamente de las transiciones internas. Para los propósitos de lo que queremos demostrar estas condiciones son suficientes dado que trabajamos sobre AFCA que resultan de la composición total de un conjunto de autómatas.\\ 

A continuación vemos un ejemplo de la Proyección de Comunicación según la Def.~\ref{def:pci}. En la primera figura se puede ver los tres autómatas originales (Fig.~\ref{a_i}), su composición $\mathcal{A}$ (Fig.~\ref{A}) y el LTS resultante de la proyección $\hat{\mathcal{A}}$ (Fig.~\ref{hatA}). Luego vemos el conjunto de mCFSMs correspondientes según Def.~\ref{def:interfazcom} (Fig.~\ref{m_i}) y su CS representado por el LTS $M$ (Fig.~\ref{M}). Comparando ambos sistemas de transición $M$ y $\hat{\mathcal{A}}$, se puede ver que son muy similares. Los nodos de cada LTS muestran distintos estados de la comunicación, incluyendo el contenido de los canales de comunicación, externos en el caso de $M$ e internalizados en forma de buffers para $\hat{\mathcal{A}}$, que en ambos casos es el mismo en cada paso. Las aristas de $M$ y $\hat{\mathcal{A}}$ denotan las mismas acciones comunicación también como esterna e interna, respectivamente. Este ejemplo es una intuición de que el mecanismo de composición preserva la semántica de comunicación externa al internalizarla.

\begin{figure}[H]
\begin{center}
\begin{tikzpicture}[->, thick]
    \node[state,initial] (q_0) {$[q_{000}],\epsilon, \epsilon$}; 
     \node[state] (q_1) [right= 3.5cm of q_0 ] {$[q_{100}], a, \epsilon$};
     \node[state] (q_2) [right = 3.5cm of q_1 ] {$[q_{120}],\epsilon,\epsilon$};
     \node[state] (q_3) [below= of q_2 ] {$[q_{121}],\epsilon, b$};
     \node[state] (q_4) [below=  of q_0] {$[q_{001}],\epsilon, b$};
     \node[state] (q_5) [below = of q_4] {$[q_{201}],\epsilon, \epsilon$};
     \node[state] (q_6) [right = 3.5cm of q_5 ] {$[q_{301}], a, \epsilon$};
     \node[state,accepting] (q_7) [below = of q_3 ] {$[q_{321}],\epsilon, \epsilon$};
\draw[]        
     (q_0) edge[above] node{$\langle q_{000}, PR \ll a, q_{100} \rangle$} (q_1)
	 (q_1) edge[above] node{$\langle q_{110}, PR \gg a, q_{120} \rangle$} (q_2)
     (q_2) edge[left] node{$\langle q_{120}, SP \ll b, q_{121} \rangle$} (q_3) 
     (q_3) edge[left] node{$\langle q_{121}, SP \gg b, q_{321} \rangle$} (q_7)
     (q_0) edge[left] node{$\langle q_{000}, SP \ll b, q_{001} \rangle$} (q_4)
     (q_4) edge[left] node{$\langle q_{001}, SP \gg b, q_{201} \rangle$} (q_5)
     (q_5) edge[below] node{$\langle q_{201}, SP \gg b, q_{301} \rangle$} (q_6)
     (q_6) edge[below] node{$\langle q_{311}, PR \ll a, q_{321} \rangle$} (q_7)
      ;
\end{tikzpicture}
\end{center}
\caption{Proyección de comunicación interna del AFCA de la Fig.~\ref{A}.}
\label{ex:pci}
\end{figure}

\begin{figure}[H]
\begin{center}
\begin{tikzpicture}[->, thick][scale=0.20]
 \node[state,initial] (q_0)   {$q_{000},\epsilon,\epsilon$}; 
 \node[state] (q_1) [right= 3cm of q_0 ] {$q_{100},a, \epsilon$};
 \node[state] (q_2) [right = 3cm of q_1 ] {$q_{120},\epsilon, \epsilon$};
 \node[state] (q_3) [below = of q_2 ] {$q_{121}, \epsilon,b$};
 \node[state,accepting] (q_4) [below=  of q_3] {$q_{321}, \epsilon,\epsilon$};
 \node[state] (q_5) [below = of q_0] {$q_{001}, \epsilon, b$};
 \node[state] (q_6) [below = of q_5 ] {$q_{201}, \epsilon, \epsilon$};
 \node[state] (q_7) [right= 3cm of q_6 ] {$q_{301}, a, \epsilon$};
 \draw[]        
        (q_0) edge[above] node{$\langle q_{0p},PR!a,q_{1p} \rangle$} (q_1)
        (q_1) edge[above] node{$\langle q_{0r},PR?a,q_{1r} \rangle$} (q_2)
        (q_2) edge[right] node{$\langle q_{0s},SP!b,q_{1s} \rangle$} (q_3)
        (q_3) edge[right] node{$\langle q_{1p},SP?b,q_{3p} \rangle$} (q_4)
        (q_0) edge[left] node{$\langle q_{0s},SP!b,q_{1s} \rangle$} (q_5)
        (q_5) edge[left] node{$\langle q_{0p},SP?b,q_{2p} \rangle$} (q_6)
        (q_6) edge[below] node{$\langle q_{2p},PR!a,q_{3p} \rangle$} (q_7)
        (q_7) edge[below] node{$\langle q_{0r},PR?a,q_{1r} \rangle$} (q_4)
        ;
\end{tikzpicture}
\end{center}
\caption{LTS del CS formado por las mCFSMs de la Fig.~\ref{m_i}.}
\label{ex:CS}
\end{figure}

Para demostrar la equivalencia semántica de comunicación entre un conjunto de AFCAs compatibles y sus mCFSMs queremos probar que los LTS que representan las respectivas semánticas son bisimilares. Antes de demostrar esto necesitamos demostrar que los estados son equivalentes. Los estados de ambos LTS contienen un estado discreto compuesto y el conjunto de buffers/canales con su contenido en cada estado. En el Lema~$\ref{lema1}$ demostramos que los estados discretos contenidos en los estados de la proyección de comunicación interna son equivalentes a los del LTS que representa el CS, mientras que en el Lema~$\ref{lema2}$ probamos que el comportamiento y el contenido de los buffers es equivalente al de los canales. 

\begin{lemma}[Equivalencia de estados discretos]
\label{lema1} 
Sea $\widehat{\mathcal{A}}=\langle \widehat{Q}, B, \widehat{q_0}, \widehat{\delta} \rangle$, la proyección de comunicación interna del AFCA $\mathcal{A}$ que surge de la composición del conjunto de autómatas  $\{\mathcal{A}_i\}_{1 \leq i \leq n}$  y $M= \langle [Q], \mathcal{C}, [q_0], [\delta] \rangle$, el LTS que representa la semántica del CS correspondiente a $\{M_i\}_{1 \leq i \leq n}$, el conjunto de mCFSM de $\{\mathcal{A}\}_{1 \leq i \leq n}$. Para todo $\langle [\overrightarrow{q}]_m, \overrightarrow{\Omega} \rangle \in \widehat{Q}$, existe $\langle \overrightarrow{p}, \overrightarrow{\Omega_{\mathcal{C}}} \rangle \in P$ tal que $q_i \in \overrightarrow{q} \iff q_i \in \overrightarrow{p}$ y viceversa. 
\end{lemma}
\begin{proof} 
\begin{itemize}
    \item[$\implies$] Dado $\widehat{q} \in \widehat{Q}$, un estado de la proyección $\widehat{\mathcal{A}}$. Por definición de $\widehat{\mathcal{A}}$, $\widehat{q}= \langle [\overrightarrow{q}]_m, \overrightarrow{\Omega} \rangle$. Por definición de equivalencia por comunicación interna (ver Def.~\ref{def:eci}) $[\overrightarrow{q}]_m = \left\{\overrightarrow{q'}\ |\ \overrightarrow{q} \xrightarrow{\sigma}\overrightarrow{q'}\right\}$ con $\overrightarrow{q},\overrightarrow{q'} \in Q$ y $\sigma \in \Sigma_\mathit{Int}^*$ o sea son los estados alcanzables por transiciones internas desde $\overrightarrow{q}$. Por lo tanto $\langle [\overrightarrow{q}]_m, \overrightarrow{\Omega} \rangle \in \widehat{Q}$ con $\overrightarrow{q} \in Q$. Por definición de composición de AFCA (ver Def.~\ref{def:composicion}) $\overrightarrow{q} = \langle q_1, \ldots, q_i, \ldots, q_n \rangle$ donde $q_i \in Q_i$ es un estado del AFCA $\mathcal{A}_i$. Como el conjunto de AFCA $\{\mathcal{A}_i\}_{1 \leq i \leq n}$ no posee comunicación interna, la comunicación interna de $\mathcal{A}$ es la comunicación externa de sus componentes internalizada a través de la composición. Por lo tanto, $q_i$ es un estado alcanzable por una sucesión de transiciones internas y externas de $\mathcal{A}_i$. Entonces $\overrightarrow{q} \in Q$ si y solo si $q_i \in Q_i$, para todo $1 \leq i \leq n$, y por definición de interfaz de comunicación de AFCA (ver Def.~\ref{def:interfazAFCA}), existe $[q_i] \in [Q_i]$, el conjunto de estados de la mCFSM $M_i$. Por lo tanto, por definición de CS (ver Def.~\ref{def:CS}), existe $\overrightarrow{p} = \langle [q_1], \ldots, [q_i], \ldots, [q_n] \rangle$ tal que, para algún $\overrightarrow{\Omega_\mathcal{C}}$, $\langle \overrightarrow{p}, \overrightarrow{\Omega_\mathcal{C}} \rangle \in P$.
    
    \item[$\impliedby$] Sea $p \in P$, un estado del LTS $M$ que representa la semántica el CS. Por definición de CS (ver Def.~\ref{def:CS}), vale $p = \langle \langle [q_1],\ldots,[q_i], \ldots, [q_n] \rangle, \overrightarrow{\Omega} \rangle \in P$ con $[q_i] \in [Q_i]$, para todo $1 \leq i \leq n$, donde $[Q_i]$ es el conjunto de estados de la mCFSM $M_i$. Sea $\mathcal{A}_i$ el AFCA cuya interfaz de comunicación es $M_i$ (ver Def.~\ref{def:interfazAFCA}) y cuyo conjunto de estados es $Q_i$, para todo $1 \leq i \leq n$. Por definición de interfaz de comunicación (ver Def.~\ref{def:interfazAFCA}), existen estados $\widehat{q_i} \in Q_i$ alcanzables en cada componente tal que $\widehat{q_i} \in [q_i]$, para todo $1 \leq i \leq n$. Por definición de composición (ver Def.~\ref{def:composicion}), $\overrightarrow{q} = \langle \widehat{q_1}, \widehat{q_2}, \ldots, \widehat{q_i}, \ldots, \widehat{q_n} \rangle \in Q$. $\overrightarrow{q}$ es alcanzable en $\mathcal{A}$ y, por equivalencia de comunicación interna (ver Def.~\ref{def:eci}), su clase de comunicación interna $[\overrightarrow{q}]_m$. Entonces, por proyección de comunicación interna (ver Def.~\ref{def:pci}), existe $\widehat{q} = \langle [\overrightarrow{q}]_m, \overrightarrow{\Omega} \rangle \in \widehat{Q}$, para algún $\overrightarrow{\Omega}$.
\end{itemize}
\end{proof}

\begin{lemma}[Equivalencia entre buffers y canales]\label{lema2} Sean $\widehat{\mathcal{A}}=\langle \widehat{Q}, B, \widehat{q_0}, \widehat{\delta} \rangle$, la proyección de comunicación interna del AFCA $\mathcal{A}$ que surge de la composición del conjunto de autómatas  $\{\mathcal{A}_i\}_{1 \leq i \leq n}$  y $M= \langle P, \mathcal{C}, p_0, [\delta] \rangle$, el LTS que representa la semántica del CS correspondiente a $\{M_i\}_{1 \leq i \leq n}$, el conjunto de mCFSM de $\{\mathcal{A}_i\}_{1 \leq i \leq n}$. Dados $\langle [\overrightarrow{q}]_m, \overrightarrow{\Omega} \rangle \in \widehat{Q}$ y $\langle \overrightarrow{p}, \overrightarrow{\Omega_{\mathcal{C}}} \rangle \in P$. Queremos ver que el conjunto de buffers $B$ de $\mathcal{A}$ es el conjunto de canales internalizados de $\mathcal{C}$ de $M$. Es decir que para todo canal $\omega$ vale que $\omega \in \mathcal{C}$ si y solo si $\omega \in B$.
\end{lemma}
\begin{proof} Sea $\langle \overrightarrow{p}, \overrightarrow{\Omega_{\mathcal{C}}} \rangle$ un estado del LTS $M$. Por definición de CS (ver Def.~\ref{def:CS}), $\overrightarrow{\Omega_{\mathcal{C}}}$ es el conjunto de buffers asociados a los canales $\mathcal{C}$, con su contenido en el estado discreto $\overrightarrow{p}$. Estos canales corresponden a las mCFSM $\{M_i\}_{1 \leq i \leq n}$ que componen el CS. Entonces vale $\omega \in \overrightarrow{\Omega_{\mathcal{C}}}$ si y solo si $\omega \in \mathcal{C}_i$ con $1 \leq i \leq n$. Sea $\mathcal{A}_i$ el AFCA tal que $M_i$ es su interfaz de comunicación. Por definición de interfaz de comunicación de AFCA (ver Def.~\ref{def:interfazAFCA}), $\omega \in \mathcal{C}_i$ si y solo si $\omega \in {\mathcal{C}}_{\mathcal{A}_i}$. Por definición de composición de AFCA (ver Def.~\ref{def:composicion}), la comunicación externa se internaliza. Entonces como $\mathcal{A}_i$ no tienen buffers ni comunicación interna, vale $\omega \in {\mathcal{C}}_{\mathcal{A}_i}$ si y solo si $\omega \in B$ donde $B$ es el conjunto de buffers de comunicación interna de $\mathcal{A}$. Luego, por definición de proyección de comunicación interna (ver Def.~\ref{def:pci}), el conjunto de buffers de $\mathcal{A}$ es el mismo que en $\widehat{\mathcal{A}}$. Por lo tanto $\omega \in B$ si y solo si $\exists \langle [\overrightarrow{q}]_m, \overrightarrow{\Omega} \rangle \in \widehat{Q}$ tal que $\omega \in \overrightarrow{\Omega}$.
\end{proof}

\begin{theorem}[Equivalencia de las semánticas de las mFCSMs y los AFCAs] Sea $\{\mathcal{A}\}_{1 \leq i \leq n}$ un conjunto de weak determinate AFCA sin transiciones de comunicación interna, tal que su composición $\mathcal{A} = \langle Q, B, \mathcal{C},\Sigma, \delta, q_0, F \rangle$, también lo es. Llamamos $\widehat{\mathcal{A}} = \langle \widehat{Q}, B, \widehat{q_0}, \widehat{\delta} \rangle$ al sistema de transición etiquetado (LTS) que resulta de la proyección $\Tau(\mathcal{A})$. Sea el conjunto $\{M_i\}_{1 \leq i \leq n}$ de mCFSM correspondientes a cada uno de los AFCAs en $\{\mathcal{A}\}_{1 \leq i \leq n}$ tal que $M_i = \langle [Q_i], \mathcal{C}, [{q_0}_i], [\delta_i] \rangle$. Llamamos $M = \langle P, \mathcal{C}, {p_0}, [\delta] \rangle$ al LTS que representa la semántica del CS. Luego $\widehat{\mathcal{A}}$ y $M$ son bisimilares. 
\end{theorem}

\begin{proof}
Queremos probar que la relación $R \subseteq \widehat{Q} \times P$, definida de la siguiente manera $R = \{ \langle \langle [\overrightarrow{q}]_m, \overrightarrow{\Omega} \rangle, \langle \overrightarrow{p}, \overrightarrow{\Omega_{\mathcal{C}}} \rangle \rangle \ | \ [\overrightarrow{q}]_mR'\overrightarrow{p} \land  \overrightarrow{\Omega}=\overrightarrow{\Omega_{\mathcal{C}}} \}$ es una bisimulación, con $R'=\{\langle [\overrightarrow{q}]_m, \overrightarrow{p} \rangle \ | \ q_i \in \overrightarrow{q} \iff q_i \in \overrightarrow{p} \}$.

Es decir que para cada par de elementos $\widehat{q} \in \widehat{Q}$, $p \in P$ para los que vale $\langle \widehat{q}, p \rangle \in R$, queremos probar que:
\begin{itemize}
\item $\forall \widehat{\sigma} \in \widehat{\Sigma} | \widehat{q} \xrightarrow{\widehat{\sigma}} \widehat{q}' \implies \exists p' \in P$ tal que $p \xrightarrow{\sigma_M} p' \land \langle \widehat{q}', p', \rangle \in R$, y simétricamente 
\item $\forall \sigma_M \in \Sigma_M | p \xrightarrow{\sigma_M} p \implies \exists \widehat{q}' \in \widehat{Q}$ tal que $\widehat{q} \xrightarrow{\widehat{\sigma}} \widehat{q}' \land \langle \widehat{q}', p'\rangle \in R$. 
\end{itemize}
Donde $\widehat{\Sigma} \subseteq B \times \{\gg,\ll\} \times \mathcal{M}$, $\Sigma_M \subseteq \mathcal{A} \times \{!,?\} \times \mathcal{M}$, el conjunto de buffers $B$ son los canales internalizados del conjunto $\mathcal{C}$ y si $\widehat{\sigma}= (a_ka_j)_l \gg m$ entonces $\sigma_M = (a_ka_j)_l?m$,  si $\widehat{\sigma}= (a_ka_j)_l \ll m$ entonces $\sigma_M = (a_ka_j)_l!m$. 

Tomamos inicialmente como etiquetas $\widehat{\sigma} = (a_ka_j)_l \gg m$ y $\sigma_M = (a_ka_j)_l?m$, 

Primero demostraremos la condición $\forall (a_ka_j)_l \gg m \in \widehat{\Sigma} | \widehat{q} \xrightarrow{(a_ka_j)_l \gg m} \widehat{q}' \implies \exists p' \in P$ tal que $p \xrightarrow{(a_ka_j)_l?m} p' \land \langle \widehat{q}', p', \rangle \in R$. 
 
\begin{itemize}
    
    \item Dado el par $\langle [\overrightarrow{q}]_m, \overrightarrow{\Omega} \rangle \in \widehat{Q}, \langle \overrightarrow{p}, \overrightarrow{\Omega_\mathcal{C}} \rangle \in P$ tal que $\langle \langle [\overrightarrow{q}]_m, \overrightarrow{\Omega} \rangle, \langle \overrightarrow{p}, \overrightarrow{\Omega_\mathcal{C}} \rangle \rangle \in R$. Sean $(a_ka_j)_l \gg m \in \widehat{\Sigma}$ y $\langle [\overrightarrow{q}']_m, \overrightarrow{\Omega}' \rangle \in \widehat{Q}$ tales que $\langle [\overrightarrow{q}]_m, \overrightarrow{\Omega} \rangle \xrightarrow{(a_ka_j)_l \gg m} \langle [\overrightarrow{q}']_m, \overrightarrow{\Omega}' \rangle \in \widehat{\Delta}$.  Esto vale si y solo si la clase de equivalencia por comunicación interna $[\overrightarrow{q}']_m$ está relacionada con $[\overrightarrow{q}]_m$ vía $(a_ka_j)_l \gg m$ (ver Def.~\ref{def:eci}), o sea $[\overrightarrow{q}]_m \xrightarrow{(a_ka_j)_l \gg m} [\overrightarrow{q}']_m$. Entonces $\overrightarrow{\Omega} = [(a_1a_2), \ldots, (a_ka_j)_l \cdot m \ldots (a_na_m)]$ y $\overrightarrow{\Omega}' = [(a_1a_2), \ldots, (a_ka_j)_l \ldots (a_na_m)] \rangle$.
    
    \item Sean $\overrightarrow{q_{\sigma}} \in [\overrightarrow{q}]_m, \overrightarrow{q_{\sigma}}' \in [\overrightarrow{q}']_m$, por definición de proyección de comunicación interna (ver Def.~\ref{def:pci}) vale $\overrightarrow{q_{\sigma}} \xrightarrow{(a_ka_j)_l \gg m} \overrightarrow{q_{\sigma}}' \in \delta$. Por definición de composición sabemos que vale $\overrightarrow{q_{\sigma}} =\langle q_0, q_1, \ldots, q_j, \ldots,q_n \rangle$ y $\overrightarrow{q_{\sigma}}' =\langle q_0, q_1, \ldots, q'_j, \ldots,q_n \rangle$ donde $q_j,q'_j$ son estados de $\mathcal{A}_j$, componente de $\mathcal{A}$. Por definición de AFCA (ver Def.~\ref{def:afca}), $(a_ka_j)_l \gg m$ es una operación de desencolamiento del mensaje $m$ sobre el buffer $(a_ka_j)_l$. Por definición de composición de AFCA (ver Def.~\ref{def:composicion}), como los AFCA del conjunto no tienen acciones de comunicación interna, esta acción proviene de una recepción de un mensaje por parte de $\mathcal{A}_j$ vía el canal $(a_ka_j)_l$. Entonces podemos afirmar que se cumple $\overrightarrow{q_{\sigma}} \xrightarrow{(a_ka_j)_l \gg m} \overrightarrow{q_{\sigma}}'\in \delta$ si y solo si $q_j,q'_j \in Q_j \land q_j \xrightarrow{(a_ka_j)_l?m} q'_j \in \delta_j$.
    
    \item Por interfaz de comunicación de AFCA (ver Def.~\ref{def:interfazAFCA}) podemos obtener $M_j$, la mCFSM correspondiente a $\mathcal{A}_j$, y sabemos que $q_j \xrightarrow{(a_ka_j)_l?m} q'_j \in \delta_j$ si y solo si se cumple que $[q_j], [q'_j] \in [Q_j] $, y $[q_j] \xrightarrow{(a_ka_j)_l?m} [q'_j] \in [\delta_j]$. 
    
    \item Entonces, por definición del LTS M (ver Def.~\ref{def:CS}), existen $p'=\langle \overrightarrow{p'}, \overrightarrow{\Omega_\mathcal{C}} \rangle \in P, (a_ka_j)_l?m \in \Sigma_M$ tal que $\overrightarrow{p'}=\langle [q_1], [q_2], \ldots, [q'_j], \ldots [q_n] \rangle$, $\overrightarrow{\Omega'_\mathcal{C}}=[\ldots, (a_ka_j)_l, \ldots]$, y $p \xrightarrow{(a_ka_j)_l?m} p'$. Donde $\overrightarrow{\Omega_\mathcal{C}}$ y $\overrightarrow{\Omega_\mathcal{C}}'$ son los canales de comunicación externa para los estados $\overrightarrow{p}$ y $\overrightarrow{p'}$, y vale $\overrightarrow{\Omega_\mathcal{C}} = [(a_1a_2), \ldots, (a_ka_j)_l \cdot m \ldots (a_na_m)]$ y $\overrightarrow{\Omega_\mathcal{C}}' = [(a_1a_2), \ldots, (a_ka_j)_l \ldots (a_na_m)] \rangle$. Por lo tanto vale $\langle \widehat{q'}, p' \rangle \in R$.
    
    \item Por lo tanto vale que $\forall (a_ka_j)_l \gg m \in \widehat{\Sigma} | \widehat{q} \xrightarrow{(a_ka_j)_l \gg m} \widehat{q}' \implies \exists p' \in P$ tal que $p' \xrightarrow{(a_ka_j)_l?m} p' \land \langle \widehat{q}', p', \rangle \in R$ donde $\widehat{\Sigma} \subseteq B \times \{\gg,\ll\} \times \mathcal{M}$
  \end{itemize}
  
  Demostraremos ahora la condición simétrica $\forall (a_ka_j)_l?m \in \Sigma_M | p \xrightarrow{(a_ka_j)_l?m} p' \implies \exists \widehat{q}' \in \widehat{Q}$ tal que $\widehat{q} \xrightarrow{(a_ka_j)_l \gg m} \widehat{q}' \land \langle \widehat{q}', p'\rangle \in R$ 
  \begin{itemize}

     \item Dado el par $\langle q, p \rangle = [\overrightarrow{q}]_m, \overrightarrow{\Omega} \rangle \in \widehat{Q}, \langle \overrightarrow{p}, \overrightarrow{\Omega_\mathcal{C}} \rangle \in P$ tal que $\langle \langle [\overrightarrow{q}]_m, \overrightarrow{\Omega} \rangle, \langle \overrightarrow{p}, \overrightarrow{\Omega_\mathcal{C}} \rangle \rangle \in R$. Sean $(a_ka_j)_l?m$ y $p' = \langle \overrightarrow{p}', \overrightarrow{\Omega_\mathcal{C}}' \rangle \in P$ tales que $\langle \overrightarrow{p}, \overrightarrow{\Omega_\mathcal{C}} \rangle \xrightarrow{(a_ka_j)_l?m} \langle \overrightarrow{p}', \overrightarrow{\Omega_\mathcal{C}}' \rangle \in \delta_P$. Por definición del LTS $M$ (ver Def.~\ref{def:CS}), $(a_ka_j)_l?m$ es una operación sobre el canal $(a_ka_j)_t$ de la mCFSM $M_j$, componente del CS. Dado que el LTS $M$ representa la semántica del CS compuesto por el conjunto de mCFSMs, vale que $\overrightarrow{p}=\langle [q_1], [q_2], \ldots [q_j], \ldots [q_n] \rangle$, $\overrightarrow{p}'= \langle [q_1], [q_2], \ldots, [q'_j], \ldots [q_n] \rangle$, donde $[q_j]$ y $[q'_j]$ son estados de la mCFSM $M_j$, y vale $[q_j] \xrightarrow{(a_ka_j)_l?m} [q'_j] \in [\delta]_j$
    
    \item Por definición de intefaz de comunicación de AFCA (ver Def.\ref{def:interfazAFCA}) sabemos que $M_J$ es la mCFSM correspondiente a $\mathcal{A}_j$, componente de $\mathcal{A}$. Por el Lema \ref{lema1}, como $[q_j] \xrightarrow{(a_ka_j)_l?m} [q'_j] \in [\delta_j]$ entonces $q_j \xrightarrow{(a_ka_j)_l?m} q'_j \in \delta_j$.
    
    \item Por definición de composición de AFCA (ver Def.~\ref{def:composicion}) $q_j$ y $q'_j$ son componentes de dos estados compuestos que llamamos $\overrightarrow{q}_{\sigma}$ y $\overrightarrow{q}_{\sigma}'$, respectivamente. De modo tal que $\overrightarrow{q}_{\sigma} \xrightarrow{(a_ka_j)_l \gg m} \overrightarrow{q}_{\sigma}' \in \delta$. Donde $(a_ka_j)_l$ es el canal internalizado en forma de buffer.
    
    \item Por definición de equivalencia por comunicación interna (ver Def.~\ref{def:eci}) como vale $\overrightarrow{q}_{\sigma} \xrightarrow{(a_ka_j)_l \gg m} \overrightarrow{q}_{\sigma}' \in \delta$, entonces existen las clases de equivalencia $[\overrightarrow{q}]_m$, $[\overrightarrow{q}']_m$, tales que $\overrightarrow{q}_{\sigma} \in [\overrightarrow{q}]_m $, $\overrightarrow{q}_{\sigma}' \in [\overrightarrow{q}']_m$, y $[\overrightarrow{q}]_m \xrightarrow{(a_ka_j)_l \gg m} [\overrightarrow{q}']_m$. Por proyección de comunicación interna (ver Def.~\ref{def:pci}), existen $\widehat{q}' = \langle [\overrightarrow{q}']_m, \overrightarrow{\Omega}' \rangle \in \widehat{Q}, (a_ka_j)_l \gg m \in \widehat{\Sigma}$, tales que $\langle [\overrightarrow{q}]_m, \overrightarrow{\Omega} \rangle \xrightarrow{(a_ka_j)_l \gg m} \langle [\overrightarrow{q}']_m, \overrightarrow{\Omega}' \rangle$. Donde, $\overrightarrow{\Omega}$, y $\overrightarrow{\Omega}'$ son los buffers de $\mathcal{A}$ para los estados $\overrightarrow{q}_{\sigma}$ y  $\overrightarrow{q}_{\sigma}'$, respectivamente, y se cumple que $\overrightarrow{\Omega} = [(a_1a_2), \ldots, (a_ka_j)_l \cdot m \ldots (a_na_m)]$ y $\overrightarrow{\Omega}' = [(a_1a_2), \ldots, (a_ka_j)_l \ldots (a_na_m)] \rangle$. Por lo tanto $\langle \widehat{q}', p' \rangle \in R$ 
    
    \item Por lo tanto vale que $\forall (a_ka_j)_l?m \in \Sigma_M | p \xrightarrow{(a_ka_j)_l?m} p \implies \exists \widehat{q}' \in \widehat{Q}$ tal que $\widehat{q} \xrightarrow{(a_ka_j)_l \gg m} \widehat{q}' \land \langle \widehat{q}', p'\rangle \in R$ donde $\Sigma_M \subseteq \mathcal{C} \times \{!,?\} \times \mathcal{M}$
    
\end{itemize}
\end{proof}

