%!TEX root = ./main.tex

Como ya hemos mencionado, en SOC los sistemas son concebidos como objetos dinámicos construidos en \emph{run-time} en la medida que su ejecución llega a un estado en el que la intervención de servicios externos se hace necesaria. Es decir un sistema de este tipo utilizará distintos servicios según las necesidades que se manifiesten a lo largo de una ejecución particular. 
 
Una aplicación que se encuentra ejecutando se conecta con los servicios que le son necesarios a través de canales de comunicación por los cuales se envían o reciben mensajes. Estos canales pueden establecer una comunicación entre un número fijo (para cada canal particular) pero no acotado a priori de servicios. En la sección anterior hemos detallado un conjunto de propiedades que garantizan una comunicación sin errores a través de estos canales (i.e. ausencia de deadlock, ausencia de mensajes huérfanos y ausencia de situaciones en las que el receptor no se encuentra a la espera de un mensaje que le fue enviado) y un procedimiento para garantizarlas. Estas condiciones y su procedimiento de análisis parten de la hipótesis de que las CFSMs correspondientes a todos y cada uno de los participantes de la comunicación sobre dicho canal se encuentran disponibles.

Ahora bien, que la aplicación arribe a un estado en el que un servicio se hace necesario sobre un canal particular, no implica que todos los participantes también lo sean en ese mismo instante y por ello, con el objeto de profundizar esta concepción incremental, a demanda, que se tiene sobre los sistemas de software surge, más o menos naturalmente, la idea de poder dotar al \emph{middleware} de la capacidad de realizar un \emph{binding} parcial sobre los canales. A esta práctica la llamaremos \emph{binding incremental}.
 
Esta percepción parcial del \emph{binding} sobre un canal requiere la utilización de un un lenguaje de descripción que soporten dichos mecanismos de composición. Por ejemplo, al componer dos CFSMs puede ocurrir que cada máquina se comunique a través de un canal con un tercer participante, estos dos canales son independientes y, además, una vez que se ha realizado la composición, deben ser percibidos por este tercer participante como canales de comunicación separados. Por lo tanto, para preservar la semántica de la comunicación, es necesario que el CFSM resultante de la composición tenga dos canales con este tercer participante. Este fenómeno será el eje rector de las modificaciones que introduciremos en esta sección. Esta característica no solo será necesario a nivel de CFSM (interfaz de comunicación de un servicio) sino también de los autómatas que caracterizan el cómputo, cuya interfaz de comunicación es expresada a través de una CFSM.

Para modelar este comportamiento introducimos los Autómatas Finitos de Comunicación Asincrónica (AFCA). Estos autómatas tienen tres tipos de transiciones: 
\begin{inparaenum}[1)]
\item internas, que sirven el propósito de representar cómputo realizado por la componente; 
\item de buffer, que representan comunicación asincrónica interna entre elementos de la componente, y que permiten representar la comunicación entre dos participantes luego de una composición; y por último 
\item de entrada / salida, que modelan acciones de comunicación con otras componentes del sistema.
\end{inparaenum} \\
Esta sección va a estar enfocada en la definición de composición total de autómatas y la noción de que dado un conjunto $\{ \mathcal{A}_i\}_{i \in I}$ de AFCA weak deterministic sin transiciones de buffer la semántica de la composición es equivalente a la del conjunto de mCFSM proyectadas de los elementos de $\{ \mathcal{A}_i\}_{i \in I}$. 

\begin{figure}[ht]
%\begin{center}
\xymatrix{   
	\{\mathcal{A}_i\}_{i \in I} \ar[rrr]_{\Pi} \ar[d]_{||} & & & \{C_i\}_{i \in I}  \ar[d]_{semantica}  \\
	  {A} \ar[r]_{\Pi'} \ar[d]_{\Pi} & \mathcal{L}' & = & {\mathcal{L}}  \\
	  \emptyset
}     
%\end{center}
\caption{La composición de componentes asincrónicas preserva la semántica asincrónica de su interfaz de comunicación.}
\label{fig:preservacion}
\end{figure}


En general, cuando se trata de formalismos que modelan simultáneamente aspectos del comportamiento interno de una componente y elementos de interfaz observable por otras componentes, la proyección de dicha interfaz no resulta bisimilar con el comportamiento real de la componente. A continuación se muestra un ejemplo de este fenómeno recurriendo a la clase de autómatas presentada en la \ref{def:afca}, tomando como comportamiento interno las transiciones internas y de buffer y como comportamiento observable las acciones de comunicación, tanto de entrada como de salida, con otras componentes.

\begin{figure}[ht]
\begin{tikzpicture}[->, thick]
 \node[state,initial] (q_0)   {$q_0$}; 
 \node[state] (q_1) [below = of q_0 ] {$q_1$};
 \node[state] (q_2) [right = of q_1 ] {$q_2$};
 \node[state] (q_3) [below = of q_1 ] {$q_3$};
 \node[state] (q_4) [below = of q_2 ] {$q_4$};
 \node[state, accepting] (q_5) [below = of q_3 ] {$q_5$};
 \draw[]        
        (q_0) edge[left] node{int1} (q_1)
        (q_0) edge[right] node{int2} (q_2)
        (q_1) edge[left] node{in(sp,a)} (q_3)
        (q_3) edge[left] node{in(sp,b)} (q_5)
        (q_2) edge[right] node{in(sp,b)} (q_4)
        ; 
\end{tikzpicture}
\qquad
\begin{tikzpicture}[->, thick]
 \node[state,initial] (q_0)   {$q_0$}; 
 \node[state] (q_1) [below = of q_0 ] {$q_1$};
  \node[state, accepting] (q_2) [below = of q_1 ] {$q_2$};
 \draw[]        
        (q_0) edge[left] node{out(sp,a)} (q_1)
        (q_1) edge[left] node{out(sp,b)} (q_2)
        ; 
\end{tikzpicture}

\caption{Los AFCA P y S, donde P no es Weak deterministic}
\label{fig:weak determinacy afca}
\end{figure}

\begin{figure}[ht]
    \centering
    
\begin{tikzpicture}[->, thick]
 \node[state,initial] (q_0)   {$q_0$}; 
  \node[state] (q_1) [below = of q_0 ] {$q_1$};
 \node[state] (q_2) [right = of q_1 ] {$q_2$};
 \node[state, accepting] (q_3) [below = of q_1 ] {$q_3$};
 \draw[]        
        (q_0) edge[left] node{sp?a} (q_1)
        (q_0) edge[right] node{sp?b} (q_2)
        (q_1) edge[left] node{sp?b} (q_3)
        ; 
\end{tikzpicture}
\qquad
\begin{tikzpicture}[->, thick]
 \node[state,initial] (q_0)   {$q_0$}; 
 \node[state] (q_1) [below = of q_0 ] {$q_1$};
  \node[state, accepting] (q_2) [below = of q_1 ] {$q_2$};
 \draw[]        
        (q_0) edge[left] node{sp!a} (q_1)
        (q_1) edge[left] node{sp!b} (q_2)
        ; 
\end{tikzpicture}

   \caption{Las mcfsm correspondientes a los autómatas S y P}
    \label{fig:weak determinacy mcfsm}
\end{figure}

En la figura 3.4 podemos ver al sistema compuesto de los AFCA $S$ y $P$. En la misma podemos observar como $S$ tiene un único camino lineal donde envía los meensajes $a$ y $b$ al autómata $P$. Por su lado $P$ puede tomar dos caminos, ejecutando una de sus dos transiciones internas. Esas transiciones no afectan directamente el comportamiento del autómata $S$. Pero cual de ellas elija ejecutar $P$, sí lo afecta. Al tomar el primer camino el sistema se ejecuta de manera correcta, pero si ejecuta la transición interna $int2$, no puede recibir el mensaje $a$ y por lo tanto el sistema entra en deadlock. El autómata $P$ es un autómata determinístico, pero esa noción no alcanza para que el sistema funciones correctamente. En la figura 3.5 se ve como la mCFSM proyectada preserva los errores comunicacionales.  

Para prevenir este tipo de problemas surgen la noción de determinismo débil o weak determinacy. La noción de determinismo es importante dado que está relacionada con la predecibilidad del sistema. Si realizamos el mismo experimento dos veces sobre un sistema determinista, comenzando ambas veces del estado inicial, esperamos obtener el mismo resultado o comportamiento cada vez. Además, la predecibilidad es algo que frecuentemente requerimos de sistemas. Por ende si elegimos especificar un sistema $S$ dándole un participante abstracto $Spec$ al que $S$ debería ser equivalente, entonces $Spec$ va a ser determinista. La idea de \emph{Weak Determinacy} \cite[Def. 11.3]{comm} es una forma más amplia de determinismo que nos permite contemplar una equivalencia observacional entre participantes. Esta equivalencia basada en la definición de Bisimulación tiene en cuenta que para un participante externo, las transiciones $\epsilon$ son invisibles y por lo tanto una serie de estados conectados por transiciones consecutivas de este tipo serían bisimilares. En este caso tomamos las transiciones internas o de comunicacion interna como observacionalmente bisimilares para un participante externo.  

\begin{definition}[Determinismo] Decimos que un autómata es determinístico cuando cumple que no hay dos transiciones con la misma etiqueta que partan de un mismo estado y vayan a estados distintos. Es decir: 
sea un autómata $ \Lambda = \langle Q, \Sigma, \delta, q_0, F \rangle$ se cumple
$ \forall \  q_i, q_j, q_k \in Q_{j \neq k}, \  \nexists \ \delta_i, \delta_j \in \delta, t \in \Sigma \ \| \  \delta_1 = \langle q_i, t, q_j \rangle, \ \delta_2 = \langle q_i, t, q_k \rangle$ \\

La composición de estos autómatas preserva el determinismo. Esto es un resultado directo de que ambos autómatas no comparten acciones y de la definición de la composición de $\delta$.\end{definition}

\begin{definition}[Weak Determinacy] \cite{comm} Un autómata $A= \langle Q, \mathcal{C}, B, \Sigma, \delta, q_0, F \rangle$ es weak determinate si para todo estado $q \in Q$ y para toda cadena $s \in \Sigma^*$ siempre que $q \xrightarrow{s} q'$ y $q \xrightarrow{s} q''$, con $q',q'' \in Q$, entonces $q' \sim q''$.
\end{definition}


\begin{definition}[Cociente por comunicación interna]
Dado un autómata $A = \langle Q, \mathcal{C}, B, \Sigma, \delta, q_0, F \rangle$, weak deterministic. Dados dos estados $q, q' \in Q$ tales que $q \xrightarrow{\sigma^*} q'$ con $\sigma \in \Sigma_{\mathit{Int}}^*$ decimos que son bisimilares $q \approx q'$. Es decir que dada la condición de weak determinism, podemos afirmar que como las transiciones internas no afectan el contenido de los buffers, los estados son equivalentes para la comunicación. Llamamos $[q]_\mathcal{M}$ a la clase de equivalencia por comunicación interna de un estado $q$ y la definimos como $[q]_\mathcal{M}= \{q'| q \xrightarrow{\sigma^*} q'\}$. Para un estado $q$ del autómata, su clase de equivalencia por comunicación interna $[q]_\mathcal{M}$ son aquellos estados $q'$ alcanzables en cero o más pasos de transiciones internas. Decimos que dos clases de equivalencia están relacionadas $[q]_\mathcal{M} \xrightarrow{\sigma} [q']_\mathcal{M}$ sii $\exists q_i \in [q]_\mathcal{M}, q'_i \in [q']_\mathcal{M}$ tales que $q_i \xrightarrow{\sigma} q'_i \in \delta_{Buff}$
\end{definition}

\begin{definition}[Proyección de comunicación interna]\label{def:pci}
Dado un AFCA $A = \langle Q, B, \mathcal{C}, \Sigma, \delta, q_0, F \rangle$, definimos $\Tau(\mathcal{A})$ como la proyección de la semántica de comunicación interna de $\mathcal{A}$. Llamamos $\hat{\mathcal{A}}$ al sistema de transición etiquetado resultante y decimos que $\Tau (\langle Q, B, \mathcal{C}, \Sigma, q_0, \delta, F \rangle)= \langle \hat{Q}, B, \hat{q_0} \hat{\Delta} \rangle$ tal que:
\begin{itemize}
    \item $ \hat{Q} = \{ \langle [q]_m, \overrightarrow{\Omega} \rangle | q \in Q \land \overrightarrow{\Omega} = (\Omega_b)_{b \in B}$ con $\Omega_b \in \mathcal{M}^* \}$ Cada estado del sistema de transición resultante es una configuración de $\mathcal{A}$ para una clase de equivalencia por cociente de comunicación.
    
    \item $B$ es el conjunto de buffers del LTS, que son los mismos de comunicación interna de $\mathcal{A}$.
    
    \item $\hat{q_0}= \langle q_0, \Omega$ es el estado inicial de  donde $q_0$ es el estado inicial de $\mathcal{A}$ y $\Omega = \langle [], \ldots, [] \rangle$ es el conjunto de buffers en su estado inicial sin mensajes.
    
    \item $\hat{\Delta}= \{\langle \langle q_i, \Omega \rangle, \hat{\sigma},\langle q_j, \Omega' \rangle \rangle | q_i,q_j \in Q, \hat{\sigma} \in \{B \times \{\gg, \ll\} \times \mathcal{M}\} \land \exists \sigma \in \Sigma$ tal que  $[q_i]_m \xrightarrow{\sigma} [q_j]_m \}$. Definimos $\hat{\Delta}$ como la relación de transición de $\hat{\mathcal{A}}$. Un estado de $\hat{q_j}$ es alcanzable desde otro $\hat{q_i}$ si la clase de equivalencia $[q_j]_m$ es alcanzable desde $[q_i]_\mathcal{M}$
    
\end{itemize}
\end{definition}

\begin{theorem}[Equivalencia de las semánticas de las mFCSMs y los AFCAs] Sea $\{\mathcal{A}\}_{1 \leq i \leq n}$ un conjunto de weak deterministic AFCA sin transiciones de comunicación interna, tal que su composición $\mathcal{A} = \langle Q, B, \mathcal{C},\Sigma, \delta, q_0, F \rangle$, también lo es. Llamamos $\hat{\mathcal{A}}=\langle \hat{Q}, B, \hat{q_0}, \hat{\delta} \rangle$ al sistema de transición etiquetado (LTS) que resulta de la proyección $\Tau(\mathcal{A})$. Sea el conjunto $\{M_i\}_{1 \leq i \leq n} \langle [Q_i], \mathcal{C}, [{q_0}_i], [\delta_i] \rangle$ de conformado por las mCFSM correspondientes al conjunto AFCA. Llamamos $M = \langle [Q], \mathcal{C}, [{q_0}], [\delta] \rangle$ al LTS que representa la semántica del communicating system. Luego $\hat{\mathcal{A}}$ y $M$ son bisimilares. 
\end{theorem}

Ante de pasar a demostrar el teorema, enunciamos dos lemasque son necesarios.

\begin{lemma}[Equivalencia de estados discretos] Sean $\hat{\mathcal{A}}=\langle \hat{Q}, B, \hat{q_0}, \hat{\delta} \rangle$, la proyección de comunicación interna del AFCA $\mathcal{A}$ que surge de la composición del conjunto de autómatas  $\{\mathcal{A}\}_{1 \leq i \leq n}$  y $M= \langle [Q], \mathcal{C}, [q_0], [\delta] \rangle$, el communicating system correspondiente a las mCFSM del conjunto. Dados $\langle [\overrightarrow{q}]_m, \overrightarrow{\Omega} \rangle \in \hat{Q}$ y $\langle \overrightarrow{p}, \overrightarrow{\Omega_{\mathcal{C}}} \rangle \in P$. Vale $\{\langle [\overrightarrow{q}]_m, \overrightarrow{p} \rangle \ | \ q_i \in \overrightarrow{q} \iff q_i \in \overrightarrow{p} \}$. 
\end{lemma}

Podemos afirmar que esta propiedad se cumple porque estamos partiendo de un conjunto de AFCA que carecen transiciones de comunicación interna, solo externa, y al componerlos las acciones de comunicación externa se transforman en comunicación interna. Por lo tanto siempre vale que los estados discretos cocientados que aparecen en la proyección de comunicación interna del autómata compuesto son los mismos que en el autómata composición tienen aristas de comunicación interna, que en los autómatas individuales son comunicación externa y por lo tanto aparecen o tienen su equivalente en las mcfsms y aparecen en el lts del communicating system.

% Sea $\langle [\overrightarrow{q}]_m, \overrightarrow{\Omega} \rangle \in \hat{Q}$ donde $\langle [\overrightarrow{q}]_m, \overrightarrow{\Omega} \rangle$ es una configuración de $\mathcal{A}$ para una clase de equivalencia por cociente de comunicación de $\overrightarrow{q} \in Q$ y $\overrightarrow{\Omega}= (\Omega_b)_{b \in B} \land \Omega_b \in \mathcal{M}^*$. Es decir $[\overrightarrow{q}]_m = \{\overrightarrow{q}' \in Q | \overrightarrow{q} \xrightarrow{\sigma^*} \overrightarrow{q}' \land \sigma  \in  \Sigma_{Int}\}$ y $\overrightarrow{\Omega}$ son los buffers de $\mathcal{A}$ para el conjunto de estados de $[\overrightarrow{q}]_m$.
% \begin{proof}
% Por definición de cociente como sabemos que todos esos estados son alcanzables desde $\overrightarrow{q}$ por transiciones internas, los buffers se mantienen y por lo tanto son bisimilares entre sí a nivel comunicacional. Por definición de composición $\overrightarrow{q} \in Q \iff \overrightarrow{q} = \langle q_1, q_2,\ldots, q_i \rangle $donde $q_i \in Q_i$, es decir que cada $\overrightarrow{q}$ es una tupla de estados donde cada elemento es un estado de un AFCA componente. Como $\overrightarrow{q}$ pertenece a un Clase de Cociente por comunicación, sus estados componentes son estados que reflejan un cambio en la comunicación. Por lo tanto todo componente de $\overrightarrow{q}$ tiene su propia $[q_i]_m$. Luego, $q_i \in P_i$, donde $P_i$ es el conjunto de estados de su mCFSM correspondiente por Interfaz de Comunicación de AFCA. Y $q_i \in P_i \iff \langle \langle q_i,..,q_r \rangle, \overrightarrow{\Omega} \rangle \in P$ dónde $M$ es el LTS que representa la semántica del communicating system.
    
    
% Sea $M$ el LTS que representa la semántica el communicating system $\langle \langle q_1,..,q_i \rangle, \overrightarrow{\Omega} \rangle \in P  \iff q_i \in P_i \iff q_i \in Q_i$ y ya establecimos $\forall q_i \in  P_i$ vale que $\exists [q_i]_m$. Por definición de composición $q_i \in Q_i \iff \langle q_1, q_2,.., q_i \rangle \in Q $. Como $q_i$ tiene su Clase de Cociente por Comunicación, ese estado refleja un paso en la comunicación entre componentes. Por composición de AFCA ese cambio en la comunicación se refleja en los estados compuestos. Por lo tanto $\exists [q_i]_m$ para $\mathcal{A}_i$ $\iff \exists [\langle q_1, q_2,.., q_i \rangle]_m$ para $\mathcal{A}$. Luego por proyección $\langle [\langle q_1, q_2,.., q_i \rangle]_m, \overrightarrow{\Omega} \rangle \in \hat{Q}$
% \end{proof}

\begin{lemma}[Equivalencia entre buffers y canales]\label{lemma2} DSean $\hat{\mathcal{A}}=\langle \hat{Q}, B, \hat{q_0}, \hat{\delta} \rangle$, la proyección de comunicación interna del AFCA $\mathcal{A}$ que surge de la composición del conjunto de autómatas  $\{\mathcal{A}\}_{1 \leq i \leq n}$  y $M= \langle P, \mathcal{C}, [p_0], [\delta] \rangle$, el communicating system correspondiente a las mCFSM del conjunto. Dados $\langle [\overrightarrow{q}]_m, \overrightarrow{\Omega} \rangle \in \hat{Q}$ y $\langle \overrightarrow{p}, \overrightarrow{\Omega_{\mathcal{C}}} \rangle \in P$. Queremos ver que el conjunto de buffers $B$ de $\mathcal{A}$ es el conjunto de canales internalizados de $\mathcal{C}$ de $M$ y que el comportamiento en el mismo en ambos casos.
\end{lemma}
\begin{proof}
    \begin{itemize}    
        \item \textbf{$\overrightarrow{\Omega} = \overrightarrow{\Omega_\mathcal{C}}$}: Por definición de proyección de comunicación, los buffers representados en $\hat{\mathcal{A}}$ son buffers de $\mathcal{A}$. Por definición de composición, estos buffers surgen de los buffers $B_i$ de los AFCA componentes más los canales de comunicación externa internalizados. Es decir los buffers de tipo $(a_ka_j)_l \in \mathcal{C}_i$. Como los AFCA del conjunto $\{\mathcal{A}_i\}_{i \in I}$ carecen de operaciones de comunicación interna, podemos afirmar que tampoco poseen buffers propios. Por lo tanto el conjunto de buffers $B$ se compone exclusivamente de los de canales internalizados. Por definición de interfaz de comunicación de AFCA, son los mismos canales que aparecen en las mCFSM $M_i$ y por lo tanto son los mismos que aparecen en los estados del lts $M$.

        \item \textbf{$\overrightarrow{\Omega_\mathcal{C}} = \overrightarrow{\Omega}$} Los $\Omega$ que se modifican entre estados de $M$ son canales de comunicación externa del conjunto de mCFSM $\{M_i\}_{i \in I}$. Para toda mCFSM $M_i$ del conjunto, un canal de comunicación está pertenece a su cojunto de canales $\mathcal{C}_{M_i}$ si y solo si, son canales en el AFCA correspondiente. Por definición de composición estos canales se internalizan como buffers de comunicación interna en $\mathcal{A}$. Por definición de proyección de comunicación interna, estos buffers están en los estados del lts $\hat{\mathcal{A}}$. Por lo tanto los canales de los estados de $M$ son exactamente los mismos que se modifican en los estados de $\hat{\mathcal{A}}$.
    \end{itemize}
\end{proof}


\begin{proof}
Queremos probar que la relación $R \subseteq \hat{Q} \times P$, definida de la siguiente manera $R = \{ \langle \langle [\overrightarrow{q}]_m, \overrightarrow{\Omega} \rangle, \langle \overrightarrow{p}, \overrightarrow{\Omega_{\mathcal{C}}} \rangle \rangle \ | \ [\overrightarrow{q}]_mR'\overrightarrow{p} \land  \overrightarrow{\Omega}=\overrightarrow{\Omega_{\mathcal{C}}} \}$ es una bisimulación, con $R'=\{\langle [\overrightarrow{q}]_m, \overrightarrow{p} \rangle \ | \ q_i \in \overrightarrow{q} \iff q_i \in \overrightarrow{p} \}$. Donde $P$ es el conjunto de estados de $M$ y para todo $\langle [\overrightarrow{q}]_m, \overrightarrow{\Omega} \rangle \in \hat{Q}$, $\langle \overrightarrow{p}, \overrightarrow{\Omega_{\mathcal{C}}} \rangle \in P$, $\langle \langle [\overrightarrow{q}]_m, \overrightarrow{\Omega} \rangle, \langle \overrightarrow{p}, \overrightarrow{\Omega_{\mathcal{C}}} \rangle \rangle \in R$ si y solo si $[\overrightarrow{q}]_mR'\overrightarrow{p}$ y $\overrightarrow{\Omega}=\overrightarrow{\Omega_{\mathcal{C}}}$. \\

Es decir que para cada par de elementos $\hat{q} \in \hat{Q}$, $p \in P$, vale $\langle \hat{q}, p \rangle \in R$ y $\forall \hat{\sigma} \in \hat{\Sigma} | \hat{q} \xrightarrow{\hat{\sigma}} \hat{q}' \implies \exists p' \in P$ tal que $p \xrightarrow{\sigma_M} p' \land \langle \hat{q}', p', \rangle \in R$, y simétricamente $\forall \sigma_M \in \Sigma_M | p \xrightarrow{\sigma_M} p \implies \exists \hat{q}' \in \hat{Q}, \hat{\sigma} \in \hat{\Sigma}$ tal que $\hat{q} \xrightarrow{\hat{\sigma}} \hat{q}' \land \langle \hat{q}', p'\rangle \in R$. Donde $\hat{\Sigma} \subseteq B \times \{\gg,\ll\} \times \mathcal{M}$, $\Sigma_M \subseteq \mathcal{A} \times \{!,?\} \times \mathcal{M}$ y $B = \mathcal{C}$ por el lema \ref{lemma2}.

% que partimos de que los AFCA del conjunto $\{\mathcal{A}_i\}_{1 < i < n}$ no tienen acciones de comunicación interna ni buffers propios, y los buffers del AFCA compuesto $\mathcal{A}$ son los canales de comunicación externa internalizados.

\begin{itemize}
    \item[] Tomamos $\hat{\sigma} = (a_ka_j)_l \gg m$ y $\sigma_M = (a_ka_j)_l?m$. Queremos ver que vale $\forall (a_ka_j)_l \gg m \in \hat{\Sigma} | \hat{q} \xrightarrow{(a_ka_j)_l \gg m} \hat{q}' \implies \exists p' \in P$ tal que $p \xrightarrow{(a_ka_j)_l?m} p' \land \langle \hat{q}', p', \rangle \in R$. 
 
    \item[i] Dado el par $\langle [\overrightarrow{q}]_m, \overrightarrow{\Omega} \rangle \in \hat{Q}, \langle \overrightarrow{p}, \overrightarrow{\Omega_\mathcal{C}} \rangle \in P$ tal que $\langle \langle [\overrightarrow{q}]_m, \overrightarrow{\Omega} \rangle, \langle \overrightarrow{p}, \overrightarrow{\Omega_\mathcal{C}} \rangle \rangle \in R$. Sean $(a_ka_j)_l \gg m$ y $\langle [\overrightarrow{q}']_m, \overrightarrow{\Omega}' \rangle \in \hat{Q}$ tales que $\langle [\overrightarrow{q}]_m, \overrightarrow{\Omega} \rangle \xrightarrow{(a_ka_j)_l \gg m} \langle [\overrightarrow{q}']_m, \overrightarrow{\Omega}' \rangle \in \hat{\Delta}$ por definición de proyección de comunicación vale que $(a_ka_j)_l \gg m \in \Sigma_{Buff}$.  Esto vale si y solo si $[\overrightarrow{q}]_m \xrightarrow{(a_ka_j)_l \gg m} [\overrightarrow{q}']_m \land \overrightarrow{\Omega} = [(a_1a_2)_1, \ldots, (a_ka_j)_l \cdot m \ldots (a_na_m)_l] \land \overrightarrow{\Omega}' = [(a_1a_2)_1, \ldots, (a_ka_j)_l \ldots (a_na_m)_l] \rangle \land \overrightarrow{q} \in [\overrightarrow{q}]_m \land \overrightarrow{q}' \in [\overrightarrow{q}]_m$.
    
    \item[ii] Por definición de cociente por comunicación sabemos que todos los estados en las clases de equivalencia son bisimilares para la comunicación. Sean $\overrightarrow{q_{\sigma}} \in [\overrightarrow{q}]_m, \overrightarrow{q_{\sigma}}' \in [\overrightarrow{q}']_m$, por definición de proyección de comunicación vale $\overrightarrow{q_{\sigma}} \xrightarrow{(a_ka_j)_l \gg m} \overrightarrow{q_{\sigma}}' \in \delta$. Por definición de composición sabemos que vale $\overrightarrow{q_{\sigma}} =\langle q_0, q_1, \ldots, q_j, \ldots,q_n \rangle$ y $\overrightarrow{q_{\sigma}}' =\langle q_0, q_1, \ldots, q'_j, \ldots,q_n \rangle$ donde $q_j,q'_j$ son estados de $\mathcal{A}_j$, componente de $\mathcal{A}$. Por definición de AFCA, $(a_ka_j)_l \gg m$ es una operación de desencolamiento del mensaje $m$ sobre el buffer $(a_ka_j)_l$. Por definición de composición esto corresponde a una recepción de un mensaje por parte de $\mathcal{A}_j$ vía el canal $(a_ka_j)_l$. Entonces podemos afirmar que $\overrightarrow{q_{\sigma}} \xrightarrow{(a_ka_j)_l \gg m} \overrightarrow{q_{\sigma}}'\in \delta \iff q_j,q'_j \in Q_j \land q_j \xrightarrow{(a_ka_j)_l?m} q'_j \in \delta_j$.
    
    \item[iii] Por la interfaz de comunicación de AFCA podemos obtener $M_j$, la mCFSM correspondiente a $\mathcal{A}_j$, y sabemos que $q_j \xrightarrow{(a_ka_j)_l?m} q'_j \in \delta_j \iff$ se cumple que $[q_j], [q'_j] \in [Q_j] \land [q_j] \xrightarrow{(a_ka_j)_l?m} [q'_j] \in [\delta_j]$. 
    
    \item[iv] Entonces, por definición del LTS M, existe $p'=\langle \overrightarrow{p'}, \overrightarrow{\Omega_\mathcal{C}} \rangle \in P$ tal que $\overrightarrow{p'}=\langle [q_1], [q_2], \ldots, [q'_j], \ldots [q_n] \rangle$ y $\overrightarrow{\Omega'_\mathcal{C}}=[\ldots, (a_ka_j)_l, \ldots]$, y $p \xrightarrow{(a_ka_j)_l?m} p'$. Por lo tanto, por los lemas 1 y 2, vale $\langle \hat{q'}, p' \rangle \in R$.
    
    % Sabemos que $\langle \langle [\overrightarrow{q}]_m,  \overrightarrow{\Omega} \rangle, \langle \overrightarrow{p}, \overrightarrow{\Omega_\mathcal{C}} \rangle \rangle \in R$. Entonces vale que $[\overrightarrow{q}]_mR'\overrightarrow{p}$, y $\overrightarrow{p} = \langle [q_1], [q_2], \ldots [q_j], \ldots [q_n] \rangle$, y, como vale $\overrightarrow{\Omega}=\overrightarrow{\Omega_\mathcal{C}}$, $\overrightarrow{\Omega_\mathcal{C}} = [\ldots, (a_ka_j)_l \cdot m, \ldots]$. Entonces  por definición del lts $M$ vale $[q_j] \xrightarrow{(a_ka_j)_l?m} [q'_j] \in [\delta_j]  \iff (\exists \langle \overrightarrow{p}', \overrightarrow{\Omega_\mathcal{C}}' \rangle \in P$ tal que $\langle \overrightarrow{p}, \overrightarrow{\Omega_\mathcal{C}} \rangle \xrightarrow{(a_ka_j)_l?m} \langle \overrightarrow{p}', \overrightarrow{\Omega_\mathcal{C}}' \rangle \in \delta_M \land \overrightarrow{p}'= \langle [q_1], [q_2], \ldots [q'_j], \ldots [q_n] \rangle \land \overrightarrow{\Omega_\mathcal{C}}'= [\ldots, (a_ka_j)_l, \ldots])$. 
    
    \item[v] Por lo tanto vale que $\forall (a_ka_j)_l \gg m \in \hat{\Sigma} | \hat{q} \xrightarrow{(a_ka_j)_l \gg m} \hat{q}' \implies \exists p' \in P$ tal que $p' \xrightarrow{(a_ka_j)_l?m} p' \land \langle \hat{q}', p', \rangle \in R$ donde $\hat{\Sigma} \subseteq B \times \{\gg,\ll\} \times \mathcal{M}$
    
    \item[] Queremos demostrar la condición simétrica $\forall (a_ka_j)_l?m \in \Sigma_M | p \xrightarrow{(a_ka_j)_l?m} p \implies \exists \hat{q}' \in \hat{Q}, (a_ka_j)_l \gg m \in \hat{\Sigma}$ tal que $\hat{q} \xrightarrow{(a_ka_j)_l \gg m} \hat{q}' \land \langle \hat{q}', p'\rangle \in R$
    
    \item[i]Dados $\langle [\overrightarrow{q}]_m, \overrightarrow{\Omega} \rangle \in \hat{Q}, \langle \overrightarrow{p}, \overrightarrow{\Omega_\mathcal{C}} \rangle, \langle \overrightarrow{p}', \overrightarrow{\Omega_\mathcal{C}}' \rangle \in P$ tales que $\langle \langle [\overrightarrow{q}]_m, \overrightarrow{\Omega} \rangle, \langle \overrightarrow{p}, \overrightarrow{\Omega_\mathcal{C}} \rangle \rangle \in R$. Dado $\langle \overrightarrow{p}, \overrightarrow{\Omega_\mathcal{C}} \rangle \xrightarrow{(a_ka_j)_l?m} \langle \overrightarrow{p}', \overrightarrow{\Omega_\mathcal{C}}' \rangle \in \delta_M$. Por definición del LTS $M$ vale $\overrightarrow{\Omega_{\mathcal{C}}}= \langle \ldots,(a_ka_j)_l \cdot m, \ldots \rangle$ y $\overrightarrow{\Omega_{\mathcal{C}}}'= \langle \ldots,(a_ka_j)_l, \ldots \rangle$.
    
    \item[ii] Por definición del LTS $M$, $\overrightarrow{p}=\langle [q_1], [q_2], \ldots [q_j], \ldots [q_n] \rangle$, $\overrightarrow{p}'= \langle [q_1], [q_2], \ldots, [q'_j], \ldots [q_n] \rangle$, y $(a_ka_j)_l?m$ es una operación sobre un canal de la mCFSM $M_j$. Entonces vale que $[q_j],[q'_j] \in [Q_j]$ y $[q_j] \xrightarrow{(a_ka_j)_l?m} [q'_j] \in [\delta]_j$
    
    \item[iii] Por definición de intefaz de comunicación de AFCA sabemos que $M_J$ es la mCFSM correspondiente a $\mathcal{A}_j$, componente de $\mathcal{A}$. Por $[\overrightarrow{q}]_mR'\overrightarrow{p}$, $[\overrightarrow{q}]_mR'\overrightarrow{p}$, como vale $[q_j] \xrightarrow{(a_ka_j)_l?m} [q'_j] \in [\delta_j]$ vale que $q_j \xrightarrow{(a_ka_j)_l?m} q'_j \in \delta_j$. 
    
    \item[iv] Por definición de composición $q_j$ y $q'_j$ son componentes de dos estaos compuestos que llamamos $\overrightarrow{q}_{\sigma}$ y $\overrightarrow{q}_{\sigma}'$, respectivamente. De modo tal que $ \overrightarrow{q}_{\sigma} \xrightarrow{(a_ka_j)_l \gg m} \overrightarrow{q}_{\sigma}' \in \delta$. Donde $(a_ka_j)_l$ es el canal internalizado en forma de buffer.
    
    \item[v] Por definición de cociente por comunicación $\overrightarrow{q}_{\sigma} \xrightarrow{(a_ka_j)_l \gg m} \overrightarrow{q}_{\sigma}' \in \delta \iff (\overrightarrow{q}_{\sigma} \in [\overrightarrow{q}]_m \land \overrightarrow{q}_{\sigma}' \in [\overrightarrow{q}']_m$ y vale $[\overrightarrow{q}]_m \xrightarrow{(a_ka_j)_l \gg m} [\overrightarrow{q}']_m$. Por proyección de comunicación entonces vale $\langle [\overrightarrow{q}]_m, \overrightarrow{\Omega} \rangle, \langle [\overrightarrow{q}']_m, \overrightarrow{\Omega}' \rangle \in \hat{Q}$ donde $\overrightarrow{\Omega}$ y $\overrightarrow{\Omega}'$ son los buffers de $\mathcal{A}$ en el estado $\overrightarrow{q}_{\sigma}$ y $\overrightarrow{q}_{\sigma}'$ respectivamente. Sabemos por definición de cociente de comunicación interna que, al ser $\{\mathcal{A}\}$ y sus componentes weak deterministic, todos los estados que están en una clase de equivalencia son bisimilares para la comunicación, o sea que los buffers tienen el mismo contenido. Además vale que $\langle [\overrightarrow{q}]_m, \overrightarrow{\Omega} \rangle \xrightarrow{(a_ka_j)_l \gg m} \langle [\overrightarrow{q}']_m, \overrightarrow{\Omega}' \rangle \in \hat{\Delta}$.
        
    \item[vi]  Sabemos que $\langle \langle [\overrightarrow{q}]_m, \overrightarrow{\Omega} \rangle, \langle \overrightarrow{p}, \overrightarrow{\Omega_\mathcal{C}} \rangle \rangle \in R$. Entonces vale que $[\overrightarrow{q}]_mR'\overrightarrow{p}$, y $[\overrightarrow{q}]_m = \langle \overrightarrow{q_1}, \ldots \overrightarrow{q_\sigma}, \ldots \overrightarrow{q_n} \rangle$, y, como vale $\overrightarrow{\Omega}=\overrightarrow{\Omega_\mathcal{C}}$, $\overrightarrow{\Omega} = [\ldots, (a_ka_j)_l \cdot m, \ldots]$. Ya vimos que vale $q_j \xrightarrow{(a_ka_j)_l?m} q'_j \in \delta_j  \iff  (\langle [\overrightarrow{q}']_m, \overrightarrow{\Omega}' \rangle \in \hat{Q}$ tal que $\langle [\overrightarrow{q}]_m, \overrightarrow{\Omega} \rangle \xrightarrow{(a_ka_j)_l \gg m} \langle [\overrightarrow{q}']_m, \overrightarrow{\Omega}' \rangle \in \hat{\Delta} \land [\overrightarrow{q}']_m= \langle \overrightarrow{q_1}, \ldots \overrightarrow{q_\sigma}', \ldots \overrightarrow{q_n} \rangle \land \overrightarrow{\Omega}'= [\ldots, (a_ka_j)_l, \ldots])$.
    
    \item[vii] Por lo tanto vale en forma simétrica que $\forall (a_ka_j)_l?m \in \Sigma_M | p \xrightarrow{(a_ka_j)_l?m} p \implies \exists \hat{q}' \in \hat{Q}, (a_ka_j)_l \gg m \in \hat{\Sigma}$ tal que $\hat{q} \xrightarrow{(a_ka_j)_l \gg m} \hat{q}' \land \langle \hat{q}', p'\rangle \in R$ donde $\Sigma_M \subseteq \mathcal{C} \times \{!,?\} \times \mathcal{M}$
    
\end{itemize}
\end{proof}


\begin{figure}[h]
\begin{example}[Transformación de Autómatas Finitos de Comunicación Asincrónica]
\label{ex:Transformación}
Consideremos los siguientes AFCA 
\begin{center}
\begin{tikzpicture}[->, thick]
 \node[state,initial] (q_0)   {$q_0$}; 
 \node[state] (q_1) [right= 1.5cm of q_0 ] {$q_1$};
 \node[state] (q_2) [below= of q_0 ] {$q_2$};
 \node[state] (q_3) [right= 1.5cm of q_2 ] {$q_3$};
 \node[state, accepting] (q_4) [below= of q_3 ] {$q_4$};
	
 \draw[]        
        (q_0) edge[above] node{out(pr,a)} (q_1)
        (q_0) edge[left] node{in(sp,b)} (q_2)
        (q_1) edge[right] node{in(sp,b)} (q_3)
        (q_2) edge[above] node{out(pr,a)} (q_3)
        (q_3) edge[right] node{$int_p$} (q_4)
        ; 
\end{tikzpicture} 
\qquad
\begin{tikzpicture}[->, thick]
 \node[state,initial] (q_0)   {$q_0$}; 
 \node[state] (q_1) [below= of q_0 ] {$q_1$};
 \node[state, accepting] (q_2) [below= of q_1 ] {$q_2$};
 \draw[]        
        
        (q_0) edge[right] node{$int_r$} (q_1)
        (q_1) edge[right] node{in(pr,a)} (q_2)
        ;
\end{tikzpicture} 
\qquad
\begin{tikzpicture}[->, thick]
 \node[state,initial] (q_0)   {$q_0$}; 
 \node[state] (q_1) [below= of q_0 ] {$q_1$};

 \draw[]        
        
        (q_0) edge[right] node{out(sp,b)} (q_1)
        
        ;
\end{tikzpicture} 

\end{center}

El autómata compuesto resultante es el AFCA $\mathcal{A}$  \\  
\begin{tikzpicture}[->, thick]
 \node[state,initial] (q_0)   {$q_{000}$}; 
 \node[state] (q_1) [right= 1.5cm of q_0 ] {$q_{100}$};
 \node[state] (q_2) [right = 1.5 of q_1 ] {$q_{110}$};
 \node[state] (q_3) [right = 1.5 of q_2 ] {$q_{120}$};
  \node[state] (q_4) [right = 1.5 of q_3 ] {$q_{121}$};
  \node[state] (q_5) [below=  of q_0] {$q_{001}$};
  \node[state] (q_6) [right = 1.5cm of q_5] {$q_{201}$};
  \node[state] (q_7) [right = 1.5 of q_6 ] {$q_{301}$};
  \node[state] (q_8) [right = 1.5 of q_7 ] {$q_{311}$};
  \node[state] (q_9) [right = 1.5 of q_8 ] {$q_{321}$};
 \node[state,accepting] (q_10) [right= 1.5cm of q_9 ] {$q_{421}$};
 \draw[]        
        (q_0) edge[above] node{$PR \ll a$} (q_1)
        (q_1) edge[above] node{$int_r$} (q_2)
		(q_2) edge[above] node{$PR \gg a$} (q_3)
        (q_3) edge[above] node{$SP \ll b$} (q_4) 
        (q_4) edge[right] node{$SP \gg b$} (q_9)
        (q_0) edge[left] node{$SP \ll b$} (q_5)
        (q_5) edge[below] node{$SP \gg b$} (q_6)
        (q_6) edge[below] node{$PR \ll a$} (q_7)
        (q_7) edge[below] node{$int_r$} (q_8)
        (q_8) edge[below] node{$PR \gg a$} (q_9)
        (q_9) edge[below] node{$int_p$} (q_10)
        ;
\end{tikzpicture}
    \caption{Transformación}
    \label{fig:transformacion}
\end{example}
\end{figure}


\begin{figure}[h]
\begin{example}[communicating system de transformación]
\label{ex:cmst} El communicating system correspondiente al conjunto de autómatas es el siguiente
\centering

\begin{tikzpicture}[->, thick]
 \node[state,initial] (q_0)   {$q_0$}; 
 \node[state] (q_1) [right= of q_0 ] {$q_1$};
  \node[state] (q_2) [below= of q_0 ] {$q_2$};
 \node[state] (q_3) [right= of q_2 ] {$q_3$};

 \draw[]        
        (q_0) edge[above] node{pr!a} (q_1)
        (q_0) edge[right] node{sp?b} (q_2)
        (q_1) edge[right] node{sp?b} (q_3)
        (q_2) edge[above] node{pr!a} (q_3)
        ; 
\end{tikzpicture} 
\qquad
\begin{tikzpicture}[->, thick]
 \node[state,initial] (q_0)   {$q_0$}; 
 \node[state] (q_1) [below= of q_0 ] {$q_2$};

 \draw[]        
        
        (q_0) edge[right] node{pr?a} (q_1)
        
        ;
\end{tikzpicture} 
\qquad
\begin{tikzpicture}[->, thick]
 \node[state,initial] (q_0)   {$q_0$}; 
 \node[state] (q_1) [below= of q_0 ] {$q_1$};

 \draw[]        
        
        (q_0) edge[right] node{sp!b} (q_1)
        
        ;
\end{tikzpicture} 

La semántica del communicating system es el sistema de transición etiquetado $\mathcal{L}$
\\
\begin{tikzpicture}[->, thick][scale=0.20]
 \node[state,initial] (q_0)   {$q_{000},\epsilon,\epsilon$}; 
 \node[state] (q_1) [right= 3cm of q_0 ] {$q_{100},a, \epsilon$};
 \node[state] (q_2) [right = 3cm of q_1 ] {$q_{120},\epsilon, \epsilon$};
 \node[state] (q_3) [below = of q_2 ] {$q_{121}, \epsilon,b$};
 \node[state,accepting] (q_4) [below=  of q_3] {$q_{321}, \epsilon,\epsilon$};
 \node[state] (q_5) [below = of q_0] {$q_{001}, \epsilon, b$};
 \node[state] (q_6) [below = of q_5 ] {$q_{201}, \epsilon, \epsilon$};
 \node[state] (q_7) [right= 3cm of q_6 ] {$q_{301}, a, \epsilon$};
 \draw[]        
        (q_0) edge[above] node{$\langle q_{0p},PR!a,q_{1p} \rangle$} (q_1)
        (q_1) edge[above] node{$\langle q_{0r},PR?a,q_{1r} \rangle$} (q_2)
        (q_2) edge[right] node{$\langle q_{0s},SP!b,q_{1s} \rangle$} (q_3)
        (q_3) edge[right] node{$\langle q_{1p},SP?b,q_{3p} \rangle$} (q_4)
        (q_0) edge[left] node{$\langle q_{0s},SP!b,q_{1s} \rangle$} (q_5)
        (q_5) edge[left] node{$\langle q_{0p},SP?b,q_{2p} \rangle$} (q_6)
        (q_6) edge[below] node{$\langle q_{2p},PR!a,q_{3p} \rangle$} (q_7)
        (q_7) edge[below] node{$\langle q_{0r},PR?a,q_{1r} \rangle$} (q_4)
        ;
\end{tikzpicture}
    \caption{Transformación}
    \label{fig:semántica mcfsm}
\end{example}
\end{figure}

\begin{figure}[ht]
    \begin{centering}
      \begin{tikzpicture}[->, thick]
         \node[state,initial] (q_0) {$[q_{000}],\epsilon, \epsilon$}; 
         \node[state] (q_1) [right= 3.5cm of q_0 ] {$[q_{100}], a, \epsilon$};
         \node[state] (q_2) [right = 3.5cm of q_1 ] {$[q_{120}],\epsilon,\epsilon$};
         \node[state] (q_3) [below= of q_2 ] {$[q_{121}],\epsilon, b$};
         \node[state] (q_4) [below=  of q_0] {$[q_{001}],\epsilon, b$};
         \node[state] (q_5) [below = of q_4] {$[q_{201}],\epsilon, \epsilon$};
         \node[state] (q_6) [right = 3.5cm of q_5 ] {$[q_{301}], a, \epsilon$};
         \node[state,accepting] (q_7) [below = of q_3 ] {$[q_{321}],\epsilon, \epsilon$};
    \draw[]        
         (q_0) edge[above] node{$\langle q_{000}, PR \ll a, q_{100} \rangle$} (q_1)
    	 (q_1) edge[above] node{$\langle q_{110}, PR \gg a, q_{120} \rangle$} (q_2)
         (q_2) edge[right] node{$\langle q_{120}, SP \ll b, q_{121} \rangle$} (q_3) 
         (q_3) edge[right] node{$\langle q_{121}, SP \gg b, q_{321} \rangle$} (q_7)
         (q_0) edge[left] node{$\langle q_{000}, SP \ll b, q_{001} \rangle$} (q_4)
         (q_4) edge[left] node{$\langle q_{001}, SP \gg b, q_{201} \rangle$} (q_5)
         (q_5) edge[below] node{$\langle q_{201}, SP \gg b, q_{301} \rangle$} (q_6)
        (q_6) edge[below] node{$\langle q_{311}, PR \ll a, q_{321} \rangle$} (q_7)
         ;
        \end{tikzpicture}
    \caption{Caption}
    \label{fig: transformación loca}
 \end{centering}
\end{figure}

\newpage


